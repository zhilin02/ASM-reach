\documentclass[preprint,12pt]{elsarticle}

%% Use the option review to obtain double line spacing
%% \documentclass[authoryear,preprint,review,12pt]{elsarticle}

%% Use the options 1p,twocolumn; 3p; 3p,twocolumn; 5p; or 5p,twocolumn
%% for a journal layout:
%% \documentclass[final,1p,times]{elsarticle}
%% \documentclass[final,1p,times,twocolumn]{elsarticle}
%% \documentclass[final,3p,times]{elsarticle}
%% \documentclass[final,3p,times,twocolumn]{elsarticle}
%% \documentclass[final,5p,times]{elsarticle}
%% \documentclass[final,5p,times,twocolumn]{elsarticle}

%% For including figures, graphicx.sty has been loaded in
%% elsarticle.cls. If you prefer to use the old commands
%% please give \usepackage{epsfig}

% If the frontmatter runs over more than one page
% use the longmktitle option.

%\documentclass[a4paper,fleqn,longmktitle]{cas-sc}

%\usepackage[numbers]{natbib}
%\usepackage[authoryear]{natbib}
%\usepackage[authoryear,longnamesfirst]{natbib}
\usepackage{hyperref}
\usepackage{latexsym}
\usepackage{setspace}
\usepackage{cancel}
\usepackage{graphicx}
\usepackage{appendix}
\usepackage{amssymb}
%\usepackage{dsfont}
\usepackage{amsmath}
\usepackage{verbatim}
\usepackage{chngpage}
%\usepackage{fullpage}
\usepackage{color}
\usepackage{mathrsfs}
\usepackage{url}
%\usepackage{stmaryrd}
\usepackage{mathtools}
\usepackage{textcomp}
%\usepackage{circle}
\usepackage{wrapfig}
\usepackage{subcaption}
\usepackage{tikz}
\usepackage{enumitem}
\usepackage{array}
\setlistdepth{9}


\setlist[itemize,1]{label=$\bullet$}
\setlist[itemize,2]{label=$-$}
\setlist[itemize,3]{label=$\ast$}
\setlist[itemize,4]{label=$\cdot$}
\setlist[itemize,5]{label=$\diamond$}
\setlist[itemize,6]{label=$\circ$}
\setlist[itemize,7]{label=$\star$}
\setlist[itemize,8]{label=$\triangleright$}
\setlist[itemize,9]{label=$\square$}

\renewlist{itemize}{itemize}{9}
\makeatletter
\newcommand\mathcircled[1]{%
  \mathpalette\@mathcircled{#1}%
}
\newcommand\@mathcircled[2]{%
  \tikz[baseline=(math.base)] \node[draw,circle,inner sep=0.2pt] (math) {$\m@th#1#2$};%
}

\newcommand{\hcancel}[1]{%
    \tikz[baseline=(tocancel.base)]{
        \node[inner sep=0pt,outer sep=0pt] (tocancel) {#1};
        \draw[black] (tocancel.south east) -- (tocancel.north west);
    }%
}%

\usetikzlibrary{shapes,snakes}
\newcommand\rectangled[1]{\tikz[baseline=(torect.base)]{
    \node[shape = rectangle, draw, inner sep=0pt, outer sep = 0pt] (torect) {#1}
    }
}

\newcommand{\mrectangled}[1]{\text{\rectangled{\ensuremath{#1}}}}
\newcommand{\mhcancel}[1]{\mrectangled{#1}}

\makeatother



%%%Author macros
%\def\tsc#1{\csdef{#1}{\textsc{\lowercase{#1}}\xspace}}
%\tsc{WGM}
%\tsc{QE}
%%%

% Uncomment and use as if needed
%\newtheorem{theorem}{Theorem}
%\newtheorem{lemma}[theorem]{Lemma}
%\newdefinition{rmk}{Remark}
\newproof{proof}{Proof}
%\newproof{pot}{Proof of Theorem \ref{thm}}
\newtheorem{definition}{Definition}
\newtheorem{theorem}{Theorem}
\newtheorem{proposition}[theorem]{Proposition}
\newtheorem{corollary}[theorem]{Corollary}
\newtheorem{example}{Example}
\newtheorem{question}{Open Question}
\newtheorem{lemma}[theorem]{Lemma}
\newtheorem{Algo}{Algorithm}
\newtheorem{remark}[theorem]{Remark}

\definecolor{highlightcolor}{rgb}{0.9,0.9,0.9}

% correct bad hyphenation here
\hyphenation{op-tical net-works semi-conduc-tor}

\newcommand\Aa{{\mathcal{A} }}
\newcommand\Bb{{\mathcal{B} }}
\newcommand\Cc{{\mathcal{C} }}
% \newcommand\Dd{{\mathbb{D} }}
\newcommand\Ee{{\mathcal{E} }}
\newcommand\Ff{{\mathcal{F} }}
\newcommand\Gg{{\mathcal{G} }}
\newcommand\Jj{{\mathcal{J}}}
\newcommand\Ll{{\mathcal{L}}}
\newcommand\Mm{{\mathcal{M} }}
\newcommand\Nn{{\mathbb{N} }}
\newcommand\Pp{{\mathcal{P} }}
\newcommand\Qq{{\mathcal{Q} }}
\newcommand\Dd{{\mathcal{D} }}
\newcommand\Ss{{\mathcal{S} }}
\newcommand\Tt{{\mathcal{T} }}
\newcommand\transet{{\mathscr{T} }}
%\def\Ii{{\mathbb{Z} }}
\newcommand\Rr{{\mathcal{R} }}
\newcommand\Vv{{\mathcal{V}}}
\newcommand\Zz{{\mathcal{Z} }}

\newcommand\act{{\sf Act}}
\newcommand\aft{{\sf Aft}}
\newcommand\lmd{{\sf Lmd}}
\newcommand\atr{{\sf Atr}}
\newcommand\tact{{\sf top}}
\newcommand\bact{{\sf btm}}
\newcommand\ract{ \bact}
\newcommand\standard{{\sf STD}}
\newcommand\singletop{{\sf STP}}
\newcommand\singletask{{\sf STK}}
\newcommand\singleinstance{{\sf SIT}}

\newcommand\ntkflag{{\sf NTK}}
\newcommand\ctpflag{{\sf CTP}}
\newcommand\stpflag{{\sf STP}}
\newcommand\ctkflag{{\sf CTK}}
\newcommand\mtkflag{{\sf MTK}}
\newcommand\rtfflag{{\sf RTF}}
\newcommand\tohflag{{\sf TOH}}

\newcommand{\AMASS}{\textsf{ASM}}
\newcommand{\LMAMASS}{\textsf{ASM}_\textsf{LM}}
\newcommand{\IFAMASS}{\textsf{ASM}_\textsf{IF}}
\newcommand{\AMOS}{\textsf{AMOS}}
\newcommand{\ASST}{\textsf{ASST}}
\newcommand{\DNSS}{\textsf{DNSS}}
\newcommand\flagset{\mathcal{F}}
\newcommand\bool{\mathbb{B}}
\newcommand\Int{\mathbb{Z} }


\newcommand\actsig{{\sf Sig}}

\newcommand\mainact{{\sf mainAct}}

\newcommand\true{{\mathsf{true} }}

\newcommand\false{{\mathsf{false} }}

\newcommand\instr{{\mathsf{Inst} }}

\newcommand\back{{\mathsf{back} }}

\newcommand\startactivity{{\mathsf{start} }}

\newcommand\switchto{{\mathsf{switchTo} }}

\newcommand\tasks{{\mathsf{Tasks} }}

\newcommand\dom{{\mathsf{dom} }}

\newcommand\range{{\mathsf{rng} }}

\newcommand\confs{{\mathsf{Confs} }}

\newcommand\reparent{{\mathsf{Rpt} }}

\newcommand\benign{{\mathsf{bgn} }}

\newcommand\malicious{{\mathsf{mlc} }}

\newcommand\conf{{\mathsf{Conf} }}

\newcommand\pre{{\mathsf{pre} }}

\newcommand\post{{\mathsf{post} }}
\newcommand\Post{{\mathsf{Post} }}

\newcommand\abs{{\mathsf{Abs} }}
\newcommand\del{{\mathsf{Del} }}
\newcommand\getabs{{\mathsf{GetAbs} }}
\newcommand\deduplicate{{\mathsf{Deduplicate} }}

\newcommand\sfx{{\mathsf{Sfx} }}

\newcommand\jump{\mathsf{Jump}}


\newcommand{\hide}[1]{}

\newcommand\natnum{{\mathbb{N} }}
\newcommand\intnum{{\mathbb{Z} }}


\newcommand{\ifz}{{\sf ifz}}
\newcommand{\inc}{{\sf inc}}
\newcommand{\dec}{{\sf dec}}


%\newcommand\pop{{\sf pop}}
%\newcommand\mtrans{{\sf MTrans}}

\newcommand{\enc}{{\sf enc}}

\newcommand{\art}{{\sf Art}}

\newcommand{\eact}{{\sf EAct}}



\newcommand{\myvec}[1]{{\footnotesize{\begin{bmatrix}#1\end{bmatrix}}}}

\newcommand{\emptybackstack}{\triangleright}
\newcommand{\noaction}{\square}

\newcommand\namespace{\mathscr{N}}
\newcommand\namefun{\mathcal{N}}
\newcommand\aname{\mathfrak{n}}
\newcommand\AutReach{\mathscr{R}}

\newcommand\Succ{\mathsf{succ}}

\newcommand\RConfs{\mathsf{RConfs}}
\newcommand\RLang{\mathsf{RRel}}
\newcommand\RRel{\mathsf{RRel}}
\newcommand\Rel{\mathsf{Rel}}

\newcommand\WLang{\mathsf{WRel}}

\newcommand{\sat}{{\sf Sat}}
\newcommand{\sub}{{\sf Sub}}
\newcommand{\suffix}{{\sf Suffix}}

\newcommand\reach{{\sf Reach}}
\newcommand\init{{\sf Init}}
\newcommand\out{{\sf Out}}
\newcommand\rt{{\sf Root}}
\newcommand\order{{\sf Order}}

\newcommand{\CTK}{\mathsf{CTK}}
\newcommand{\PUSH}{\mathsf{PUSH}}
\newcommand{\STK}{\mathsf{STK}}
\newcommand{\SIT}{\mathsf{SIT}}
\newcommand{\STD}{\mathsf{STD}}
\newcommand{\CTP}{\mathsf{CTP}}
\newcommand{\RTF}{\mathsf{RTF}}
\newcommand{\POP}{\mathsf{POP}}
\newcommand{\NUP}{\mathsf{NUP}}
\newcommand{\STP}{\mathsf{STP}}
\newcommand{\LTK}{\mathsf{LTK}}

\newcommand{\id}{\mathsf{id}}

\newcommand{\addr}{\symbol{64}}

\newcommand{\opset}{\mathcal{O}}

\newcommand{\opn}{op}
\newcommand{\op}{o}


\newcommand\Aut{{\mathfrak{A} }}
\newcommand\AutS{{\mathfrak{S} }}
\newcommand\AutB{{\mathfrak{B} }}
\newcommand\Tran{{\mathfrak{T} }}
\newcommand\Lang{{\mathscr{L} }}
\newcommand\TLang{{\mathscr{T} }}
\newcommand\TranSet{{\mathscr{T} }}
\newcommand\UTrans{{\mathscr{T}_{RLP} }}
\newcommand\Tranbasis{{\mathscr{B} }}
\newcommand\ConfSet{{\mathscr{C} }}


\newcommand{\PDS}{\textsf{PDS}}
\newcommand{\TrPDS}{\textsf{TrPDS}}
\newcommand{\WSTrPDS}{\textsf{WSTrPDS}}
\newcommand{\WOTrPDS}{\textsf{WPOTrPDS}}
\newcommand{\WOMTrPDS}{\textsf{WPOMTrPDS}}
\newcommand{\WSTrNFA}{\textsf{WSTrNFA}}
\newcommand{\WOTrNFA}{\textsf{WPOTrNFA}}
\newcommand{\TrNFA}{\textsf{TrNFA}}
\newcommand{\NFA}{\textsf{NFA}}
\newcommand{\MA}{\textsf{MA}}
\newcommand{\NFT}{\textsf{NFT}}
\newcommand{\LTLNFT}{\textsf{L2LNFT}}

\newcommand\Nat{\mathbb{N} }

\newcommand{\dwsts}{\mathscr{S}}
\newcommand{\wstsnodes}{\mathscr{S}}

\newcommand\toptsk{\mathsf{TopTsk}}
\newcommand\topact{\mathsf{Top}}
\newcommand\btmact{\mathsf{Btm}}
\newcommand\push{\mathsf{Push}}
\newcommand\pop{\mathsf{pop}}
\newcommand\Pop{\mathsf{Pop}}
\newcommand\mvacttop{\mathsf{MvAct2Top}}
\newcommand\clrtop{\mathsf{ClrTop}}
\newcommand\clrtsk{\mathsf{ClrTsk}}
\newcommand\mvtsktop{\mathsf{MvTsk2Top}}
\newcommand\rmtsk{\mathsf{RmTsk}}
\newcommand\rmact{\mathsf{RmAct}}
\newcommand\newtsk{\mathsf{NewTsk}}
\newcommand\gettsk{\mathsf{GetTsk}}
\newcommand\getsit{\mathsf{GetSIT}}
\newcommand\getrealtsk{\mathsf{GetRealTsk}}

\newcommand\rmv{\mathsf{Rmv}}
\newcommand\weight{\mathsf{Weight}}



\newcommand\shortlong[2]{#2}
%\newcommand\shortlong[2]{#1}

\newif\ifdraft\drafttrue
%\newif\ifdraft\draftfalse
\ifdraft
\newcommand{\jinlong}[1]{\color{red} {JL: #1 :LJ} \color{black}}
\newcommand{\zhilin}[1]{\color{blue} {ZL: #1 :LZ} \color{black}}
\newcommand{\tl}[1]{\color{magenta} {TL: #1 :LT} \color{black}}
\else
\newcommand{\jinlong}[1]{}
\newcommand{\zhilin}[1]{}
\newcommand{\tl}[1]{}
\fi

\journal{Information and Computation}

\begin{document}

\begin{frontmatter}

%% Title, authors and addresses

%% use the tnoteref command within \title for footnotes;
%% use the tnotetext command for theassociated footnote;
%% use the fnref command within \author or \address for footnotes;
%% use the fntext command for theassociated footnote;
%% use the corref command within \author for corresponding author footnotes;
%% use the cortext command for theassociated footnote;
%% use the ead command for the email address,
%% and the form \ead[url] for the home page:
%% \title{Title\tnoteref{label1}}
%% \tnotetext[label1]{}
%% \author{Name\corref{cor1}\fnref{label2}}
%% \ead{email address}
%% \ead[url]{home page}
%% \fntext[label2]{}
%% \cortext[cor1]{}
%% \affiliation{organization={},
%%             addressline={},
%%             city={},
%%             postcode={},
%%             state={},
%%             country={}}
%% \fntext[label3]{}

\title{Decision Procedures for Configuration Reachability of Android Stack Machine}

%% use optional labels to link authors explicitly to addresses:
%% \author[label1,label2]{}
%% \affiliation[label1]{organization={},
%%             addressline={},
%%             city={},
%%             postcode={},
%%             state={},
%%             country={}}
%%
%% \affiliation[label2]{organization={},
%%             addressline={},
%%             city={},
%%             postcode={},
%%             state={},
%%             country={}}

\author[a1,a2]{Jinlong He}
\ead{hejl@ios.ac.cn}
\author[a1,a2]{Zhilin Wu}
\ead{wuzl@ios.ac.cn}
\author[a3]{Taolue Chen}
\ead{t.chen@bbk.ac.uk}

%\hide{
\affiliation[a1]{organization={Key Laboratory of System Software \& State Key Laboratory of Computer Science, \\
Institute of Software, Chinese Academy of Sciences},
            %addressline={}, 
            city={Beijing},
            %postcode={}, 
            %state={},
            country={China}}

\affiliation[a2]{organization={University of Chinese Academy of Sciences},
            %addressline={}, 
            city={Beijing},
%%          citysep={}, % Uncomment if no comma needed between city and postcode
            %postcode={}, 
            %state={Beijing},
            country={China}}
            
\affiliation[a3]{organization={Birkbeck, University of London},
	%addressline={}, 
	city={London},
	%%          citysep={}, % Uncomment if no comma needed between city and postcode
	%postcode={}, 
	%state={Beijing},
	country={UK}}
%}

% Footnote of the second author
%\fnmark[2]

% Email id of the second author
%\ead{}

% URL of the second author
%\ead[url]{}

% Credit authorship
%\credit{}

% Address/affiliation

 
\begin{abstract}
In this paper, we consider the reachability problem of Android Stack Machine (ASM), a formal model to capture key mechanisms of Android multi-tasking such as activities, back stacks, launch modes and task affinities. The model is based on pushdown systems with multiple stacks, and focuses on the evolution of the back stack of the Android system when interacting with activities carrying specific launch modes and task affinities. We investigate the configuration reachability problem of ASM. While the decidability of the reachability problem of ASM is open, we identify an expressive sub-model for which various techniques for pushdown systems or their extensions are harnessed to show the decidability of the reachability problem.
\end{abstract}

% Use if graphical abstract is present
%\begin{graphicalabstract}
%\includegraphics{}
%\end{graphicalabstract}

% Research highlights

% Keywords
% Each keyword is seperated by \sep
\begin{keyword}
Android, multitasking mechanism, stack, configuration reachability, pushdown systems, saturation, transduction, well partial order, downward well-structured transition systems
\end{keyword}
\end{frontmatter}

%\maketitle

%%%%%%%%%%%%%%%%%%%%%%%%%%%%%%%%%%%%%%%%%%%%%%%%%%%%%%%%%%%%%%%%%%%%%%%%%%%%%%%%%%%%%%%%%%%%%

\section{Introduction} \label{sec:intro}
%!TEX root = main.tex

%Motivation

Multi-tasking plays a central role in the Android platform. %, and certainly is a complicated issue.
%It is considerably different from, say, PC multi-tasking. \tl{why?}
%As a fundamental concept, according to Android documentation \cite{}, ``a task is a collection of activities that users interact with when performing a certain job". Here an activity usually refers to UI components,  %In other words,
%and a task contains activities that may belong to multiple apps.
%
%each app can run in one or multiple processes.\FU{I am not sure whether an app can run in multiple processes, usually an app is run in a sandbox.}
%The unique design of Android multitasking
Its unique design, via activities and back stacks, greatly facilitates organizing user sessions through tasks, and provides rich features such as handy application switching, background app state maintenance, smooth task history navigation (using the ``back" button), etc \cite{RZXWL15}. We refer the readers to Section~\ref{sec:amm} for an overview. 

Android multitasking mechanism has substantially enhanced user experiences of the Android system and promoted personalized features in app design. However, the mechanism is also notoriously difficult to understand. As a witness, it constantly baffles app developers and has become a common topic of question-and-answer websites (for instance, \cite{stackoverflow}).
%
%summarize related problems in stackoverflow as a table, e.g., question 3219726 . This may provides some ideas on what kind of properties can be checked by ASS.
%
%Surprisingly, the Android multitasking mechanism, despite its importance, has not been thoroughly studied before, let along a formal treatment. This has impeded further developments of computer-aided (static) analysis and verification for Android apps, which are indispensable for vulnerability analysis (for example, detection of task hijacking \cite{RZXWL15}) and app performance enhancement (for example, estimation of energy consumption \cite{HL+13}).
For the formalization of Android multitasking mechanism, hitherto the most complete %formal semantics 
model is the Android Stack Machine ({\AMASS}~\cite{HC+19}). % to capture the key features of Android multitasking mechanism. 
%
{\AMASS} addresses the behavior of Android \emph{back stacks}, a key component of the multitasking machinery, and their interplay with attributes of the activity and the intent flags. 
%In this paper, for these attributes we consider four basic \emph{launch modes}, i.e., standard ({\bf $\standard$}), singleTop ({\bf $\singletop$}), singleTask ({\bf $\singletask$}), singleInstance ({\bf $\singleinstance$}), and \emph{task affinities}. (For simplicity more complicated activity attributes such as \emph{allowTaskReparenting} will not be addressed in the present paper.)
%We remark that, however, the intent flags of activities are abstracted away, to keep the model as neat as possible.
%We believe that the semantics of ASM, specified as a transition system, captures faithfully the actual mechanism of  Android systems. For each case  of the semantics, we have created ``diagnosis" apps with corresponding launch modes and task affinities, and carried out extensive experiments using these apps, ascertaining its conformance to the %mechanism supported by the
%Android platform. 
(Details will be reviewed in Section~\ref{sec:amm}.)

%
%The model addresses certain key features of Android multi-tasking such as launchMode and taskAffinity, while skip the other attributes.
%From an engineering perspective,
%For Android, technically ASM can be viewed as the counterpart of pushdown systems with multiple stacks, which are the \emph{de facto} model for (multi-threaded) concurrent programs.
%ASM gives--to the best of our knowledge--a first formal semantics for Android's multi-tasking mechanism.
%Being rigours, this model opens a door towards a formal account of Android's multi-tasking mechanism, which  would greatly facilitate developers' understanding, freeing them from lengthy, ambiguous, elusive Android documentations. We remark that it is known that the evolution of Android back stacks could also be affected by the \emph{intent flags} of the activities. ASM does not address intent flags explicitly. However, %it can be easily adapted to simulate
%the effects of most intent flags (e.g., {\small $\sf FLAG\_ACTIVITY\_NEW\_TASK$, $\sf FLAG\_ACTIVITY\_CLEAR\_TOP$}) can be simulated by %since they are similar to those of
%launch modes, so this is \emph{not} a real limitation of ASM. %However, in this paper, we  %focuses on two attributes of activities, namely the launch mode and the task affinity;
%do not address more complicated activity attributes such as allowTaskReparenting,
%which are left as the future work.

Based on {\AMASS}, we %make the first step towards a 
investigate formal analysis and verification of the behaviors of Android apps  with respect to the multitasking mechanism, %by investigating 
in particular, the \emph{configuration reachability problem} which is fundamental to all such analysis. 
%In this paper, we focus on the configuration reachability problem of $\AMASS$. This problem is fundamental to the formal (static) analysis and verification of the behaviors of Android apps
%
{\AMASS} is akin to pushdown systems with multiple stacks, so it is perhaps not surprising that the problem is hard in general; 
%in fact, we show undecidability for most interesting fragments even with just two launch modes. %(See Theorem~\ref{thm:undec}---\ref{prop:tridec} for details.)
in the interest of seeking more expressive, practice-relevant decidable fragments,
%observe that $\standard/\singletop$ activities must be supported, and %$\singletask/\singleinstance$
%$\singleinstance$ activities are desirable.\zhilin{seems strange ?}  %commonly used in Android apps. %\footnote{statistics data?}.
%Although the combination of all of them is unfortunately undecidable,
%\tl{I am not satisfied here, any comments on how to improve?}.
%On top of that, we hypothesize that restricting $\singletask$ and $\singleinstance$ activities \emph{individually} is a promising way. To this end,
we identify a fragment referred to as \textbf{$\STK$-dominating {\AMASS}} 
which 
%assumes $\STK$ activities have different task affinities and which further restricts the use of $\STK$ activities and the intent flag $\ntkflag$. This fragment 
covers a majority of open-source Android apps (e.g., from F-Droid) we have found so far. (The precise definition of $\STK$-dominating {\AMASS} is technical and will be given in Section~\ref{sec-conf-reach}.) 

%\tl{claiming this fragment can cover most of open-resource apps we have found?shall we}\zhilin{how about the previous sentence ?}
The primary contribution of the current paper is to give decision procedures for the reachability problem of $\STK$-dominating {\AMASS}. We extend the saturation procedure of pushdowns systems (\PDS) to multiple stacks.  Moreover, we propose \emph{pushdown systems with well-partially-ordered transductions} (\WOTrPDS), an extension of finite pushdown system with transductions (cf.\ Definition~\ref{def:trpds}), where the closure of transductions may be infinite but admits a basis which is well-partially-ordered with respect to the superset relation, we show that the configuration reachability of {\WOTrPDS} is decidable, and utilize {\WOTrPDS} to show that the reachability problem of $\STK$-dominating {\AMASS} is decidable.

This paper is technical by nature. However, the model {\AMASS} and the decision procedures for the reachability problem lay a solid foundation of formal verification of Android apps when the multitasking mechanism is taken into account. This is a new application of pushdown model checking in program analysis by large. On top, we believe some of the techniques, for instance, pushdown systems with well-partially-ordered transductions are interesting in their own right and may be used to solve other  problems in pushdown model checking.     
%As a complement, we also studied a fragment \textbf{$\singleinstance$-acyclic-mediating ASS}, which include $\singleinstance$, but are free of $\singletask$, activities, subject to additional restrictions.
%The work, apart from independent interests in the study of multi-stack pushdown systems, lays a solid foundation for further (static) analysis and verification of Android apps related to multi-tasking, enabling model checking of Android apps, security analysis (such as discovering task hijacking), or typical tasks in software engineering such as  % Assist programmers with
%automatic debugging, model-based testing, etc.
%task-sensitive analysis of Android apps, generate testcases from the model and test consistent of variants of Android OS.

%We summarize the main contributions as follows: (1) We extend the saturation procedure of PDS to multiple stacks. (2) We propose pushdown systems with well-partially-ordered transductions (WPOTrPDS), and show that the reachability problem for WPOTrPDS is decidable.
%%---to the best of our knowledge---the first comprehensive formal model, Android stack machine, for Android back stacks, which %captures both launch modes and task affinities of activities. The model
%%is also validated by extensive experiments. 
%%	
%	%, for the Android multi-task mechanism. To validate the conformance of the model with respect to the Android platform, we have created diagnosis apps and designed extensive experiments. To the best of our knowledge, Android stack systems is the first model for Android back stack systems that captures both the launch modes and task affinities of activities.
%%	
%%	The first model of this kind.
%%
%    (3) We integrate the saturation procedure for multiple stacks and WPOTrPDS to design a decision procedure for the reachability problem for Android stack machine ({\AMASS}). 
%    %Apart from strongest possible undecidablity results in the general case, we %. Show  that the reachability problem is undecidable in general, and

\smallskip	
\noindent\emph{Organization.} The rest of this paper is structured as follows. In Section~\ref{sec:prel}, 
%In Section~\ref{sec:amm}, 
we first give an overview of the Android multitasking mechanism, and then give %the notations are fixed and some 
basic concepts and results regarding the formal models to used in the paper.  %such as pushdown systems are given. 
The {\AMASS} model and the configuration reachability problem will be defined in Section~\ref{sec:amm}.  %we define the configuration reachability problem, $\STK$-dominating {\AMASS}, and state the main result of this paper. 
%Since the decision procedure is involved, to ease the reading, we choose to present the decision procedure step by step. 
In the next two sections, we shall present the semantics rules and decision procedures of the configuration reachability problem
for two fragment of %$\STK$-dominating 
{\AMASS}. In particular, in Section~\ref{sec:reach-lmamass}, we extend the saturation procedure of {\PDS} to multiple stacks to solve the reachability problem of $\STK$-dominating {\AMASS} where the intent flags are ignored. 
In Section~\ref{sec:reach-ifamass}, we introduce {\WOTrPDS}, show the decidability of its reachability problem, and utilize it to solve the reachability problem of $\STK$-dominating {\AMASS} where the launch modes of all activities are ``standard'' or ``singleTop''. Finally, we show how the ideas in Section~\ref{sec:reach-lmamass} and Section~\ref{sec:reach-ifamass} can be combined to solve the reachability problem of $\STK$-dominating {\AMASS}. 

This paper unifies and extends some of the results from the previous publications \cite{CHS+18,HC+19}.  The main differences are %and differs from the decidability result in~\cite{CHS+18} in the following aspects: 
as follows. (1) In this paper, we study the {\AMASS} model proposed in \cite{HC+19}, while the (earlier) model~\cite{CHS+18} considered only the launch modes but no intent flags. (2) The decision procedure for the fragment $\STK$-dominating $\LMAMASS$~\cite{CHS+18} relies on pushdown systems with transductions; this paper presents a new (simplified) decision procedure, which is obtained by a natural extension of the saturation procedure for pushdown systems. % (see Section~\ref{sec:reach-lmamass}). 3) Furthermore, in this paper, 
(3) We propose a novel model of pushdown systems with well-partially-ordered transductions (\WOTrPDS) %is proposed 
to solve the reachability problem of $\STK$-dominating {\AMASS}. This result is completely new.  


%======================================================================

 
\section{Android Multitasking mechanism} \label{sec:amm}
%!TEX root = main.tex

%\zhilin{The introduction to Android system below is copied from https://developer.android.com/}
In Android, an application, usually referred to as an \emph{app}, is regarded as a collection of \emph{activities}. An activity is a type of app components, an instance of which provides a graphical user interface on screen and  serves the entry point for interacting with the user~\cite{Androiddoc}. An app typically has many activities for different user interactions (e.g., dialling phone numbers, reading contact lists, etc). A distinguished activity is the \emph{main} activity, which is started when the app is launched.
%\tl{I donot know where, but better mention the main activity}
%\zhilin{add main activity here, please check.}
%\tl{I have the slight concern that in the model, there is only one main activity, so the main activity should not defined on the level of apps? it belongs to the system??...}
%\FU{I think it is defined in the manifest file.}
%As ,  an  activity  represents a single screen with a user interface.
%
%For example, an email app might have one activity that shows a list of new messages, another activity to compose an email, and another activity for reading emails. Although the activities work together to form a cohesive user experience in the email app, each one is independent of the others. As such, a different app can start any one of these activities if the email app allows it. For example, a camera app can start the activity in the email app that composes new mail to allow the user to share a picture.
 
 
A \emph{task} is a collection of activities that users interact with when performing a certain job.
The activities in a task are arranged in a stack %---the \emph{back stack}---\tl{this is not quite precise?!}
in the order in which each activity is opened. For example, an email app might have one activity to show a list of latest messages. When the user selects a message, a new activity opens to view that message. This new activity is pushed to the stack. If the user presses the ``Back'' button, an activity is finished and is popped off the stack. [In practice, the onBackPressed() method can be overloaded and triggered when the ``Back'' button is clicked. Here we assume---as a model abstraction---that the onBackPressed() method is not overloaded.]
Furthermore, multiple tasks may run concurrently in the Android platform and the \emph{back stack} stores all the tasks as a stack as well. In other words, it has a nested structure being a stack of stacks (tasks). The first task that is started when an app is launched is called the \emph{main task}. Note that the main task contains the main activity as the bottom activity. 
%
%In other words, there may exist multiple tasks in the back stack, and they are organized as a stack as well.
We remark that %the back stack evolves in a way that the role of apps is diminished, in the  sense that
in android, activities from different apps can stay in the same task, and activities from the same app can enter different tasks.

Typically, the evolution of the back stack is dependent mainly on three basic attributes: \emph{launch modes},\emph{task affinities} and \emph{intent flags}. All the activities of an app, as well as their attributes, including the launch modes and task affinities, are defined in the \emph{manifest file} of the app. Differently, intent flags are set by caller activities to declare how to activate target activities by calling startActivity() or startActivityForResult() with the intent flags as its arguments. The launch mode of an activity decides the corresponding operation of the back stack when the activity is launched. There are four basic launch modes in Android: ``standard'', ``singleTop'', ``singleTask'' and ``singleInstance''. 

The task affinity of an activity indicates to which task the activity prefers to belong. By default, all the activities from the same app have the same affinity (i.e., all activities in the same app prefer to be in the same task). However, one can modify the default affinity of the activity. Activities defined in different apps can share a task affinity, or activities defined in the same app can be assigned with different task affinities. 
%Below we will use a simple app to demonstrate the evolution of the back stack.




%===================================================================================


\section{Preliminaries}~\label{sec:prel}
%!TEX root = main.tex

\subsection{Android Multitasking Mechanism} \label{sec:amm}
%!TEX root = main.tex

%\zhilin{The introduction to Android system below is copied from https://developer.android.com/}
In Android, an application, usually referred to as an \emph{app}, is regarded as a collection of \emph{activities}. An activity is a type of app components, an instance of which provides a graphical user interface on screen and  serves the entry point for interacting with the user~\cite{Androiddoc}. An app typically has many activities for different user interactions (e.g., dialling phone numbers, reading contact lists, etc). A distinguished activity is the \emph{main} activity, which is started when the app is launched.
%\tl{I donot know where, but better mention the main activity}
%\zhilin{add main activity here, please check.}
%\tl{I have the slight concern that in the model, there is only one main activity, so the main activity should not defined on the level of apps? it belongs to the system??...}
%\FU{I think it is defined in the manifest file.}
%As ,  an  activity  represents a single screen with a user interface.
%
%For example, an email app might have one activity that shows a list of new messages, another activity to compose an email, and another activity for reading emails. Although the activities work together to form a cohesive user experience in the email app, each one is independent of the others. As such, a different app can start any one of these activities if the email app allows it. For example, a camera app can start the activity in the email app that composes new mail to allow the user to share a picture.
 
 
A \emph{task} is a collection of activities that users interact with when performing a certain job.
The activities in a task are arranged in a stack %---the \emph{back stack}---\tl{this is not quite precise?!}
in the order in which each activity is opened. For example, an email app might have one activity to show a list of latest messages. When the user selects a message, a new activity opens to view that message. This new activity is pushed to the stack. If the user presses the ``Back'' button, an activity is finished and is popped off the stack. [In practice, the onBackPressed() method can be overloaded and triggered when the ``Back'' button is clicked. Here we assume---as a model abstraction---that the onBackPressed() method is not overloaded.]
Furthermore, multiple tasks may run concurrently in the Android platform and the \emph{back stack} stores all the tasks as a stack as well. In other words, it has a nested structure being a stack of stacks (tasks). The first task that is started when an app is launched is called the \emph{main task}. Note that the main task contains the main activity as the bottom activity. 
%
%In other words, there may exist multiple tasks in the back stack, and they are organized as a stack as well.
We remark that %the back stack evolves in a way that the role of apps is diminished, in the  sense that
in android, activities from different apps can stay in the same task, and activities from the same app can enter different tasks.

Typically, the evolution of the back stack is dependent mainly on three basic attributes: \emph{launch modes},\emph{task affinities} and \emph{intent flags}. All the activities of an app, as well as their attributes, including the launch modes and task affinities, are defined in the \emph{manifest file} of the app. Differently, intent flags are set by caller activities to declare how to activate target activities by calling startActivity() or startActivityForResult() with the intent flags as its arguments. The launch mode of an activity decides the corresponding operation of the back stack when the activity is launched. There are four basic launch modes in Android: ``standard'', ``singleTop'', ``singleTask'' and ``singleInstance''. 

The task affinity of an activity indicates to which task the activity prefers to belong. By default, all the activities from the same app have the same affinity (i.e., all activities in the same app prefer to be in the same task). However, one can modify the default affinity of the activity. Activities defined in different apps can share a task affinity, or activities defined in the same app can be assigned with different task affinities. 
%Below we will use a simple app to demonstrate the evolution of the back stack.



\subsection{Basic Automata Models}

Let $\natnum$ be the set of natural numbers. For $k \in \natnum$, let $[k]=\{1,\cdots, k\}$. 
%
An alphabet $\Gamma$ is a finite set of letters.  A string over $\Gamma$ is a finite sequence of letters from $\Gamma$. In particular, $\varepsilon$ denotes %the empty sequence is called 
the empty string. We use $\Gamma^*$ (resp.\ $\Gamma^+$) to denote the set of (resp.\ nonempty) strings over $\Gamma$. A language $L$ over $\Gamma$ is a  subset of $\Gamma^*$. We write $\Gamma_\varepsilon: = \Gamma \cup \{\varepsilon\}$. Let $w$ be a string over $\Gamma$ and $\Gamma' \subseteq \Gamma$. We use $w |_{\Gamma'}$ to denote the string obtained from $w$ by removing the symbols not in $\Gamma'$. For instance, let $w = a  c b c$ and $\Gamma' = \{a,b\}$, then $w|_{\Gamma'} = ab$. Moreover, for a language $L \subseteq \Gamma^*$ and $\Gamma' \subseteq \Gamma$, we use $L|_{\Gamma'}$ to denote the set of strings $w |_{\Gamma'}$ for $w \in L$. 
 
\subsubsection{Nondeterministic finite automata}
%Fix a finite \emph{alphabet} $\Gamma$. Elements in $\Gamma^*$ are called \emph{strings}. Let $\varepsilon$ denote the empty string and  $\Gamma^+ = \Gamma^* \setminus \{\varepsilon\}$. A \emph{language} is a subset of $\Gamma^*$. Moreover, we also use $\Gamma_\varepsilon$ to denote $\Gamma \cup \{\varepsilon\}$.

%We will use $a,b,\cdots$ to denote letters from $\Gamma$ and $u, v, w, \cdots$ to denote strings from $\Gamma^*$. For a string $u \in \Gamma^*$, let $|u|$ denote the \emph{length} of $u$ (in particular, $|\varepsilon|=0$). A \emph{position} of a nonempty string $u$ of length $n$ is a number $i \in [n]$ (Note that the first position is $1$, instead of  0). In addition, for $i \in [|u|]$, let $u[i]$ denote the $i$-th letter of $u$. For two strings $u$ and $v$, if there is a string $w$ such that $u = w v$, then $v$ is called a \emph{suffix} of $u$.

%For two strings $u_1, u_2$, we use $u_1 \cdot u_2$ to denote the \emph{concatenation} of $u_1$ and $u_2$, that is, the string $v$ such that $|v|= |u_1| + |u_2|$ and for each $i \in [|u_1|]$, $v[i]= u_1[i]$ and for each $i \in |u_2|$, $v[|u_1|+i]=u_2[i]$. Let $u, v$ be two strings. If $v = u \cdot v'$ for some string $v'$, then $u$ is said to be a \emph{prefix} of $v$. In addition, if $u \neq v$, then $u$ is said to be a \emph{strict} prefix of $v$. If $u$ is a prefix of $v$, that is, $v = u \cdot v'$ for some string $v'$, then  we use $u^{-1} v$ to denote $v'$. In particular, $\varepsilon^{-1} v = v$.

A nondeterministic finite automaton (\NFA) $\Aut$ is a tuple $(Q, \Gamma, \delta, I, F)$, where $Q$ is a finite set of states, $\Gamma$ is a finite alphabet, $I \subseteq Q$ is the set of initial states, $F \subseteq Q$ is the set of final states, and $\delta \subseteq Q \times \Gamma_\varepsilon \times Q$ is the transition relation. Note that {\NFA}s defined here allow $\varepsilon$-transitions.
%
For a string $w = a_1 \dots a_n$, a run of $\Aut$ on $w$ is a sequence $q_0 b_1q_1 \dots q_{m-1} b_m q_m$ such that $q_0 \in I$,  $(q_{i-1}, b_i, q_i) \in \delta$ for each $i \in [m]$ and $b_1 \cdots b_m = w$ (note that $b_i$ may be $\varepsilon$ for some $i$). In this case, we normally write $q \xRightarrow[\Aut]{w} q_m$ 
%to denote that there is a path from $q$ to $q'$ where the transitions are labeled by $b_1, \cdots, b_m$ with $b_1 \cdots b_m = w$. 
%respectively such that $b_1 \dots b_m = a_1 \dots a_n$ in $\Aa$. 
(where $\Aut$ is usually omitted %in $\xRightarrow[\Aa]{w}$ 
for brevity).
%
A run $q_0 b_1q_1 \dots q_{m-1} b_m q_m$ is accepting if $q_m \in F$. A string $w$ is accepted by $\Aut$ if there is an accepting run of $\Aut$ on $w$. 

Let $\Lang(\Aut)$ denote the language defined by $\Aut$, namely, the set of strings accepted by $\Aut$. 
A language over $\Gamma$ is regular if it can be defined by an \NFA.
 

%%%%%%%%%%%%%%%%%%%%%%%%%%%%%%%%%%%%%%%%%%%%%%%%%%%%%%%%%%%%%%%%%%%%%%%%%%%%%%%%%%%%%%%%%%%%%%%

\subsubsection{Pushdown systems}
A pushdown system ({\PDS}) $\Pp$ is a tuple $(P,\Gamma,\Delta)$, where $P$ is a finite set of control states, $\Gamma$ is a finite stack alphabet, and $\Delta\subseteq P\times\Gamma\times\Gamma^{\le 2}\times P$ is a finite set of transition rules, where $\Gamma^{\le 2} = \{\epsilon\}\cup\Gamma\cup\Gamma\times\Gamma$.
Let $\Pp=(P,\Gamma,\Delta)$ be a {\PDS}. A configuration of $\Pp$ is a pair $(p,w)\in P\times\Gamma^*$, where $w$ denotes the content of the stack (with the leftmost symbol being the top of the stack). Let $\confs(\Pp)$ denote the set of configurations of $\Pp$. we define a binary relation $\xrightarrow{\Pp}$ over $\confs(\Pp)$ as follows: $(p,w)\xrightarrow{\Pp}(p',w')$ iff $w = \gamma w_1$ and there exists $w''\in\Gamma^*$ such that $(p,\gamma,w'',p')\in\Delta$ and $w'=w''w_1$. We use $\xRightarrow{\Pp}$ to denote the reflexive and transitive closure of $\xrightarrow{\Pp}$.

A configuration $(p',w')$ is reachable from $(p,w)$ if $(p,w)\xRightarrow{\Pp}(p',w')$. For $C \subseteq \confs(\Pp)$, let $\pre^*_\Pp(C)$ (resp.\ $\post^*_\Pp(C)$) denote the set of predecessors (resp.\ successors) of elements of $C$, that is, $\{(p',w')\mid \exists (p,w)\in C,(p',w')\xRightarrow{\Pp}(p,w)\}$ (resp.\ $\{(p',w')\mid \exists (p,w)\in C,(p,w)\xRightarrow{\Pp}(p',w')\}$).

%As a standard machinery to solve the reachability of {\PDS}, a $\Pp$-\emph{nondeterministic-finite-automaton} ($\Pp$-{\NFA}) is an {\NFA} $\Aut = (P',\Gamma,\delta,I,F)$ such that $I\subseteq P\subseteq P'$. 
%Let $(p,w)\in\conf_{\Pp}$, $(p,w)$ is accepted by $\Aut$ if $p\in I$ and $p\xRightarrow[\Aut]{w}p_f$ and $p_f\in F$. We use $\Aut(p)$ to denote the $\Pp$-{\NFA} obtained from $\Aut$ by replacing $I$ with $\{p\}$.

As a standard machinery to solve reachability for PDS, a $\Pp$-nondeterministic finite automaton ($\Pp$-NFA) is an NFA $\Aut=(P', \Gamma, \delta, I, F)$ such that $I \subseteq P \subseteq P'$ \cite{BEM97}.
Evidently, $\Pp$-NFA are a special class of NFA. 
Let $\Aut = (P', \Gamma, \delta, I, F)$ be a  $\Pp$-NFA and  $(p, w) \in \confs(\Pp)$. Then $(p, w)$ is \emph{accepted} by $\Aut$ if $p \in I$ and there is an accepting run $p_0 p_1 \cdots p_n$ of $\Aut$ on $w$ with $p_0 = p$. Let $\confs(\Aut)$ denote the set of configurations accepted by $\Aut$. Moreover, let $\Lang(\Aut)$ denote the set of words $w$ such that $(p, w) \in \confs(\Aut)$ for some $p \in I$. 
%For brevity, we usually write NFA instead of $\Pp$-NFA when $\Pp$ is clear from the context. 
For an $\Pp$-NFA $\Aut = (P', \Gamma, \delta, I, F)$ and $p' \in P$, we use $\Aut(p')$ to denote the $\Pp$-NFA obtained from $\Aut$ by replacing $I$ with $\{p'\}$. A set of configurations $C \subseteq \confs(\Pp)$ is \emph{regular} if there is a $\Pp$-NFA $\Aut$ recognizing $C$, that is, $\confs(\Aut) = C$. %\zhilin{added some notation for MA}\FU{I added the definition of regular set.}

\begin{theorem}[\cite{BEM97}]\label{thm-pds}
	Given  a PDS $\Pp$ and a set of configurations accepted by a $\Pp$-NFA $\Aut$, we can compute, in polynomial time in $|\Pp|+|\Aut|$, two $\Pp$-NFAs $\Aut^{\pre^*}_{\Pp}$  and $\Aut^{\post^*}_{\Pp}$ that recognize $\pre^*_\Pp(\confs(\Aut))$ and $\post^*_\Pp(\confs(\Aut))$ respectively.
\end{theorem}

The $\Pp$-NFAs $\Aut^{\pre^*}_{\Pp}$ and $\Aut^{\post^*}_{\Pp}$ in Theorem~\ref{thm-pds} are constructed by applying the saturation procedures. Let us recall the saturation procedure for $\Aut^{\post^*}_{\Pp}$ in the sequel, since later on, we focus on the computation of $\post^*$.

Let $\Aut'$ be the current $\Pp$-NFA. Then there are the following three saturation rules for computing $\post^*$. 

\smallskip
\fbox
{
\begin{minipage}{0.9\textwidth}
\begin{enumerate}
    %
    \item If $(p, \gamma, \varepsilon, p') \in \Delta$ and $p \xRightarrow[\Aut']{\gamma} q$, then add a transition $(p', \varepsilon, q)$ to $\Aut'$. 
%
    \item If $(p, \gamma, \gamma', p') \in \Delta$ and $p \xRightarrow[\Aut']{\gamma} q$, then add a transition $(p', \gamma', q)$ to $\Aut'$. 
%
    \item If $(p, \gamma, \gamma' \gamma'', p') \in \Delta$ and $p \xRightarrow[\Aut']{\gamma } q$, then add two transitions $(p', \gamma', \langle p', \gamma'\rangle)$ and $(\langle p', \gamma' \rangle, \gamma'', q)$ to $\Aut'$.     
\end{enumerate}
\end{minipage}
}

\smallskip

Therefore, for a PDS $\Pp=(P,\Gamma,\Delta)$ and a $\Pp$-NFA $\Aut = (P', \Gamma, \delta, I, F)$, the set of states in $\Aut^{\post^*}_\Pp$ is $P' \cup \{\langle p, \gamma \rangle \mid p \in P, \gamma \in \Gamma \}$.


\subsubsection{Transductions}
A transduction $\tau$ over an alphabet $\Gamma$ is a binary relation over $\Gamma^*$, i.e., a subset of $\Gamma^* \times \Gamma^*$. (Note that $\emptyset$ is considered as a transduction.)
%In particular, $\emptyset \subseteq \Gamma^* \times \Gamma^*$ is the transduction that transduces no strings.
We use $\tau_{\id}$ to denote the identity transduction, namely,  $\{(w,w) \mid w \in \Gamma^*\}$. 
Moreover, we use $\tau_\varepsilon$ to denote the transduction $\{\varepsilon, \varepsilon\}$, namely the transduction comprising a single element $(\varepsilon, \varepsilon)$.
For a transduction $\tau$ and a string $w$, we write $\tau(w):= \{w' \mid (w, w') \in \tau \}$.

%\tl{why not just define l2l transducers as we only use them?}
A nondeterministic finite transducer  (\NFT) $\Tran$ is a tuple $(Q, \Gamma, \delta, I, F)$, 
where $Q, \Gamma, I, F$ are the same as \NFA, %is a finite set of states,
%$q_0 \in Q$ is the initial state, $F \subseteq Q$ is a finite set of final states, 
%$\Gamma$ and $\Gamma$ are the input and output alphabets respectively, 
and %$\delta \subseteq Q \times \Gamma^* \times \Gamma^* \times Q$ 
$\delta \subseteq Q \times \Gamma \times \Gamma \times Q$ is a finite set of transition rules.
%
A computation of $\Tran$ is a sequence of transitions
%$
%q_0 \xrightarrow[w'_1]{w_1} q_1  \xrightarrow[w'_2]{w_2} q_2 \cdots q_{n-1} \xrightarrow[w'_n]{w_n} q_n 
%$
$
q_0 \xrightarrow[a'_1]{a_1} q_1  %\xrightarrow[a'_2]{a_2} q_2 
\cdots q_{n-1} \xrightarrow[a'_n]{a_n} q_n 
$
such that $(q_{i-1}, a_i, a'_i, q_i) \in \delta$ for $i \in [n]$. For brevity, this is denoted by  $q_0 \xRightarrow[w']{w} q_n$ for $w= a_1 \cdots a_n$ and $w'= a'_1 \cdots a'_n$. % to denote the fact that there exists such a computation from $q_0$ to $q_n$.  
Moreover, the transduction defined by $\Tran$, denoted by $\TLang(\Tran)$, is defined as $\{(w, w') \mid \mbox{there exist } q \in I, q' \in F \mbox{ such that } q \xRightarrow[w']{w} q'\}$. Let $\UTrans$ denote the set of transductions defined by \NFT. (Note that in this paper, we only consider so called letter-to-letter transducers as defined above; the resulting transduction must be length-preserving and regular.)
%
%A \emph{letter-to-letter} transducer (\LTLNFT) is an {\NFT} $\Tran = (Q, \Gamma, \delta, I, F)$ where $\delta \subseteq Q \times \Gamma \times \Gamma \times Q$. 
%It transduces a string $w = a_1 \cdots a_n$ into a string $w' = b_1 \cdots b_n$
%if there exists a state sequence $q_0 \cdots q_n$ such that for each $i \in [n]$, $(q_{i-1}, a_i, b_i, q_{i}) \in \delta$, and $q_n \in F$.
%The language $L(T)$ of an L2LFST $T$ is the set of pair $(w, w')$ such that $T$ can transduce $w$ into $w'$.

%A transduction is \emph{regular} if it can be defined by an \NFT. A regular transduction is \emph{length-preserving} if it can be defined by an \LTLNFT. In this paper, we focus on $\UTrans$, the set of \emph{regular and length-preserving} transductions. %Let us use  to denote the set of regular and length-preserving transductions.

{\NFT}s define letter-to-letter transductions and can be seen as {\NFA}s over the alphabet $\Gamma \times \Gamma$. It follows that the inclusion problem of {\NFT}s is decidable.

\begin{proposition}\label{prop-nft-inclusion}
%The inclusion problem of {\NFT}s is decidable: 
Given two {\NFT}s $\Tran_1$ and $\Tran_2$,  whether $\TLang(\Tran_1) \subseteq \TLang(\Tran_2)$ is decidable.
\end{proposition} 

%
We define several operations on transductions relevant to this paper. 
\begin{itemize}
\item The \emph{union} of $\tau_1$ and $\tau_2$ is $\tau_1 \cup \tau_2:=\{(w, w') \mid (w, w') \in \tau_1 \mbox{ or } (w,w') \in \tau_2\}$.
%
%\item The \emph{converse} of $\tau$ is $\overline{\tau }:= \{(w, w') \mid (w', w) \in \tau\}$. 
%
\item %Given two transductions $\tau_1$ and $\tau_2$ over $\Gamma$, 
The \emph{composition} of $\tau_1$ and $\tau_2$ both of which are over $\Gamma$ is $\tau_1 \circ \tau_2 := \{(w, w'') \mid \exists w' \in \Gamma^*.\ (w,w') \in \tau_1$ and $(w',w'') \in \tau_2\}$. For $\tau$ and $n \in \Nat$ with $n \ge 1$, we use $\tau^n$ to denote the $n$-fold composition of $\tau$. 
%
\item For  $\tau$ and  $w_1,w_2 \in \Gamma^*$ with $|w_1| = |w_2|$, the \emph{left quotient} of $\tau$ with respect to $(w_1, w_2)$, denoted by $\lceil w_1, w_2 \rfloor^{-1} \tau$, is  $\{(w,w') \mid (w_1w, w_2w') \in \tau\}$. In particular, $\lceil \varepsilon, \varepsilon\rfloor^{-1} \tau =\tau$. Note that  $\lceil w_1, w_2 \rfloor^{-1} \tau = \emptyset$ if no  $w, w'$ exist with $(w_1w, w_2 w') \in \tau$.
%
\end{itemize}

\begin{proposition}
    For   $\gamma,\gamma'\in\Gamma$ and transductions $\tau_1,\tau_2$,
%    $$\lceil \gamma,\gamma'\rfloor^{-1}(\tau_1\circ\tau_2) = \bigcup_{\eta: \exists w, w'.(\gamma w,\eta w')\in\tau_1} (\lceil \gamma,\eta\rfloor^{-1}\tau_1)\circ(\lceil\eta,\gamma'\rfloor^{-1}\tau_2)  $$
    $$\lceil \gamma,\gamma'\rfloor^{-1}(\tau_1\circ\tau_2) = \bigcup_{\gamma'' \in \Gamma} (\lceil \gamma,\gamma''\rfloor^{-1}\tau_1)\circ(\lceil\gamma'',\gamma'\rfloor^{-1}\tau_2). $$
\end{proposition}

\begin{definition}[Transduction closure]
Let $\TranSet$ be a set of transductions. The \emph{closure of $\TranSet$ under composition, union, and left quotient}, denoted by $\langle\TranSet \rangle $, is defined as the least set of transductions satisfying the following three constraints,
\begin{enumerate}
    \item $\TranSet \subseteq \langle\TranSet \rangle $, $\emptyset \in \langle\TranSet \rangle $, $\tau_{id} \in \langle\TranSet \rangle $, and $\tau_{\varepsilon} \in \langle\TranSet \rangle $,
%
    \item if $\tau_1, \tau_2 \in \langle\TranSet \rangle $, then $\tau_1 \circ \tau_2 \in \langle\TranSet \rangle $ and $\tau_1 \cup \tau_2 \in \langle\TranSet \rangle $,
%
    \item if $\tau \in \langle\TranSet \rangle $, then $\lceil \gamma, \gamma' \rfloor^{-1} \tau \in \langle\TranSet \rangle $ for all $\gamma, \gamma' \in \Gamma$.
\end{enumerate}
\end{definition}

\begin{example}
Let $\Gamma = \{a, b, c\}$, $\TranSet = \{\tau_1, \tau_2\}$ where $\tau_1 = \{(aw, bw) \mid w \in \Gamma^*\}$ and $\tau_2 = \{(bw, cw) \mid w \in \Gamma^*\}$. Then 
%the closure $\langle \TranSet \rangle$ can be obtained from $\TranSet$ by iteratively adding the new transductions resulted from the satisfaction of the three constraints, until no new transductions can be produced. It is not hard to verify that 
$$\langle \TranSet \rangle = \{\tau_1, \tau_2, \emptyset, \tau_{id}, \tau_\varepsilon, \tau_3\} \cup  \{\tau_S \mid \emptyset \neq S \subseteq \{\tau_1, \tau_2, \tau_{id}, \tau_\varepsilon, \tau_3\} \},$$
where $\tau_3 := \tau_1 \circ \tau_2 = \{(aw, cw) \mid w \in \Gamma^*\}$ and $\tau_S = \bigcup \limits_{\tau' \in S} \tau'$. 
\end{example}

Finally, for a transduction $\tau$, $\overline{\tau}$ denotes the \emph{inversion} of $\tau$, that is, $\overline{\tau} = \{(w_2, w_1) \mid (w_1, w_2) \in \tau\}$. Moreover, for a set of transductions, say $\TranSet$, let $\overline{\TranSet}$  denote the set of $\overline{\tau}$ for $\tau \in \TranSet$.  

%\begin{itemize}
%\item from the first constraint, we deduce that $\{\tau_1, \tau_2, \emptyset, \tau_{id}\} \subseteq \tau_{id}$, 
%\item from the second constraint, we deduce that $\tau_3 \in \TranSet$, where $\tau_3 = \{(aw, cw) \mid w \in \Gamma^*\}$, and $\{\tau_1 \cup \tau_2, \tau_1 \cup \tau_{id}, \tau_2 \cup \tau_{id}\} \subseteq \TranSet$, 
%
%\item from the third constraint, we deduce that $(a, b)^{-1} \tau_1 = (b, c)^{-1} \tau_2 = \tau_{id} \in \TranSet$.
%\end{itemize}
%$\langle \TranSet \rangle$ can be computed by iteratively adding the new transductions resulted from the satisfaction of the three constraints, 
%\begin{itemize}
%\item $\TranSet_1 = \{\tau_1, \tau_2, \emptyset, \tau_{id}\}$,
%\item $\TranSet_2 = \{\tau_1, \tau_2, \emptyset, \tau_{id}, \tau_3, \tau_1 \cup \tau_2, \tau_1 \cup \tau_{id}, \tau_2 \cup \tau_{id}\}$, 
%\item $\TranSet_3 = \TranSet_2$.
%\end{itemize}
%$\TranSet_2 = \{\tau_1, \tau_2, \emptyset, \tau_{id}, \tau_3, \tau_1 \cup \tau_2, \tau_1 \cup \tau_{id}, \tau_2 \cup \tau_{id}, \tau_\}$
%comprises the union of the transductions from $\{\tau_1, \tau_2, \emptyset, \tau_{id}, \tau_3\}$.

%For a set of transductions $\TranSet$, we use $\overline{\TranSet}$ to denote $\{\overline{\tau} \mid \tau \in \TranSet\}$. It is easy to observe that $\overline{\langle\TranSet \rangle } = \langle\overline{\TranSet} \rangle$. 

%We say $(\langle \Tt \rangle, \subseteq)$ is \emph{dual well-ordered}  if it satisfies ACC and has no infinite antichains with respect to $\subseteq$. 
%In this paper, we reserve the following partial order relation $\preceq$ between transductions in $\UTrans$: For $\tau_1, \tau_2 \in \UTrans$, $\tau_1 \preceq \tau_2$ iff $\tau_2 \subseteq \tau_1$.
%(Note that $\preceq$ is actually $\supseteq$, but we introduce it for readability.) 
%
%In this paper, we consider the the subset relation $\subseteq$ between transductions. 
%Since %the converse operation preserves the $\preceq$ relation between transductions, that is, 
%$\tau_1 \preceq \tau_2$  iff $\tau_2 \subseteq \tau_1$ iff $\overline{\tau_2} \subseteq \overline{\tau_1}$ iff $\overline{\tau_1} \preceq \overline{\tau_2}$, we have %deduce the following result.
%\begin{proposition}
%For every set of transductions $\TranSet \subseteq \UTrans$, 
%    $(\langle \TranSet \rangle, \preceq)$ is well-ordered iff
%    $(\langle \overline{\TranSet} \rangle, \preceq)$ is well-ordered.

\subsubsection{Pushdown systems with transductions}

%Let us introduce an extension of pushdown systems, that is, pushdown systems with transductions \cite{UM13,Song18}. 

\begin{definition}[\TrPDS, \cite{UM13,Song18}] \label{def:trpds}
    A \emph{pushdown system with transductions} (\TrPDS) is a tuple $\Pp = (P, \Gamma, \TranSet, \Delta)$, 
    where $P$ is a finite set of $control\ states$, $\Gamma$ is a finite $stack\ alphabet$, $\TranSet$ is a finite set of letter-to-letter transductions, and $\Delta \subseteq P \times \Gamma \times \Gamma^{\le 2} \times \TranSet \times P$ is a finite set of transition rules, where $\Gamma^{\le 2} = \{\varepsilon\} \cup \Gamma \cup \Gamma \times \Gamma$. In other words, each transition in $\Delta$ is of one of the following forms, $(p, \gamma, \varepsilon, \tau, p')$, $(p, \gamma, \gamma',\tau, p')$, or $(p, \gamma, \gamma'_1 \gamma'_2, \tau, p')$ such that $\gamma, \gamma', \gamma'_1, \gamma'_2 \in \Gamma$ and $\tau \in \TranSet$. 
   Moreover, it is required that the transduction closure $\langle \TranSet \rangle$ is finite.  
\end{definition}
For readability, we write a transition $(p, \gamma, w, \tau, p') \in \Delta$ as $p \xrightarrow{\gamma/w | \tau} p' \in \Delta$.

%Let $\Pp = (P, \Gamma, \TranSet, \Delta)$ be  a {\WOTrPDS}. 
A \emph{configuration} $c$ of $\Pp$ is a pair $(p, w)$, where $w = \gamma_1 \cdots \gamma_k \in \Gamma^*$ is the  stack content with $\gamma_1$ as the topmost symbol. We define a binary relation $\xrightarrow{\Pp}$ over the set of configurations as follows. Given two configurations $(p_1, w_1)$ and $(p_2, w_2)$, $(p_1, w_1) \xrightarrow{\Pp} (p_2, w_2)$ if $w_1 = \gamma w'_1$, and one of the following conditions holds,
\begin{itemize}
\item $p_1 \xrightarrow{\gamma/\varepsilon|\tau} p_2$ for some $\tau$ with  $w_2 \in \tau(w'_1)$, 
%
\item $p_1 \xrightarrow{\gamma/\gamma'|\tau} p_2$ for some $\gamma'$ and $\tau$ with $w_2 = \gamma' w'_2$ for some $w'_2 \in \tau(w'_1)$, 
%
\item $p_1 \xrightarrow{\gamma/\gamma' \gamma''|\tau} p_2$ for some $\gamma', \gamma''$ and $\tau$ with $w_2 = \gamma' \gamma'' w'_2$ for some $w'_2 \in \tau(w'_1)$.
\end{itemize}
Moreover, we use $\xRightarrow{\Pp}$ to denote the reflexive and transitive closure of $\xrightarrow{\Pp}$. We say that $(p_2, c_2)$ is \emph{reachable} from $(p_1, c_1)$ if $(p_1, c_1) \xRightarrow{\Pp} (p_2, c_2)$.

%In this paper, we consider the \emph{regular reachability problem} of {\TrPDS}.
%\begin{quote} 
%	Given a {\TrPDS} $\Pp = (P, \Gamma, \TranSet, \Delta)$, control states $p_1, p_2 \in P$, regular languages $L_1, L_2$ over $\Gamma$, determine whether there exist %strings $w_1, w_2$ such that 
%	$w_1 \in L_1$, $w_2 \in L_2$ such that $(p_1, w_1) \xRightarrow{\Pp} (p_2, w_2)$.
%\end{quote}

\begin{definition}[Finite automata with transductions, {\TrNFA}] %[Finite automata with finite ascending chians and antichains transductions]
	Given a {\TrPDS} $\Pp=(P, \Gamma, \TranSet, \delta)$, a $\Pp$-nondeterministic-finite-automaton with transductions ($\Pp$-{\TrNFA}) is a tuple $\Aut =(P', \Gamma, \delta, I, 
	F)$,
	where $P'$ is a set of states such that $P \subseteq P'$, $I \subseteq P$ is a set of initial states, $F$ is a set of final states,  
	%and the set of initial states is $P$, %is a finite set of states, $\Sigma$ is a finite alphabet, 
	%$\TranSet$ is a finite set of transductions, %$I \subseteq Q$ (reps. $F \subseteq Q$) is a finite set of initial (reps. final) states, 
	and $\delta \subseteq P' \times \Gamma_\varepsilon \times \langle \TranSet\rangle \times P'$ is a finite set of transition rules.
\end{definition}
Evidently, a $\Pp$-{\NFA} can be seen as a  $\Pp$-{\TrNFA} where only the identity transduction $\tau_{id}$ is used. 
%
For convenience, we write a transition $(q, a, \beta, q') \in \delta$ as $q \xrightarrow{a | \beta} q'$. 
To define the semantics of $\Aut$, we extend the transition rules $q \xrightarrow{a | \beta} q' \in \delta$  to a relation $q \xRightarrow[\Aut]{w | \tau} q'$ for a string $w$,  
%
%The relation $q \xRightarrow{w | \tau} q'$ is  
defined inductively as follows:
\begin{itemize}
	\item if $q \xrightarrow{a |\beta} q'$, then $q \xRightarrow[\Aut]{a | \beta} q'$, 
	%
	\item if $q \xRightarrow[\Aut]{w | \tau} q'$ and $q' \xrightarrow{b |\beta } q'' $, then $q \xRightarrow[\Aut]{w a | (\lceil a, b\rfloor^{-1}\tau) \circ \beta} q''$ for every $a \in \Gamma_\varepsilon$ such that $\lceil a, b\rfloor^{-1}\tau \neq \emptyset$.
\end{itemize}
Intuitively, $q \xRightarrow[\Aut]{w | \tau} q'$ means that in $\Aut$,  $q'$ is reached starting from $q$ after reading $w$. Moreover, the transduction $\tau$, which is  to be applied to the remaining suffix of the input string, is produced. If $\Aut$ is clear from the context, then we may omit $\Aut$ and write $q \xRightarrow{w | \tau} q'$.

A configuration $(p, w)$ of $\Pp$ is accepted by $\Aut$ if $p \xRightarrow[\Aut]{w | \tau} q$ for some $q \in F$ and $\tau \in \langle \TranSet \rangle$ such that $(\varepsilon, \varepsilon) \in \tau$.
We use $\confs(\Aut)$ to denote the set of configurations of $\Pp$ accepted by $\Aut$.


\begin{theorem}[\cite{Song18}]\label{thm-trpds}
	Given  a {\TrPDS} $\Pp$ and a set of configurations accepted by a $\Pp$-{\TrNFA} $\Aut$, we can compute two $\Pp$-{\TrNFA}s $\Aut^{\pre^*}_{\Pp}$  and $\Aut^{\post^*}_{\Pp}$ that recognize $\pre^*_\Pp(\confs(\Aut))$ and $\post^*_\Pp(\confs(\Aut))$ respectively.
\end{theorem}

The $\Pp$-{\TrNFA}s $\Aut^{\pre^*}_{\Pp}$ and $\Aut^{\post^*}_{\Pp}$ in Theorem~\ref{thm-trpds} are constructed by applying the saturation procedures. We recall the saturation procedure for $\Aut^{\post^*}_{\Pp}$ since we shall focus on the computation of $\post^*$  later.

Let $\Aut'$ be the current $\Pp$-\TrNFA. Then there are the following three saturation rules for computing $\post^*$. 

\smallskip
\fbox
{
\begin{minipage}{0.9\textwidth}
\begin{enumerate}
    %
    \item If $p \xrightarrow{\gamma/\varepsilon | \tau} p' \in \Delta$ and $p \xRightarrow[\Aut']{\gamma | \tau' } q$, then add a transition $p' \xrightarrow{\varepsilon | \tau \circ \tau'} q$ to $\Aut'$. 
%
    \item If $p \xrightarrow{\gamma/\gamma' | \tau} p' \in \Delta$ and $p \xRightarrow[\Aut']{\gamma | \tau' } q$, then add a transition $p' \xrightarrow{\gamma' | \tau \circ \tau'} q$ to $\Aut'$. 
%
    \item If $p \xrightarrow{\gamma/\gamma'_1 \gamma'_2 | \tau} p' \in \Delta$ and $p \xRightarrow[\Aut']{\gamma | \tau' } q$, then add two transitions $p' \xrightarrow{\gamma'_1 | \tau_{id}} \langle p', \gamma'_1\rangle$ and $\langle p', \gamma'_1\rangle \xrightarrow{\gamma'_2 | \overline{\tau} \circ \tau' } q$ to $\Aut'$.  
\end{enumerate}
\end{minipage}
}

\smallskip

Therefore, for a {\TrPDS} $\Pp=(P,\Gamma,\Delta)$ and a $\Pp$-{\TrNFA} $\Aut = (P', \Gamma, \delta, I, F)$, the set of states in $\Aut^{\post^*}_\Pp$ is $P' \cup \{\langle p, \gamma \rangle \mid p \in P, \gamma \in \Gamma \}$.


%\end{proposition}

%%%%%%%%%%%%%%%%%%%%%%%%%%%%%%%%%%%%%%%%%%%%%%%%%%%%%%%%%%%%%%%%%%%%%%%%%%%%%%%%%%%%%%%%%%%%%%%%%%%%%%%%%%%
\subsubsection{Downward well-structured transition systems}


Let $S$ be a partially ordered set endowed with a partial order $\leq$. Recall that two elements $a,b\in S$ are said to be \emph{comparable} if $a\leq b$ or $b\leq a$; otherwise, they are \emph{incomparable}.
%, denoted as $a \incomp b$.
%\jl{\incomp is not used in this paper}
Moreover, we use $a < b$ to denote that $a \leq b$ and $a \neq b$. Likewise, $a \geq b$ if $b \leq a$ and $a > b$ if $a \geq b$ and $a \neq b$.
% they are called ; that is, $a\eqcirc b$ if neither $a\le b$ nor $b\le a$.
%
%An \emph{antichain} in $S$ is a subset $A\subseteq S$ in which every two different elements are incomparable. 
An \emph{antichain} in $S$ is a sequence $a_1, a_2, \cdots$ of elements from $S$ where $a_i$ and $a_j$ are incomparable for every $ i < j$.
%every pair of distinct elements are incomparable, i.e. 
%$a_i \incomp a_j$ for $i \neq j$. 
%\jl{\incomp is not used in this paper}
%; that is, there is no order relation between any two different elements in $A$. 
A \emph{descending chain} in $S$ is a sequence $a_1, a_2, \cdots$ where $a_1 \geq a_2 \geq \cdots$. It is \emph{strictly descending} if $a_1 > a_2 > \cdots$. 
%An \emph{ascending chain} in $S$ is a sequence $a_1, a_2, \cdots$ where $a_1 \leq a_2 \leq \cdots$. It is \emph{strictly ascending} if $a_1 < a_2 < \cdots$. 
A (strictly) \emph{ascending} chain can be defined accordingly.  
%
The partial order $\leq$ is said to satisfy %the \emph{ascending chain condition} (ACC) \tl{if not used, remove it?} if it contains \emph{no infinite} strictly ascending chains. Similarly, $S$ is said to 
the \emph{descending chain condition} (DCC) if $(S, \leq)$ contains \emph{no infinite} strictly descending chains.
The partial order $\leq$ is a \emph{well-partial-order} (abbreviated as wpo) on $S$ if it satisfies DCC and contains no infinite antichains.  For instance, the standard ``less than or equal to'' relation $\leq$ is a well-partial-order on $\Nat$; it is not a well-partial-order on $\Int$, since $(\Int, \leq)$ contains a strictly descending chain $-1, -2, -3, \cdots$. 
% since it contains neither infinite strictly descending chains nor infinite antichains. 
%But it is not dually well-ordered as it contains an infinite strictly ascending chain $0<1<2<\cdots$. On the other hand, %$(-\Nat, \le)$ is 
%the set of negative numbers is dually well-ordered, but not well-ordered. %, where $-\Nat$ is the set of non-positive integers.
%A partial order $\leq$ on $S$ is called a \emph{well-order} if $\leq$ is a wpo and meanwhile a total order on $S$. 



%An infinite sequence $a_1, a_2, \cdots$ of elements of $S$ is \emph{``good''} if there are $i < j$ such that $a_i \le a_j$. An infinite sequence is \emph{``bad''} if it is not good.
%
%\begin{lemma}[\cite{SS12}]\label{lem-wpo}
%Let $(S, \leq)$ be a partially ordered set. Then the following three conditions are equivalent: 
%\begin{itemize}
%\item $\leq$ is a wpo on $S$,
%\item each infinite sequence $a_1, a_2, \cdots$ of elements in $S$ is good,
%\item each infinite sequence $a_1, a_2, \cdots$ of elements in $S$ contains an infinite increasing subsequence, namely, there are  $n_1 < n_2 < \cdots$ such that $a_{n_1} \le a_{n_2} \le \cdots$.
%\end{itemize}
%\end{lemma}

\begin{definition}[Downward-WSTS]
A downward well-structured transition system (downward-WSTS) $\dwsts$ is a triple $(S, \rightarrow, \le)$ such that 1) $\le$ is a wpo\footnote{A more general notion than wpo, i.e. well-quasi-order, was used to define WSTS \cite{FS01}. } over $S$, 2) $\rightarrow \subseteq S \times S$ is downward-compatible wrt $\le$, namely, for all $s \ge s'$ and $s \rightarrow t$, there is $t' \in S$ such that $s' \rightarrow^* t'$ and $t \ge t'$ (where $ \rightarrow^*$ denote the reflexive and transitive closure of $\rightarrow$).
\end{definition}


Let $\dwsts = (S, \rightarrow, \le)$ be a downward-WSTS. For $s \in S$, the set of successors of $s$, denoted by $\Succ(s)$, is defined as $\{s' \in S \mid s \rightarrow s'\}$.
\begin{itemize}
\item $\dwsts$ is called \emph{downward reflexive-compatible} if for all $s \ge s'$ and $s \rightarrow t$, either $t \ge s'$ or there is $t'$ such that $s' \rightarrow t'$ and $t \ge t'$.
%
\item $\dwsts$ admits \emph{effective} $\Succ$ if for every $s \in S$, $\Succ(s)$ is effectively computable from $s$.
%
\item $\dwsts$ admits \emph{decidable $\le$} if there is a procedure to decide $s \le t$ for all $s, t \in S$.
%
\end{itemize}

The \emph{sub-covering} problem of $\dwsts$ is to decide, for given $s, t \in S$, whether %starting from $s$ it is possible to sub-cover $t$, that is, 
there is $t' \in S$ such that $s \rightarrow^* t'$ and $t \ge t'$.

\begin{theorem}[\cite{FS01}, Theorem 5.5]\label{thm-dwsts}
The sub-covering problem is decidable for downward-WSTSs with 1) downward reflexive compatibility,  2) effective $\Succ$, and 3) decidable $\le$.
\end{theorem}



%%%%%%%%%%%%%%%%%%%%%%%%%%%%%%%%%%%%%%%%%%%%%%%%
%%%%%%%%%%%%%%%%%%%%%%%%%%%%%%%%%%%%%%%%%%%%%%%%
%\hide{
%The $\Pp$-MA $\Aa^{post^*} = (Q'', \Gamma, \delta', I, F)$ is computed by the following procedure.
%\begin{enumerate}
%\item Let $Q''_0 := Q'$, $\delta'_0:=\delta$, and $i:=0$.
%
%\item Iterate the following procedure, until $Q''_{i} = Q''_{i-1}$ and $\delta'_{i} = \delta'_{i-1}$.
%\begin{enumerate}
%\item Let $Q''_{i+1}:=Q''_{i}$, $\delta'_{i+1}:=\delta'_{i}$.
%%
%\item For each $(q, \gamma, w, q') \in \Delta$,
%\begin{itemize}
%    \item if $w = \varepsilon$, then for each $(q, \gamma, q'') \in \delta'_{i}$ and $(q'', \gamma', q''') \in \delta'_{i}$, let $\delta'_{i+1}: = \delta'_{i+1}  \cup \{(q', \gamma', q''')\}$,
%    %
%    \item if $w= \gamma'$, then for each $(q, \gamma, q'') \in \delta'_i$, let  $\delta'_{i+1}: = \delta'_{i+1}  \cup \{(q', \gamma', q'')\}$,
%    %
%    \item if $w = \gamma' \gamma''$, then for each $(q, \gamma, q'') \in \delta'_i$, let $Q''_{i+1} := Q''_{i+1} \cup \{\langle q', \gamma' \rangle\}$ and $\delta'_{i+1}: = \delta'_{i+1} \cup \{(q', \gamma', \langle q',\gamma' \rangle )\} \cup \{( \langle q',\gamma' \rangle, \gamma'', q'')\}$.
%\end{itemize}
%\item Let $i:=i+1$.
%\end{enumerate}
%\item Let $Q'': =Q''_i$ and $\delta':=\delta'_i$.
%\end{enumerate}
%From the construction, we observe that in $\Aa^{post^*}$, $Q'' \subseteq Q' \cup (Q \times \Gamma)$. 
%\tl{to see how important it will be}
%}
%%%%%%%%%%%%%%%%%%%%%%%%%%%%%%%%%%%%%%%%%%%%%%%%
%%%%%%%%%%%%%%%%%%%%%%%%%%%%%%%%%%%%%%%%%%%%%%%%





%%%%%%%%%%%%%%%%%%%%%%%%%%%%%%%%%%%%%%%%%%%%%%%%%%%%%%%%%%%%%%%%%%%%%%%%%%%%%%%%%%%%%

\section{Android stack machine} \label{sec:amass}

%!TEX root = main.tex

In this section, we introduce \textbf{A}ndroid \textbf{S}tack \textbf{M}achine (\AMASS), a formal model to capture the Android multitasking mechanism. 
%The {\AMASS} model is the same as that in \cite{HC+19} and strictly extends that in \cite{CHS+18}.  
%is inspired by the previous work~\cite{ChenHSWWY18,HCWWY19}, but the model significantly deviates from the ASM therein. 
%
Following the overview of Section~\ref{sec:amm}, we concentrate on the launch mode, task affinity and intent flags when an activity is launched.  

There are four launch modes in Android: ``standard (STD)'', ``singleTop (STP)'', ``singleTask (STK)'' and ``singleInstance (SIT)''. Furthermore, Android provides 23 intent flags related to activities,\footnote{https://developer.android.com/reference/android/content/Intent\#flags}
%\cite{intent-flags}, 
namely, the flags whose names start with $\rm FLAG\_ACTIVITY$. Among these 23 intent flags, we consider the following five that are commonly used in Android apps, namely,
\begin{itemize}
	\item $\rm FLAG\_ACTIVITY\_NEW\_TASK$ ($\ntkflag$),
	\item $\rm FLAG\_ACTIVITY\_CLEAR\_TOP$ ($\ctpflag$),
	\item $\rm  FLAG\_ACTIVITY\_SINGLE\_TOP$ ($\stpflag$),
	\item $\rm  FLAG\_ACTIVITY\_CLEAR\_TASK$ ($\ctkflag$),
	% \item $\rm FLAG\_ACTIVITY\_MULTIPLE\_TASK$ ($\mtkflag$),
	\item $\rm FLAG\_ACTIVITY\_REORDER\_TO\_FRONT$ ($\rtfflag$),
	% \item $\rm FLAG\_ACTIVITY\_TASK\_ON\_HOME$ ($\tohflag$).	
	%\item $FLAG\_ACTIVITY\_EXCLUDE\_FROM\_RECENTS$ (FAEFR),
	%	\item $\dots$.
\end{itemize}

\subsection{Android Stack Machine}

Let $\flagset=\{\ntkflag, \ctpflag, \stpflag, \ctkflag, \rtfflag \}$ denote the set of intent flags, $\bool(\flagset)$ denote the set of formulae $\phi = \bigwedge \limits_{F \in \flagset} \theta_F$, where $\theta_F = F$ or $\neg F$. For convenience, we use $\bot$ to denote $ \bigwedge \limits_{F \in \flagset} \neg F$. 

\begin{definition}[Android Stack Machine, \AMASS] \label{def:afsm}
An {\AMASS} is a tuple $\Mm = (\act, A_0, \lmd, \aft, \Delta)$, where 
\begin{itemize}
\item $\act$ is a finite set of activities, and $A_0 \in \act$ is the main activity. Let $m=|\act|$.
\item $\lmd : \act \rightarrow \{\standard,\singletop,\singletask,\singleinstance\}$ is the launch-mode function,
%
\item $\aft : \act \rightarrow [m]$ is the task-affinity function, 
% for each activity $A$ with $\lmd(A) = \SIT$, $\aft(A)$ is unique,
%
\item $\Delta \subseteq \{\back\} \cup \act \times \{\startactivity(A,\phi)\mid A \in \act, \phi \in \bool(\flagset)\}$ is a finite set of transition rules.
%$(\beta_1(F_1, i_1), \cdots, \beta_k(F_k, i_k))$ such that for each $j \in [k]$, $\beta_j \in \{\ADD,\REM, \REP\}$, $F_j \in \frag$, and $i_j \in \Nn$. 
\end{itemize}
%
\end{definition}
We use $\act_{\star}$ to denote $\{B\in\act\mid \lmd(B)= \star\}$ for $\star\in\{\STD,\STP,\STK,\SIT\}$.
For readability, we write a transition rule $(A, \startactivity(B, \phi))\in \Delta$ as $A \xrightarrow{\startactivity(\phi)} B$


We call an activity of the launch mode $\STD$ as an $\STD$-activity, similarly for $\STP$, $\STK$, and $\SIT$. 
%%%%%%%%%%%%%%%%%%%%%%%%%%%%%%%%%%%%%%%%%%%
%\subsection{Semantics of \AMASS}\label{sec:semantics-amass}


%Let us first introduce some notations to be used in the definition of the semantics.

\paragraph{Tasks and configurations} A \emph{task} of $\Mm$ is encoded as a pair $(S, \aname)$, where $S= [A_1, \cdots, A_n] \in \act^+$ denotes the content of the stack, with $A_1$ (resp. $A_n$) as  the top (resp. bottom) symbol, denoted by $\topact(S)$ (resp. $\btmact(S)$), and $\aname$ is the affinity of the task, namely, the affinity of the activity which was pushed into the task when the task was created. A task $(S, \aname)$ is called an $\SIT$-task if $\lmd(\btmact(S)) = \SIT$.
%We define the \emph{affinity of a task} $S$, denoted by $\aft(S)$, to be $\aft(\btmact(S))$. 
For $S_1 \in \act^+$ and $S_2 \in \act^+$, we use $S_1 \cdot S_2$ to denote the concatenation of $S_1$ and $S_2$.

A \emph{configuration} of $\Mm$ is $\rho = ((S_1,\aname_1), \cdots, (S_n,\aname_n))$ such that
\begin{itemize}
\item for each $i \in [n]$, $(S_i,\aname_i)$ is a task, $S_i = [A_{i,1}, \cdots, A_{i, m_i}]$ is a task and $\aname_i$ is the task affinity of $S_i$ for each $i \in [n]$, moreover, 
%
\item for each $1 \le i < j \le n$ such that $(S_i, \aname_i)$ and $(S_j, \aname_j)$ are non-$\SIT$ tasks,  we have $ \aname_i \neq \aname_j$.  (Intuitively, the task affinities of every two non-$\SIT$ tasks are different. )
\end{itemize}
The task $(S_1,\aname_1)$ and $(S_n,\aname_n)$ are called the \emph{top task} and the \emph{bottom task} respectively. 
%The \emph{task affinity} of a task is the task affinity of the activity which was pushed into the task as the bottom activity when the task is created. 


% Let $\conf_{\Mm}$ denote the set of configurations of $\Mm$. 
Let $\confs(\Mm)$ denote the set of configurations of $\Mm$.
The initial configuration is $(([A_0],\aft(A_0)))$ which contains one task $([A_0], \aft(A_0))$ only. 
% Let $\confs(\Mm)$ denote the set of configurations $\rho$ such that $([A_0])\rightarrow_{\Mm}^*\rho$.

%We define the relation $\xrightarrow[\Mm]{}$ which comprises the quadruples $(\rho, \tau, \rho') \in \conf_\Mm \times \Delta  \times \conf_\Mm$ to formalise the semantics of $\Mm$. For readability, we write $(\rho, \tau, \rho')\in\ \xrightarrow[\Mm]{}$  as $\rho \xrightarrow[\Mm]{\tau} \rho'$.

We shall define the semantics of an {\AMASS} $\Mm$ as a transition relation $\rho \xrightarrow[\Mm]{\tau} \rho'$, where $\tau \in \Delta$ is a transition rule, $\rho, \rho' \in \confs(\Mm)$ and $\rho'$ is obtained by applying $\tau$ to $\rho$. As usual, $\xRightarrow[\Mm]{}$ denotes the reflexive and transitive closure of $\xrightarrow[\Mm]{}$.

%%%%%%%%%%%%%%%%
%\tl{to see where to put this.} Moreover, it is not hard to verify that the transition relation $\xrightarrow[\Mm]{\tau}$ preserves the invariant of configurations, that is, if $\rho$ satisfies that the task affinities of every two non-$\singleinstance$ tasks are different, and $\rho \xrightarrow[\Mm]{\tau} \rho'$ for some $\tau$, then $\rho'$ satisfies this invariant as well. 
%%%%%%%%%%%%%%%%%%%%%%



%\tl{end of comments: to see where to put this.} 


It turns out that the semantics of {\AMASS} is highly complicated due to the complex interplay between launch modes and intent flags. %Therefore, in the sequel, 
To make it more accessible, we %separate the concerns further and 
shall consider two sub-models of $\AMASS$, namely, $\LMAMASS$ and $\IFAMASS$, which focus on launch modes and intent flags of $\AMASS$ respectively. 
More precisely, 
\begin{itemize}
	\item an $\LMAMASS$ is an $\AMASS$ $\Mm = (\act, A_0, \lmd, \aft, \Delta)$ where all the transition rules $A \xrightarrow{\alpha(\phi)} B$ (except $\back$) satisfy that $\phi = \bot$, 
	\item an $\IFAMASS$ is an $\AMASS$ $\Mm = (\act, A_0, \lmd, \aft, \Delta)$ where all the activities $A\in\act$ satisfy that $\lmd(A) = \STD$ and $\aft(A) = \aft(A_0)$.
\end{itemize}
%To ease the understanding, in the sequel, we shall define the semantics of the two sub-models $\LMAMASS$ and $\IFAMASS$ first, then define the semantics of $\AMASS$. 



	
%%%%%%%%%%%%%%%%%%%%%%%%%%%%%%%%%%%%%%%%%%%%%%%%%%%%%%%%%%%%%%%%%%%%%%%%%%%%%%%%%%%%%%

%%%%%%%%%%%%%%%%%%%%%%%%%%%%%%%%%%%%%%%%%%%%%%%%%%%%%%%%%%%%%%%%%%%%%%%%%%%%%%%%%%%%%%

%As we have seen, 
%The semantics of {\AMASS} %in Section~\ref{sec:amass} 
%is rather complicated and it is highly challenging to achieve a decision procedure for the %configuration reachability problem of {\AMASS}. 
%
In this paper, to study the configuration reachability, we %restrict our attention to a sub-model of {\AMASS}, called 
concentrate on $\STK$-dominating {\AMASS} which imposes further constraints on {\AMASS}. %, and design a decision procedure for it. 
%We shall see later on that the decision procedure for the sub-model is already quite involved and requires nontrivial novel ideas.  
	
	%Due to the semantics of {\AMASS} defined in Section~\ref{sec:semantics-amass} is highly complicated, it is difficult to provide a decision algorithm for solving the configuration reachability problem of {\AMASS}. To this end, we introduce a fragment of {\AMASS} called STK-dominating {\AMASS}, which accommodates all launch mdoes and intent flags. 
	
\begin{definition}[$\STK$-dominating \AMASS] \label{def:stk-amass}
Let $\Mm = (\act, A_0, \lmd, \aft, \Delta)$ be an $\AMASS$. Then $\Mm$ is  said to be \emph{$\STK$-dominating} if the following constraints are satisfied.
		\begin{enumerate}
			\item $\back\in\Delta$,
			\item the task affinities of $\singletask$-activities are mutually distinct,
			%        \item for each transition $A \xrightarrow{\startactivity(\phi)} B\in\Delta$ such that $A\in\act_{\singleinstance}$, it holds that $B\in\act_{\singleinstance}\cup\act_{\singletask}$,
			\item for each transition $A \xrightarrow{\startactivity(\phi)} B\in\Delta$ such that $A \in \act_\singleinstance$ or $\phi\models \ntkflag$, $B \in \act_{\singleinstance} \cup \act_{\singletask}$.
		\end{enumerate}
\end{definition}

Intuitively, in an $\STK$-dominating $\AMASS$, the task creation or task switching actions are all triggered by the $\singleinstance$ or $\singletask$ launch mode. More precisely,  
each task of $\Mm$ contains at most one $\STK$-activity, and  %whenever a task creation or task switching action occurs, the launched activity is of $\singleinstance$ or $\singletask$ launch mode, equivalently speaking, 
whenever an $\standard$ or $\singletop$-activity is started, no task creation or task switching actions occur.  %This intuition is formalized by the proposition below.
	%For the two sub-models of $\AMASS$, i.e., $\LMAMASS$ and $\IFAMASS$, an $\LMAMASS$ (resp. $\IFAMASS$) is said to be $\STK$-dominating if the above four constraints are satisfied.
	
\begin{example}
		We  use the following example to illustrate the concept of $\STK$-dominating $\AMASS$.
		Let $\Mm=(\act,A,\lmd,\aft,\Delta)$, where 
		\begin{itemize}
			\item $\act = \{A, B, C, D, E\}$, 
			%
			\item $\lmd(A)=\lmd(B)=\standard$, $\lmd(C) = \singleinstance$, $\lmd(D) = \lmd(E) = \singletask$, 
			%
			\item $\aft(A) = 0$, $\aft(B) = \aft(C) = 1$, $\aft(D) = 2$, $\aft(E) = 3$, 
			%
			\item $\Delta = \{\back,\tau_1, \tau_2, \tau_3, \tau_4, \tau_5, \tau_6\}$, where 
			\begin{itemize}
				\item 	$\tau_1 = A \xrightarrow{\startactivity(\ntkflag)} C$,
				$\tau_2 = B \xrightarrow{\startactivity(\ntkflag)} E$,
				$\tau_3 = C \xrightarrow{\startactivity(\bot)} D$,
				\item		$\tau_4 = C \xrightarrow{\startactivity(\bot)} C$,
				$\tau_5 = D \xrightarrow{\startactivity(\ctpflag)} B$,
				$\tau_6 = E \xrightarrow{\startactivity(\bot)} D$. 
			\end{itemize}
		\end{itemize}
		Then $\Mm$ is an $\STK$-dominating {\AMASS}, since
		\begin{itemize}
			\item in $\tau_1 = A \xrightarrow{\startactivity(\ntkflag)} C$, $\lmd(C) = \singleinstance$,  
			%
			\item in $\tau_2 = B \xrightarrow{\startactivity(\ntkflag)} E$, $\lmd(E) = \singletask$, 
			%
			\item in $\tau_3 = C \xrightarrow{\startactivity(\bot)} D$, $\lmd(C) = \singleinstance$ and $\lmd(D) = \singletask$, 
			%
			\item in $\tau_4 = C \xrightarrow{\startactivity(\bot)} C$, $\lmd(C) = \singleinstance$. 
		\end{itemize}
		The configurations that are reachable from the initial configuration $(([A_0], 0))$ are illustrated in Figure~\ref{stk-asm-example}, where the vertices denote the configurations and the edges denote the elements of $\xrightarrow[\Mm]{}$. 
		Note that for $\Mm$, there are only finitely many configurations reachable from the initial configuration.  
\begin{figure}[htbp]
			% \vspace{-3mm}
			\centering
			\includegraphics[scale = 0.5]{stk-asm-example.pdf}
			\caption{Configurations that are reachable from the initial configuration $(([A_0], 0))$.} % in the $\STK$-dominating $\AMASS$ $\Mm$}
			%in $\phi$
			% \vspace{-6mm}	
			\label{stk-asm-example}
\end{figure}
\end{example}


%%%%%%%%%%%%%%%%%%%%%%%%%%%%%%%%%%%%%%%%%%%%%%%%%%%%%%%%%%%%%%%%%%%%%%%%%%%%%%%%%%%%%%

\subsection{Configuration reachability of $\AMASS$} \label{sec-conf-reach}

Let $\Mm =(\act, A_0, \lmd, \aft, \Delta)$ be an  {\AMASS}. We define $\RConfs(\Mm)$ as the set of configurations $\rho$ that are reachable from the initial configuration, that is, 
$(A_0, \aft(A_0)) \xRightarrow[\Mm]{} \rho$. 
Let $(\Aut_1,\cdots,\Aut_k)$ be an {\NFA} tuple over the alphabet $\act$. We define $\Rel((\Aut_1, \cdots, \Aut_k))$ as $(S_1, \cdots, S_k)$ such that for each $i \in [k]$, $S_i \in \Lang(\Aut_i)$. Moreover, let $\theta = \aname_1\cdots\aname_k$ be an affinity sequence. Then $\confs((\theta, (\Aut_1,\cdots,\Aut_k)))$, \emph{the set of configurations accepted by $(\theta, (\Aut_1,\cdots,\Aut_k))$}, is defined as as the set of configurations $\rho = ((S_1,\aname'_1),\cdots,(S_k,\aname'_k))$  such that $(S_1, \cdots, S_k) \in \Rel((\Aut_1, \cdots, \Aut_k))$, and for each $i \in [k]$, $\aname_i=\aname_i'$.
%
% $\confs((\theta, (\Aut_1,\cdots,\Aut_k)))$, \emph{the set of configurations accepted by $(\theta, (\Aut_1,\cdots,\Aut_k))$}, as the set of configurations $\rho = ((S_1,\aname'_1),\cdots,(S_k,\aname'_k))$  such that for each $i \in [k]$, $\aname_i=\aname_i'$ and $S_i \in \Lang(\Aut_i)$. 

\emph{The configuration reachability problem} of {\AMASS} is defined as follows. 
\begin{quote}
	Given an {\AMASS} $\Mm= (\act, A_0, \lmd, \aft, \Delta)$, an affinity sequence $\theta = \aname_1\cdots\aname_k$, and an {\NFA} tuple $(\Aut_1,\cdots,\Aut_k)$ over the alphabet $\act$, decide whether $ \confs({(\theta, (\Aut_1,\cdots,\Aut_k))}) \cap \RConfs(\Mm) \neq \emptyset$.
\end{quote}

The ultimate goal of the paper is to show the decidability of the configuration reachability problem of $\STK$-dominating $\AMASS$  stated in Theorem~\ref{thm:st-amass-reach}. However, we shall proceed in multiple steps given in Section~\ref{sec:reach-lmamass}--\ref{sec:reach-amass}.

For technical convenience, we assume that \emph{the affinities of $\singleinstance$-activities are mutually distinct and are different from those of the other activities}, that is, for $A, B \in \act_\singleinstance$ with $A \neq B$, we assume that  $\aft(A) \neq \aft(B)$ and $\aft(\act_\singleinstance) \cap \aft(\act \setminus \act_\singleinstance) = \emptyset$.
%
Note that this assumption is not a restriction, since regarding the task allocation mechanism, the affinities of $\singleinstance$-activities are redundant and independent of those of the other activities. 

%With this assumption, the task allocation mechanism (formalized by $\gettsk(\rho, B)$) can be specified by the \emph{name function} $\namefun_B(\theta)$ defined in the sequel:  Let $\theta = \aname_1 \cdots \aname_k$ and $B \in \act$. Then $\namefun_B(\theta) = i$ if $\aft(B) = \aname_i$ for some $i \in [k]$, and $\namefun_B(\theta) = \bot$ otherwise. 

%Note that the invariant of configurations, that is, the task affinities of every two non-$\singleinstance$ tasks are mutually distinct, guarantees the well-definedness of $\namefun_B(\theta)$.
%%%%%%%%%%%%%%%%%%%%%%%%%%%%%%%%%%%%%%%%%%%%%%%%%%%%%%%%%%%%%%%%%%%%%%%%%%%%%%%%%%%%%%



%===================================================================================================
% \section{Doubly-Nested Stack Systems} \label{sec:dnss}

% In this section, we introduce the model of Doubly-Nested Stack Systems (\DNSS).  
\begin{definition}[Doubly-Nested Stack Systems] \label{def:eps}
    A {\DNSS} is a tuple $\Dd = (\Gamma,\Gamma_\STK, \gamma_{\init}, \Delta)$, where $\Gamma$ is a finite stack alphabet with $\gamma_\init\in \Gamma$, and $\Delta \subseteq \{\POP\} \cup \Gamma \times \opset$ is the transition relation with 
    $\opset=\{\opn(\gamma)\mid \opn \in \{\PUSH, \STP, \CTP, \RTF\}, \gamma \in \Gamma\setminus\Gamma_\STK \} \cup \{\STK(\gamma)\mid \gamma\in\Gamma_\STK\}$. Moreover, it is required that $\POP \in \Delta$. (Intuitively, the $\POP$ operation can be applied anytime.)
	%\cup \Gamma \cup\{\back\}$
\end{definition}

%--------------------------------------------------------
Intuitively, a {\DNSS} specifies the evolution of a stack of stacks. Inspired by the Android multitasking mechanism, the lower-level stacks are referred to as \emph{tasks} and the upper-level stack, possibly comprising multiple tasks, is referred to as the \emph{task stack}. More formally, a task $S$ is an element of $\Gamma^*$, i.e., a sequence of symbols from $\Gamma$, with the leftmost symbol representing the top of the task; A task stack $\rho$ is of the form $(S_1,\cdots, S_n)$, where each $S_i$ is a task and $S_1$ is the topmost task. When necessary we use $\emptybackstack$ to denote an empty task stack, i.e., when $n=0$.

To precisely define the semantics of a {\DNSS}, we need some additional machinery. We assume that there is a finite \emph{name space} $\namespace$ such that each task $S$ is associated with a \emph{name} $\namefun(S)\in \namespace$. 
%
%Notice that the concrete definition will be given when a concrete system is instantiated. 
We assume that the name $\namefun(S)$ is determined when the task $S$ is created (i.e., being pushed into the task stack as a new task). Therefore, the name function $\namefun$ can also be seen as a function from $\Gamma$ to $\namespace$, that is, it maps from the first symbol of the task (which is pushed into the task when it is created) to some name in $\namespace$. 
%
Throughout  this paper we will insist this variant of $\namefun$. 
%\zhilin{add this explanation of $\namefun$.} 
Notice that distinct tasks may share the same name. Concrete definitions of the name space $\namespace$ and the name function $\namefun$ vary and are largely domain-specific. For instance, in the {\DNSS} dedicated to the Android multitasking mechanism,  the name of a task is defined as a pair $(A, s)$, where $A$ is the first \emph{activity} of the task when it is launched (this is called \emph{real activity} of the task) and $s$ is the \emph{task affinity} of $A$; we defer the detailed explanations to Section~\ref{sec:asm}.  

In \DNSS, a task stack $\rho=(S_1, \cdots, S_n)$ encompasses multiple tasks. When an operation $o \in \opset$ is to be executed, a distinguished feature of the {\DNSS} model---different from most of the other models based on pushdown systems---is that $\opset$ is not necessarily applied to the top task $S_1$, but may be applied to some task $S_i$ for $1<i\leq n$, depending on the names of all $S_i$'s given by the name function $\namefun$.\footnote{Readers might have already noticed that some operations in $\opset$, i.e., $\PUSH, \STP, \CTP, \RTF$, have two version with one being annotated with the @ symbol. The operation with @ is applied to the top task directly, while the one with @ needs to invoke the addressing mechanism.} To this end, we 
need to define an \emph{addressing} mechanism, i.e., to prescribe to which task the operation $o$ should be applied. In \DNSS, there are two possibilities: either a task is located to which $\op$ is then applied, or no task can be selected in which case a new task must be created. 

%given a sequence (stack) of stacks and $\gamma\in \Gamma$, which stack is to be operated? In our model,  
We consider an addressing mechanism which bases the decision on the word $\namefun(S_1)\cdots \namefun(S_n) = \namefun(\gamma_1)\cdots \namefun(\gamma_n) \in \namespace^+$ (where for each $i \in [n]$, $\gamma_i$ is the first symbol pushed to the task $S_i$ when $S_i$ was created) and the stack symbol $\gamma$ contained in $\opn(\gamma) \in \opset$ or $\opn^\addr(\gamma) \in \opset$. Formally, we define $(\alpha_\gamma)_{\gamma\in \Gamma}$, where % for each $\gamma\in \Gamma$, there is an addressing 
each function $\alpha_\gamma: \namespace^+\rightarrow \natnum\cup\{\bot\}$  is subject to the following two conditions, viz.  
\begin{itemize} 
	\item[(a)] $\alpha_\gamma(w)\in [|w|]\cup\{\bot\}$,  and 
	\item[(b)] for each $w \in \namespace^+$, $\namefun(\gamma)$ occurs in  $w$ entails that $\alpha_\gamma(w) \neq \bot$. 
\end{itemize}
%\zhilin{add this condition to guarantee that the number of tasks is bounded when $\LTK$ is missing. }. 
%In case that $\alpha_\gamma(w)$ returns the index of the lower-level stack in the upper-level stack that will be operated, or otherwise---in case of returning $\bot$, a new lower-level stack will be created. 
As mentioned, concrete definitions of $(\alpha_\gamma)_{\gamma\in \Gamma}$ vary; for the Android multitasking mechanism, %Concretely, the addressing machinery, i.e., $\alpha_\gamma$ is 
we shall employ an automata model (tentative: Cost Register Automata, CRA). 

\subsection{Semantics}

As usual, we define the semantics of {\DNSS} as a transition system over configurations.
Configuration of {\DNSS} $\Dd$ are task stacks which are encoded as $\rho=(S_1,\cdots, S_n)$
with $S_i = [\gamma_{i,1}, \cdots, \gamma_{i, m_i}]$ for each $i \in [n]$. We define $\topact(\rho):=\gamma_{1,1}$. The \emph{initial} configuration is $([\gamma_{\init}])$ which contains one task $[\gamma_\init]$ only. The \emph{height} of a configuration $\rho$ is defined as the maximum height of tasks in $\rho$.
%As usual, the semantics of {\DNSS} is defined as 
The transition relation is denoted by $\rightarrow_{\Dd}$, defined in the sequel.  %over the set of all configurations. Namely, 
\begin{itemize}
\item $\POP \in \Delta$ induces the transitions $\rho\rightarrow_\Dd \rho'$ such that one of the following two conditions holds,
\begin{itemize}
\item  $\rho = (S_1, S_2, \cdots, S_n)$ and $\rho' = (S'_1, S_2, \cdots, S_n)$, $S_1 = [\gamma_{1, 1}, \cdots, \gamma_{1, m_1}]$ with $m_1 \ge 2$, and $S_1' = [\gamma_{1, 2}, \cdots, \gamma_{1, m_1}]$, 
%
\item $\rho = (S_1, S_2, \cdots, S_n)$, $S_1 = [\gamma_{1,1}]$, and $\rho' = (S_2, \cdots, S_n)$.
\end{itemize}
Note that the $\POP$ operation does not depend on the top symbol, thus can be applied anytime.

\item Each $(\gamma, o)\in \Delta$ induces transitions $\rho\rightarrow_\Dd \rho'$ with $\topact(\rho)=\gamma$. We now define the effect of the operation $o\in \opset$ on $\rho$, i.e., the resulting $\rho'$. 
    Recall that $\opset=\{\opn(\gamma)\mid \opn \in \{\PUSH, \STP, \RTF, \CTP\}, \gamma \in \Gamma\setminus\Gamma_\STK \} \cup \{\STK(\gamma)\mid \gamma\in\Gamma_\STK\}$.
\begin{itemize}
	\item $\PUSH(\gamma)$. This operation is similar to the push operation in traditional pushdown systems, namely, $\gamma$ is simply pushed into the top task $S_1$ where no addressing mechanism is needed.	 
  	\item $\STP(\gamma)$. This is a shorthand for ``\textbf{S}ingle \textbf{T}o\textbf{P}''. It is the same as $\PUSH(\gamma)$, except that if $\gamma$ is already the top of $S_1$ then $\gamma$ is \emph{not} pushed into $S_1$ (and thus $\rho':=\rho$). 
%-------------------------------------------------------------------------------------------------------
	\item $\RTF(\gamma)$. This is a shorthand for ``\textbf{R}eorder \textbf{T}o \textbf{F}ront". If $\gamma$ occurs in $S_1$ whereby $S_1=[\gamma_{1,1}, \cdots, \gamma_{1,m_1}]$ with $\gamma_{1,j} =\gamma$ for some $j \in [m_1]$  and $\gamma_{1, j'} \neq \gamma$ for all $j': 1 \le j' < j$. Then $\RTF(\gamma)$ escalates $\gamma_{1,j}$ to be the top of $S_1$, resulting in  
	$\widetilde{S_1} = [\gamma_{1,j}, \gamma_{1,1}, \cdots, \gamma_{1,j-1}$, $\gamma_{1,j+1}, \cdots, \gamma_{1,m_1}]$.
	Otherwise, $\gamma$ does not occur in $S_1$, then $\RTF(\gamma)$ pushes $\gamma$ into $S_1$, resulting in $\widetilde{S_1} = [\gamma, \gamma_{1,1}, \cdots, \gamma_{1,m_1}]$. In either case,  $\rho':=(\widetilde{S_1}, S_2, \cdots, S_n)$.
%----------------------------------------------------------------------------------------
	
	\item $\CTP(\gamma)$. This is a shorthand for ``\textbf{C}lear \textbf{T}o\textbf{P}". Suppose $\gamma$ occurs in $S_1$ whereby $S_1=[\gamma_{1,1}, \cdots, \gamma_{1,m_1}]$ with $\gamma_{1,j} =\gamma$ for some $j \in [m_1]$ and $\gamma_{1,j'} \neq \gamma$ for all $j': 1 \le j' < j$. Then $\CTP(\gamma)$ removes all the symbols on top of $\gamma_j$ from $S_1$, resulting in $\widetilde{S_1} = [\gamma_{1,j}, \cdots, \gamma_{1,m_1}]$; %keep the other tasks of $\rho$ unchanged. 
%
Otherwise $\gamma$ does not occur in $S_1$, then $\CTP(\gamma)$ pushes $\gamma$ into $S_1$, resulting in $\widetilde{S_1} := [\gamma, S_1]$. In either case, $\rho':=(\widetilde{S_1}, S_2, \cdots, S_n)$.  
%-------------------------------------------------------------------------------------
\item $\STK(\gamma)$. This is a variant of the $\CTP$ operation with addressing. 
	 If $\alpha_\gamma$ returns $i \in [n]$, then $\STK(\gamma)$ first escalates $S_i$ to be top task of the task stack, resulting in $\rho''=(S_i, S_1, \cdots, S_{i-1}, S_{i+1}, \cdots, S_n)$, and then applies the operation $\CTP(\gamma)$ to $\rho''$ to obtain $\rho'$. 	 
	 Otherwise, $\alpha_\gamma$ returns $\bot$, then a new task $S'=[\gamma]$ is created and $\rho':=(S', S_1, \cdots, S_n)$.
%-------------------------------------------------------------------------------------
\end{itemize}
We will use $\rightarrow_\Dd^*$ to denote the reflexive and transitive closure of $\rightarrow_\Dd$.

\paragraph{Reachability problem} Given a $\DNSS$ $\Dd$ and a configuration $\rho = (S_1, \cdots, S_n)$, decide whether $\rho$ is reachable from the initial configuration in $\Dd$.
\end{itemize}







%===================================================================================================
% \section{Pushdown systems with well-partially-ordered transductions} \label{sec:wpotrpds}

% \input{wpotrpds.tex}

%===================================================================================================
%\section{Configuration reachability problem of $\STK$-dominating $\AMASS$} \label{sec:reach}

%!TEX root = main.tex

%For simplicity, we assume that $\Mm$ contains $\standard$ and $\singletask$ activities only. 
%The proof of Theorem~\ref{thm:st-amass-reach} is technically the most challenging part of this paper. 
%To ease the understanding, we shall prove the configuration reachability problem of $\STK$-dominating $\LMAMASS$ (resp. $\STK$-dominating $\IFAMASS$) is decidable in Section~\ref{sec:reach-lmamass} (resp. Section~\ref{sec:reach-ifamass}) first, then prove the configuration reachability problem of $\STK$-dominating $\AMASS$ is decidable in Section~\ref{sec:reach-amass}.


% Following the same approach as the semantic definition, we address the problem step by step in these three scenarios, i.e., $\LMAMASS$, $\IFAMASS$ and {\AMASS}.
% To tackle the configuration reachability problem for $\STK$-dominating $\AMASS$, we consider three case, i.e., 
% $(A,\STK(A'))\notin\Delta$, $\lmd(A_0)=\singletask$ and $\lmd(A_0)\neq\singletask$. 


%\section{Configuration reachability problem of $\AMASS$}\label{sec-conf-reach}
%
%In this paper, we focus on the configuration reachability problem of $\AMASS$. This problem is fundamental to the formal (static) analysis and verification of the behaviors of Android apps with respect to the multitasking mechanism.  
%
%Let $\Mm =(\act, A_0, \lmd, \aft, \Delta)$ be an  {\AMASS}. Then we define $\RConfs(\Mm)$ as the set of configurations $\rho$ that are reachable from the initial configuration, that is, 
%$(A_0, \aft(A_0)) \xRightarrow[\Mm]{} \rho$. 
%Let $(\Aut_1,\cdots,\Aut_k)$ be an {\NFA} tuple over the alphabet $\act$. We define $\Rel((\Aut_1, \cdots, \Aut_k))$ as $(S_1, \cdots, S_k)$ such that for each $i \in [k]$, $S_i \in \Lang(\Aut_i)$. Moreover, let $\theta = \aname_1\cdots\aname_k$ be an affinity sequence. Then $\confs((\theta, (\Aut_1,\cdots,\Aut_k)))$, \emph{the set of configurations accepted by $(\theta, (\Aut_1,\cdots,\Aut_k))$}, is defined as as the set of configurations $\rho = ((S_1,\aname'_1),\cdots,(S_k,\aname'_k))$  such that $(S_1, \cdots, S_k) \in \Rel((\Aut_1, \cdots, \Aut_k))$, and for each $i \in [k]$, $\aname_i=\aname_i'$.
%%
%% $\confs((\theta, (\Aut_1,\cdots,\Aut_k)))$, \emph{the set of configurations accepted by $(\theta, (\Aut_1,\cdots,\Aut_k))$}, as the set of configurations $\rho = ((S_1,\aname'_1),\cdots,(S_k,\aname'_k))$  such that for each $i \in [k]$, $\aname_i=\aname_i'$ and $S_i \in \Lang(\Aut_i)$. 
%Then \emph{the configuration reachability problem} of {\AMASS} is defined as follows. 
%\begin{quote}
%Given an {\AMASS} $\Mm= (\act, A_0, \lmd, \aft, \Delta)$, an affinity sequence $\theta = \aname_1\cdots\aname_k$, and an {\NFA} tuple $(\Aut_1,\cdots,\Aut_k)$ over the alphabet $\act$, decide whether $ \confs({(\theta, (\Aut_1,\cdots,\Aut_k))}) \cap \RConfs(\Mm) \neq \emptyset$.
%\end{quote}

 





%The rest of this paper is devoted to the proof of Theorem~\ref{thm:st-amass-reach}. 

%Since the decision procedure is involved, to ease the understanding, we present the decision procedure for $\singletask$-dominating $\LMAMASS$ first (Section~\ref{sec:reach-lmamass}), then $\singletask$-dominating $\IFAMASS$ (Section~\ref{sec:reach-ifamass}), and finally consider $\singletask$-dominating $\AMASS$ in its most general case (Section~\ref{sec:reach-amass}). 



%For simplicity, we assume that $\Mm$ contains $\standard$ and $\singletask$ activities only. 
%The proof of Theorem~\ref{thm:st-amass-reach} is technically the most challenging part of this paper. 
%To ease the understanding, we shall prove the configuration reachability problem of $\STK$-dominating $\LMAMASS$ (resp. $\STK$-dominating $\IFAMASS$) is decidable in Section~\ref{sec:reach-lmamass} (resp. Section~\ref{sec:reach-ifamass}) first, then prove the configuration reachability problem of $\STK$-dominating $\AMASS$ is decidable in Section~\ref{sec:reach-amass}.





%===================================================================================================


\section{Related work}
%
%!TEX root = main.tex


There are a plethora of extensions/variants of pushdown systems resulting from different modeling/verification needs of practical systems. 
We focus on the reachability problem while the more general CTL/LTL model checking problems are skipped. 

Pushdown systems (PDSs) are a classical model of computation which is an extension of finite state automata equipped with a stack of symbols. PDSs have played a prominent role in, among others, program analysis and verification, especially for modeling procedural (imperative) programs. In the past 20 years, new programming paradigms have renewed interests of extensions of PDSs to model important features of emerging programs in practice. For instance, 
multi-stack pushdown systems have been proposed for multi-threaded procedural programs \cite{QR05,BESS05,TMP07}, and higher-order pushdown systems have been utilized for higher-order functional programs \cite{HMOS08,HMOS17}.

Standard PDSs allow two types of stack operations only, i.e., push and pop. In practice, more general operations on stacks are often required in, e.g.,  conditional pushdown systems \cite{EKS03,LO10} and discrete-time pushdown systems \cite{AAS12}. Uezato and Minamide propose pushdown systems with transductions (TrPDSs, \cite{UM13}), where (letter-to-letter) transductions can be applied to update the stack. They are not of theoretical interests only: TrPDSs have been 
used in, e.g., procedural programs with call-by-references parameter passing \cite{SM+15,Song18}.


%Well-structured pushdown systems (WSPDS) are an extension of pushdown systems with (possibly infinite) well-quasi-ordered control states and  stack alphabets \cite{CO13,CO14}. It was shown therein that several subclasses of WSPDS is decidable, including recursive vector addition systems with states, multi-set pushdown systems, and WSPDS where the set of control states is finite. 

%While it seems possible to transform WPOTrPDS into WSPDS by putting the transductions into control states or stack alphabets, it is unclear whether WPOTrPDS can be casted into these decidable subclasses of WSPDS.
%While it seems possible to transform WPOTrPDS into WSPDS by putting the transductions into control states, the resulting WSPDS are beyond the aforementioned decidable subclasses in \cite{CO13}.
%pushdown systems: single stack, multi-stack, well-structured, pushdown systems with transductions,
%

Well-structured PDSs (WSPDSs \cite{CO13}) combine well-structured transition systems and PDSs, where the set of control states and/or the stack alphabet may be infinite but admit a well-quasi-order. WSPDSs are expressive as they subsume, e.g., recursive vector addition systems with states \cite{BouajjaniE13} and multi-set PDSs \cite{SenV06}. 
%and dense-timed PDSs \cite{CaiO14}. 
The reachability problem is undecidable for WSPDSs, but the coverability is decidable if the set of control states is finite. While it appears to be possible to transform a WPOTrPDS into a WSPDS by encoding the transductions in control states or stack alphabets, it is unclear whether  
the resulting WSPDS belongs to these decidable subclasses of WSPDS known in the literature insofar.

A timed pushdown automaton is a timed extension of a pushdown automaton.  
Besides the discrete-time pushdown systems mentioned in the introduction, 
dense-time pushdown systems, where real-valued ages can be assigned to stack symbols and stored into the stack, were also studied and shown to be decidable \cite{AbdullaAS12}. Moreover, it was shown that dense-time pushdown systems can be simulated by WSPDS and thus the decidability results of WSPDS can also be utilized to show their decidability \cite{CO14}.

Apart from TrPDS, there are other models that involve more general operations on stack symbols. 
PDSs with checkpoints, as an extension of PDSs, can check whether the stack contents belong to regular languages \cite{EsparzaKS03}. They can be simulated by TrPDSs with finite transduction closures.
%
Stack automata are an extension of pushdown automata where %besides the normal push and pop operations, 
the interior stack symbols can be read but not rewritten by an additional head \cite{GGH67}. 
%
In pushdown systems with an upper stack \cite{PDT17}, 
the symbols popped out of the stack remain in the stack and can be overwritten when new symbols are pushed.

Stack automata and pushdown systems with an upper stack are incomparable with WPOTrPDS: on the one hand, WPOTrPDSs allow more powerful updates on the stack; on the other hand, while stack automata and pushdown systems with an upper stack can be simulated by TrPDSs, the transduction closures therein, nevertheless, are infinite and do not have a well-partially-ordered union-basis.

Furthermore, pushdown systems where control states, stack alphabet, and transition relation, instead of being finite, are first-order definable in a fixed countably-infinite structure were studied \cite{ClementeL15}.   


%vector addition systems with states, counter machines,


There are other types of extensions of PDS which are remote from the current work. For instance, weighted PDSs and extended weighted PDSs were introduced  \cite{RepsSJM05, LalRB05} for data-flow analysis purpose. These two extensions associate transitions with elements from semiring domains. The reachability problem is decidable for bounded idempotent semirings. 

%(Extended) weighted PDSs and TrPDSs are quite different two computation models. At least, the elements from a semiring can neither inspect nor modify the stack content except the top most symbol on the stack. To
%overcome this problem, weighted pushdown systems with indexed weight domains were proposed in [24,28], which generalize weighted PDSs and TrPDSs.




 


%There are lots of works with context-sensitive infinite state systems. A process rewrite systems combines a PDS and a Petri net, in which vector additions/subtractions between adjacent stack frames during push/pop operations
%are prohibited [17]. With this restrictions, its reachability becomes decidable. A
%WQO automaton [9], is a WSTS with auxiliary storage (e.g., stacks and queues).
%It proves that the coverability is decidable under compatibility of rank functions
%with a WQO, of which an Multiset PDS is an instance. 

 




%We show that the reachability analysis can be addressed with the well-known saturation technique for the wide class of oligomorphic structures. Moreover, for the more restrictive homogeneous structures, we are able to give concrete complexity upper bounds. We show ample applicability of our technique by presenting several concrete examples of homogeneous structures, subsuming, with optimal complexity, known results from the literature. We show that infinitely many such examples of homogeneous structures can be obtained with the classical wreath product construction.

%===================================================================================================

\section{Conclusion}
% 
In this paper, we investigated the decidability of the configuration reachability problem of Android Stack Machines (\AMASS). We proposed a decision procedure for an expressive sub-model of {\AMASS}, that is, $\STK$-dominating {\AMASS}. The decision procedure integrates the following two main ideas: 1) We extended the saturation procedure of {\PDS} to multiple stacks. 2) We proposed a novel model of pushdown systems with well-partially-ordered transductions (\WOTrPDS) and showed that the reachability problem for {\WOTrPDS} is decidable.
In the future, we plan to continue to investigate the decidability of {\AMASS}. 

%\newpage
%\bibliographystyle{cas-model2-names}
\bibliographystyle{elsarticle-num-names} 
\bibliography{reference}



%

% Main text
%% Loading bibliography style file
%\bibliographystyle{model1-num-names}
%\bibliographystyle{plain}
%
%% Loading bibliography database
%\bibliography{}
%
%% Biography
%\bio{}
%% Here goes the biography details.
%\endbio
%
%\bio{pic1}
%% Here goes the biography details.
%\endbio
\newpage
%\appendix
% \input{appendix.tex}

\end{document}

