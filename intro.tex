%!TEX root = main.tex

%Motivation

Multi-tasking plays a central role in the Android platform. %, and certainly is a complicated issue.
%It is considerably different from, say, PC multi-tasking. \tl{why?}
%As a fundamental concept, according to Android documentation \cite{}, ``a task is a collection of activities that users interact with when performing a certain job". Here an activity usually refers to UI components,  %In other words,
%and a task contains activities that may belong to multiple apps.
%
%each app can run in one or multiple processes.\FU{I am not sure whether an app can run in multiple processes, usually an app is run in a sandbox.}
%The unique design of Android multitasking
Its unique design, via activities and back stacks, greatly facilitates organizing user sessions through tasks, and provides rich features such as handy application switching, background app state maintenance, smooth task history navigation (using the ``back" button), etc \cite{RZXWL15}. We refer the readers to Section~\ref{sec:amm} for an overview. 

Android multitasking mechanism has substantially enhanced user experiences of the Android system and promoted personalized features in app design. However, the mechanism is also notoriously difficult to understand. As a witness, it constantly baffles app developers and has become a common topic of question-and-answer websites (for instance, \cite{stackoverflow}).
%
%summarize related problems in stackoverflow as a table, e.g., question 3219726 . This may provides some ideas on what kind of properties can be checked by ASS.
%
%Surprisingly, the Android multitasking mechanism, despite its importance, has not been thoroughly studied before, let along a formal treatment. This has impeded further developments of computer-aided (static) analysis and verification for Android apps, which are indispensable for vulnerability analysis (for example, detection of task hijacking \cite{RZXWL15}) and app performance enhancement (for example, estimation of energy consumption \cite{HL+13}).
For the formalization of Android multitasking mechanism, the most complete existing formal semantics is the Android Stack Machine ({\AMASS}) \cite{HC+19} to capture the key features of Android multitasking mechanism. {\AMASS} addresses the behavior of Android \emph{back stacks}, a key component of the multitasking machinery, and their interplay with attributes of the activity and the intent flags. 
%In this paper, for these attributes we consider four basic \emph{launch modes}, i.e., standard ({\bf $\standard$}), singleTop ({\bf $\singletop$}), singleTask ({\bf $\singletask$}), singleInstance ({\bf $\singleinstance$}), and \emph{task affinities}. (For simplicity more complicated activity attributes such as \emph{allowTaskReparenting} will not be addressed in the present paper.)
%We remark that, however, the intent flags of activities are abstracted away, to keep the model as neat as possible.
%We believe that the semantics of ASM, specified as a transition system, captures faithfully the actual mechanism of  Android systems. For each case  of the semantics, we have created ``diagnosis" apps with corresponding launch modes and task affinities, and carried out extensive experiments using these apps, ascertaining its conformance to the %mechanism supported by the
%Android platform. 
(Details will be provided in Section~\ref{sec:amm}.)

%
%The model addresses certain key features of Android multi-tasking such as launchMode and taskAffinity, while skip the other attributes.
%From an engineering perspective,
%For Android, technically ASM can be viewed as the counterpart of pushdown systems with multiple stacks, which are the \emph{de facto} model for (multi-threaded) concurrent programs.
%ASM gives--to the best of our knowledge--a first formal semantics for Android's multi-tasking mechanism.
%Being rigours, this model opens a door towards a formal account of Android's multi-tasking mechanism, which  would greatly facilitate developers' understanding, freeing them from lengthy, ambiguous, elusive Android documentations. We remark that it is known that the evolution of Android back stacks could also be affected by the \emph{intent flags} of the activities. ASM does not address intent flags explicitly. However, %it can be easily adapted to simulate
%the effects of most intent flags (e.g., {\small $\sf FLAG\_ACTIVITY\_NEW\_TASK$, $\sf FLAG\_ACTIVITY\_CLEAR\_TOP$}) can be simulated by %since they are similar to those of
%launch modes, so this is \emph{not} a real limitation of ASM. %However, in this paper, we  %focuses on two attributes of activities, namely the launch mode and the task affinity;
%do not address more complicated activity attributes such as allowTaskReparenting,
%which are left as the future work.

Based on {\AMASS}, we also make the first step towards a formal analysis of Android multitasking apps by investigating the \emph{reachability problem} which is fundamental to all such analysis. {\AMASS} is akin to pushdown systems with multiple stacks, so it is perhaps not surprising that the problem is hard in general; 
%in fact, we show undecidability for most interesting fragments even with just two launch modes. %(See Theorem~\ref{thm:undec}---\ref{prop:tridec} for details.)
In the interest of seeking more expressive, practice-relevant decidable fragments,
%observe that $\standard/\singletop$ activities must be supported, and %$\singletask/\singleinstance$
%$\singleinstance$ activities are desirable.\zhilin{seems strange ?}  %commonly used in Android apps. %\footnote{statistics data?}.
%Although the combination of all of them is unfortunately undecidable,
%\tl{I am not satisfied here, any comments on how to improve?}.
%On top of that, we hypothesize that restricting $\singletask$ and $\singleinstance$ activities \emph{individually} is a promising way. To this end,
we identify a fragment \textbf{$\STK$-dominating {\AMASS}} which assumes 
$\STK$ activities have different task affinities and which further restricts the use of $\STK$ activities and the intent flag $\ntkflag$. This fragment covers a majority of open-source Android apps (e.g., from F-Droid) we have found so far. 

%\tl{claiming this fragment can cover most of open-resource apps we have found?shall we}\zhilin{how about the previous sentence ?}
One of our technical contributions is to give a decision procedures for the reachability problem of $\STK$-dominating {\AMASS}.
We also propose pushdown systems with well-partially-ordered transductions (WPOTrPDS), an extension of finite TrPDS, where the closure of transductions may be infinite, but admits a basis which is well-partially-ordered wrt the superset relation, we show that the configuration reachability of WPOTrPDS is decidable, and utilize WPOTrPDS to show that the reachability problem of $\STK$-dominating {\AMASS} is decidable.
%As a complement, we also studied a fragment \textbf{$\singleinstance$-acyclic-mediating ASS}, which include $\singleinstance$, but are free of $\singletask$, activities, subject to additional restrictions.
%The work, apart from independent interests in the study of multi-stack pushdown systems, lays a solid foundation for further (static) analysis and verification of Android apps related to multi-tasking, enabling model checking of Android apps, security analysis (such as discovering task hijacking), or typical tasks in software engineering such as  % Assist programmers with
%automatic debugging, model-based testing, etc.
%task-sensitive analysis of Android apps, generate testcases from the model and test consistent of variants of Android OS.

We summarize the main contributions as follows: (1) We propose pushdown systems with well-partially-ordered transductions (WPOTrPDS), and study the reachability problem for WPOTrPDS.
%---to the best of our knowledge---the first comprehensive formal model, Android stack machine, for Android back stacks, which %captures both launch modes and task affinities of activities. The model
%is also validated by extensive experiments. 
%	
	%, for the Android multi-task mechanism. To validate the conformance of the model with respect to the Android platform, we have created diagnosis apps and designed extensive experiments. To the best of our knowledge, Android stack systems is the first model for Android back stack systems that captures both the launch modes and task affinities of activities.
%	
%	The first model of this kind.
%
    (2) We study the reachability problem for Android stack machine ({\AMASS}). 
    %Apart from strongest possible undecidablity results in the general case, we %. Show  that the reachability problem is undecidable in general, and
	We provide the decision procedures for two practically relevant fragments. %l, by exploiting
	

