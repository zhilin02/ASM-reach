%!TEX root = main.tex

%Recall that the definition of $\STK$-dominating $\LMAMASS$, then we let $\tau= A\xrightarrow{\startactivity(\bot)}B$ and let $\rho = (S_1, \cdots, S_n)$ be the current configuration for some $n \ge 1$ and $\topact(\rho) = A$. 

%we define the semantics of $\STK$-dominating $\LMAMASS$ as follows:
%\begin{itemize}
%    \item If $\lmd(B) = \STD$, then $\rho' = \push(\rho, B)$,
%    \item If $\lmd(B) = \STK$, then 
%    \begin{itemize}
%        \item if $\gettsk(\rho, B) = S_i$ for some $i\in[n]$, then
%        \begin{itemize}
 %           \item if $B \not \in S_i$, then $\rho' = \push(\mvtsktop(\rho, i), B)$,
 %           \item if $B  \in S_i$, 	then $\rho' =  \clrtop(\mvtsktop(\rho, i), B)$,
 %       \end{itemize}
%    \item if $\gettsk(\rho, B) = *$, then $\rho' = \newtsk(\rho, B)$.
%    \end{itemize}
%\end{itemize}

Let us assume that $\Mm=(\act, A_0, \lmd, \aft, \Delta)$ is an $\singletask$-dominating $\LMAMASS$.

We show how to solve the configuration reachability problem of $\Mm$.  
Let $(\Aut_1,\cdots,\Aut_k)$ be an {\NFA} tuple over the alphabet $\act$, and $\theta = \aname_1\cdots\aname_k$ be an affinity sequence. 
Our goal is to decide whether there is a configuration $\rho$ that is reachable from the initial configuration and accepted by $(\theta, (\Aut_1,\cdots,\Aut_k))$.
% satisfying that for each $i\in[k]$, $\aname_i=\aname_i'$, $S_i\in\ConfSet(\Aut_i)$, moreover, $(([A_0],\aft(A_0)))\xRightarrow[\Mm]{}\rho$.

The main idea is to show that the set of configurations that are reachable from the initial configuration can be represented finitely as recognizable relations defined in the sequel. 

\begin{definition}[Recognisable relations]
A $k$-ary string relation $R \subseteq (\act^*)^k$ is \emph{recognisable}  if it is a finite union of products of regular languages, that is, $R=\bigcup \limits_{i =1 }^n L_{i,1} \times \cdots \times L_{i, k}$, where each $L_{i,j}$ is a regular language. An {\NFA}-representation of $R$ is $\{(\Aut_{i,1},\cdots,\Aut_{i,k})\}_{i\in[n]}$, where each $\Aut_{i,j}$ is an {\NFA} defining $L_{i,j}$, that is, $\Lang(\Aut_{i,j}) = L_{i,j}$.
\end{definition}

%Our approach to tackle this case is to simulate the behaviors of a single task by a {\PDS}, and use a set of {\NFA} tuples to represent the configurations of {\AMASS} $\Mm$.
%Theorem~\ref{thm:iff-recog} shows the reachability problem of $\STK$-dominating $\LMAMASS$ is decidable, and we prove Theorem~\ref{thm:iff-recog} by Lemma~\ref{lem:iff-forward} and Lemma~\ref{lem:iff-backward}.

Let $\Theta_\Mm = \left \{ \aname_1 \cdots \aname_k \in (\aft(\act))^+ \mid k \le |\act_{\singleinstance}| + |\aft(\act)| \right\}$. Intuitively, $\Theta_\Mm$ denotes the set of affinity sequences  in the configurations of $\Mm$. For each $\theta=\aname_1 \cdots \aname_k \in \Theta_\Mm$, we define 
%
$$
\begin{array}{l}
\RLang(\Mm, \theta) = \\
\ \ \left\{(S_1, \cdots, S_k) \in (\aft^+)^k\ \big{\vert}\  ([A_0], \aft(A_0)) \xRightarrow[\Mm]{} ((S_1, \aname_1), \cdots, (S_k, \aname_k)) \right\}.
\end{array}
$$ 
%
Since each task $S_i$ can be seen as a string from $\act^+$, $\RLang(\Mm, \theta)$ can be seen as a $k$-ary string relation over $\act^+$. 


\begin{lemma}\label{lem:iff-recog}
    For each $\theta \in \Theta_\Mm$, $\RLang(\Mm, \theta)$ is a recognizable relation and its $\NFA$-representation can be effectively computed.
\end{lemma} 
% We solve the reachability problem in this case by deciding whether there is a {\WOTrNFA}-representation $(\theta,(\Aut_1,\cdots,\Aut_k))$ of $\conf_{\theta}$ for $\theta\in\Theta_{\Mm}$, satisfying $\Aut_i\cap\Aut_i\neq\emptyset$ for each $i\in[k]$ with $\ConfSet(\Aut_i) = \ConfSet_i$.

According to Lemma~\ref{lem:iff-recog}, for each $\theta = \aname_1 \cdots \aname_k \in \Theta_\Mm$, an {\NFA}-representation of $\RLang(\Mm, \theta)$, say $(\Aut_{\theta, i,1},\cdots,\Aut_{\theta, i,k})_{i \in [n_\theta]}$, can be effectively computed. 
Then the configuration reachability problem of $\Mm$ is solved as follows: If there is $\theta =  \aname_1 \cdots \aname_k \in \Theta_\Mm$ such that $\Lang(\Aut_{\theta, i, j}) \cap \Lang(\Aut_j) \neq \emptyset$ for each $j \in [k]$, then report ``yes'', otherwise, report ``no''.

% The rest of this section is devoted to the proof of Theorem~\ref{thm:iff-recog}.
The rest of this section is devoted to the proof of Lemma~\ref{lem:iff-recog}. We shall distinguish whether $\lmd(A_0) = \STK$ or not. 
The former case is simpler because, by Proposition~\ref{prop-stk}, all tasks will be rooted at $\STK$ activities. For the latter, the more general case, the back stack may contain, apart from the tasks rooted at $\STK$ activities, one single task rooted at $A_0$. 
We shall consider the two cases in Section~\ref{sec:lmamass-stk} and Section~\ref{sec:lmamass-nostk} respectively.

%Let us introduce a notation. 


\subsection{The case $\lmd(A_0) = \STK$ or $\SIT$}\label{sec:lmamass-stk}

In this case, we prove Lemma~\ref{lem:iff-recog} by showing that the set of configurations that are reachable from the initial configuration can be \emph{saturated by repeatedly applying a finite set of rules}. 

Before presenting the rules, let us introduce several notations. 

Let $A \in \act_\singletask$. 
We define a {\PDS} $\Pp_{A} = (P_A, \Gamma_{A}, \Delta_{A})$, where 
\begin{itemize}
\item $P_A = \{p_0, p_{\pop}\}$, 
\item $\Gamma_{A} = \act_{\standard} \cup \act_{\singletop} \cup \{A\}$ if $\lmd(A) = \singletask$.
\item $\Delta_{A}$ is defined as follows: 
\begin{itemize}
%\item $(p_1, A, p_1) \in \Delta_A$. 
%
\item For each activity $A' \in \Gamma_A $, we have $(p_0, A', \epsilon, p_0) \in \Delta_{A}$.
%
\item For each transition rule $A' \xrightarrow{\startactivity(\bot)} A'' \in \Delta$ such that $\lmd(A') \neq \singleinstance$ and $\lmd(A'') \neq  \singleinstance$, 
\begin{itemize}
    \item if either $\lmd(A'') = \STD$ or $\lmd(A'') = \singletop$ and $A' \neq A''$, we have $(p_0, A', A''A', p_0) \in \Delta_{A}$,    
    % \item For each transition $A\xrightarrow{\startactivity(\bot)}B\in\Delta$, if $\lmd(B) = \STP$ and $A\neq B$, then $(p_0, A, BA, p_0)\in\Delta_{A'}$,
    \item if $\lmd(A) = \singletask$, $A' \neq A$ and $A'' = A$, then $(p_0, A', \epsilon, p_{\pop}) \in \Delta_{A}$, and for each $B \in \Gamma_{A} \setminus \{A\}$, $(p_{\pop}, B, \epsilon, p_{\pop}) \in \Delta_{A}$, moreover, $(p_{\pop}, A, A, p_0) \in \Delta_{A}$. (Intuitively, these transition rules simulate the action of popping the task until reaching an instance of $A$. )
\end{itemize}
\end{itemize}
Note that in the definition of $\Delta_{A}$, the transition rules $A' \xrightarrow{\startactivity(\bot)} A'' \in \Delta_A$ satisfying the following condition are ignored: $\lmd(A') = \singleinstance$, or $\lmd(A'') = \singleinstance$, or $\lmd(A'') = \singletask$ and $A'' \neq A$. 
\end{itemize}
Intuitively, $\Pp_{A}$ simulates the behavior of a task of $\Mm$ where $A$ is already in the bottom of the task. 

Moreover, for $A \in \act$ such that $\lmd(A) = \singleinstance$, we define $\Pp_A  = (P_A, \Gamma_A, \Delta_A)$, where $P_A = \{p_0\}$, $\Gamma_A = \{A\}$, and $\Delta_A = \{(p_0, A, \epsilon, p_0)\}$.



%By slightly abusing the notation, for a $\Pp_{A}$-{\NFA} $\Aut$ where $p_0$ is the only initial state, let us use $\Lang(\Aut)$ to denote $w \in \act^+$ such that $(p_0, w) \in \conf_{\Aut}$. 
In the sequel, we define the $\Pp_{A}$-{\NFA}s $\Aut_{A}$ and $\Aut_{A\rightsquigarrow B}$ for $B \in \Gamma_A$ such that $\Lang(\Aut_A) = \{A\}$ and $\Lang(\Aut_{A\rightsquigarrow B}) =B\act^* \cap \Lang((\Aut_{A})^{\post^*}_{\Pp_A})$. 
\begin{itemize}
    \item $\Aut_{A} = (P', \Gamma_A, \{(p_0, A, p_f)\},\{p_0\},\{p_f\} )$, where $P' = P_A \cup \{p_f\} \cup \{\langle p_0,A'\rangle \mid A'\in\Gamma_A\}$ with $p_f \not \in P_A$.  
    %
    \item For $B \in \Gamma_A$, $\Aut_{A\rightsquigarrow B}$ is obtained from $(\Aut_{A})^{\post^*}_{\Pp_A}$ by \emph{removing all non-$B$ transitions out of $p_0$}, then removing all transitions that cannot be reached from $p_0$. 
\end{itemize}
Note that it is possible that $\Lang(\Aut_{A\rightsquigarrow B}) = \emptyset$.
From the definition of $\Pp_A$, $\Aut_A$, and $\Aut_{A\rightsquigarrow B}$, we know that $\Lang((\Aut_A)^{\post^*}_{\Pp_A}) = \bigcup\limits_{B\in\Gamma_A} \Lang(\Aut_{A\rightsquigarrow B})$ and $\Lang(\Aut_{A\rightsquigarrow A}) = \Lang(\Aut_A)$. 
%Note that for each $S\in\ConfSet(\Aut_A^B)$, we have $\btmact(S) = A$.

%The main idea is to show that the set of configurations that are reachable from the initial configurations can be 

%
%We start with a simpler case, i.e., $\lmd(A_0)=\STK$. Intuitively, for a configuration $\rho = (S_1,\cdots,S_k)$, there is an {\NFA}-representation $(\Aut_1,\cdots,\Aut_k)$ in $\AutReach$ satisfying that $S_i\in\ConfSet(\Aut_i)$ for each $i\in[k]$. From the Proposition~\ref{prop-stk}, we know each task $S_i$ is rooted at an $\STK$-activity which sits on the bottom of $S_i$. Suppose $\topact(S_1) = A$. When a transition $A\xrightarrow{\startactivity(\bot)}B$ with $B\in\act_{\STK}$ is fired, according to the semantics of $\Mm$, the $B$-task of $\rho$, say $S_i$, is switched to the top of $\rho$ and changed into $[B]$. To simulate this,  we add the tuple $(\Aut_i',\Aut_1,\cdots,\Aut_{i-1},\Aut_{i+1},\cdots,\Aut_k)$ into $\AutReach$ with $\ConfSet(\Aut_i') = [B]$.

We are ready to present the procedure that computes effectively $\AutReach$, a finite set of pairs $(\aname_1 \cdots \aname_k, (\Aut_1, \cdots, \Aut_k))$ where each of $\Aut_i$'s are of the form $\Aut_{A\rightsquigarrow B}$ for $A \in \act_\singletask\cup\act_\singleinstance$ and $B \in \Gamma_A$, by repeatedly applying a finite set of saturation rules, such that
\[\RConfs(\Mm) = \bigcup \limits_{(\aname_1 \cdots \aname_k, (\Aut_1, \cdots, \Aut_k)) \in \AutReach} \confs((\aname_1 \cdots \aname_k, (\Aut_1, \cdots, \Aut_k))).\]

Initially, let $\AutReach := \{(\aft(A_0),(\Aut_{A_0}))\}$.
Then it adds pairs to $\AutReach$, according to the following saturation rules, until no more pairs can be added. 

\smallskip
\fbox
{
\begin{minipage}{0.9\textwidth}
\begin{enumerate}
    %
    \item If $A \xrightarrow{\startactivity(\bot)}A' \in\Delta$, $\lmd(A')=\STK$ or $\singleinstance$, $(\theta, (\Aut_1,\cdots,\Aut_k)) \in \AutReach$ such that $\Lang(\Aut_1) \neq \emptyset$ and $\Aut_1 = \Aut_{B\rightsquigarrow A}$ for some $B \in \act_\singletask\cup\act_\singleinstance$, moreover, $\namefun_{A'}(\theta) = \bot$,
    then let $\AutReach: = \AutReach \cup \{(\aft(A')\theta, (\Aut_{A'},\Aut_1,\cdots,\Aut_k))\}$.
    % where $\Aut_1'$ is obtained from $\Aut_1$ by 
    % removing all non-$A$ transitions out of $p_0$, then removing all transitions cannot be reached from $p_0$.
        \textbf{[launch an $A'$-task]}

    \item If $A \xrightarrow{\startactivity(\bot)}A' \in\Delta$, $\lmd(A')=\STK$ or $\singleinstance$, $(\theta, (\Aut_1,\cdots,\Aut_k)) \in \AutReach$ such that $\Lang(\Aut_1) \neq \emptyset$ and $\Aut_1 = \Aut_{B\rightsquigarrow A}$ for some $B \in \act_\singletask\cup\act_\singleinstance$, moreover, $\namefun_{A'}(\theta) = i$ for some $i \in [k]$ with $i > 1$, 
        then let $\AutReach:= \AutReach \cup \{(\theta', (\Aut_{A'}, \Aut_1, \cdots,\Aut_{i-1},\Aut_{i+1},\cdots,\Aut_{k}))\}$, where $\theta' = \aname_i\aname_1\dots\aname_{i-1}\aname_{i+1}\dots\aname_k$. 
        % and $\Aut_1'$ is obtained from $\Aut_1$ by removing all non-$A$ transitions out of $p_0$, then removing all transitions cannot be reached from $p_0$.
        \textbf{[escalate the $A'$-task to be the top and reset its content to $[A']$]}
        %
    \item If $(\aname_1 \cdots \aname_k, (\Aut_1,\cdots,\Aut_k)) \in \AutReach$ and $k>1$, then let $\AutReach := \AutReach \cup \{(\aname_2\dots\aname_k, (\Aut_2,\cdots,\Aut_k))\}$.
        \textbf{[pop all activities in the top task]}
%
    \item If $(\theta, (\Aut_1,\cdots,\Aut_k)) \in \AutReach$, $\Aut_1 = \Aut_{A\rightsquigarrow B}$ for some $A\in\act_{\STK}\cup\act_{\SIT}$ and $B \in \Gamma_A$, and $B'  \in \Gamma_A$ such that $\Lang(\Aut_{A\rightsquigarrow B'}) \neq \emptyset$, then let 
    $\AutReach := \AutReach \cup \{(\theta, (\Aut_{A\rightsquigarrow B'}, \Aut_2,\cdots,\Aut_k))\}$. 
    % then adds the tuple $(\theta, (\Aut_1^{\post^*},\Aut_2,\cdots,\Aut_k))$ to $\AutReach$.
%    such that $\Aut_1 = \Aut_A^B$ for some $A\in\act_{\STK},B\in\act$, then adds the tuple $(\theta, (\Aut_A^{B'},\Aut_2,\cdots,\Aut_k))$ to $\AutReach$ for each $B'\in\act$ with $\ConfSet(\Aut_A^{B'})\neq\emptyset$.
        \textbf{[simulate the behaviors of the top task]}
\end{enumerate}
\end{minipage}
}

\medskip

The aforementioned procedure terminates since $\Theta_\Mm$ is finite and the set of {\NFA}s $\Aut_{A\rightsquigarrow B}$ occurring in $\AutReach$ is finite.
Let $\AutReach_f$ denote the value of $\AutReach$ when the aforementioned procedure terminates. 




%We are going to present a procedure to compute the {\NFA}-representations of $\conf_\theta$ for $\theta \in \Theta_\Mm$. Specifically, the procedure computes a set of tuples $(\aname_1 \cdots \aname_k, (\Aut_1, \cdots, \Aut_k))$, denoted by $\AutReach$, such that for each $\theta =  \aname_1 \cdots \aname_k$, the {\NFA}-tuples $(\Aut_1, \cdots, \Aut_k)$ with $(\theta, (\Aut_1, \cdots, \Aut_k)) \in \AutReach$ constitute an {\NFA}-representation of $\conf_\theta$. 

% Moreover, all these {\NFA}s in $\AutReach$ are satisfied that the states are from $Q$, $p_1$ (resp. $p_f$) is the only initial (resp. final) state.

% To compute $\AutReach$, we need an additional notation $\topact_{A}(\Aut)$ for a $\Pp_{A}$-{\NFA} $\Aut$, an activity $A\in\act$ and we require that $\ConfSet(\Aut)\cap\ConfSet(A\act^*)\neq\emptyset$. Moreover we have $\ConfSet(\topact_{A}(\Aut)) = \ConfSet(\Aut)\cap\ConfSet(A\act^*)$, and $\topact_{A}(\Aut)$ could be obtained from $\Aut$ by removing all non-$A$ transitions out of $p_1$, then removing all transitions cannot be reached from $p_1$.
% which intuitively specifies how to obtain a new {\NFA} from $\Aut$ 
% when an $\STK$ activity $A$ is started. Moreover, we let $\Aut = (Q', \act, \delta, \{p_b\}, \{p_f\})$ for some $b\in\{0,1\}$, and we let all transitions out of $p_b$ are labeled with $B$, hence all strings accepted by $\Aut$ must (resp. \emph{not}) contain $A_0'$, if $b = 0$ (resp. $b=1$) and started with $B$.
% Then we define 


%Moreover, all these {\NFA}s in $\AutReach$ are in form $\Aut_A^B$ for some $A\in\act_{\STK}, B\in\act$. 

%For each $\theta=\aname_1\dots\aname_k\in\Theta_{\Mm}$ and an $\NFA$-tuple $(\Aut_1, \cdots, \Aut_k)$, we use $\Lang(\theta, (\Aut_1, \cdots, \Aut_k))$ to denote the set of $(S_1, \cdots, S_k) \in (\act^+)^k$ such that $((S_1, \aname_1), \cdots, (S_k, \aname_k)) \in \confs((\theta, (\Aut_1, \cdots, \Aut_k)))$.

We are going to prove Lemma~\ref{lem:iff-recog} for the case $\lmd(A_0) = \singletask$, that is, for each $\theta=\aname_1\dots\aname_k\in\Theta_{\Mm}$, 
%
\[\RLang(\Mm, \theta) = \bigcup \limits_{(\theta, (\Aut_1, \cdots, \Aut_k)) \in \AutReach_f} \Rel((\Aut_1, \cdots, \Aut_k)).\]

We prove the equation by showing the left-hand side is a subset of the right-hand side and vice versa. 

\paragraph*{The proof of $\RLang(\Mm, \theta) \subseteq \bigcup \limits_{(\theta, (\Aut_1, \cdots, \Aut_k)) \in \AutReach_f} \Rel((\Aut_1, \cdots, \Aut_k))$}

%\begin{lemma}\label{lem:iff-forward}
%    Let $\AutReach$ be the set computed by the aforementioned saturation procedure. Then for each $\theta=\aname_1\dots\aname_k\in\Theta_{\Mm}$, we have
%    $$\conf_{\theta}\subseteq\bigcup \limits_{(\theta, (\Aut_1, \dots, \Aut_k)) \in \AutReach} \ConfSet(\Aut_1) \times \dots \times \ConfSet(\Aut_k).$$
%\end{lemma}

%\begin{lemma}\label{lem:iff-forward}
%    Let $\AutReach$ be the set computed by the aforementioned saturation procedure. Then for each $\theta=\aname_1\dots\aname_k\in\Theta_{\Mm}$, we have
%    $$\conf_{\theta}\subseteq\bigcup \limits_{(\theta, (\Aut_1, \dots, \Aut_k)) \in \AutReach} \ConfSet(\Aut_1) \times \dots \times \ConfSet(\Aut_k).$$
%\end{lemma}

\begin{proof}
For $n \ge 1$, let $\xrightarrow[\Mm]{}^n$ denote the $n$-fold composition of $\xrightarrow[\Mm]{}$. Moreover, by convention, let $\xrightarrow[\Mm]{}^0$ be the identity relation on $\confs(\Mm)$. 
Then for each $\theta = \aname_1 \dots \aname_k \in \Theta_\Mm$, we prove by an induction on $n \ge 0$ the following claim.

\smallskip

\noindent {\bf Claim}. \emph{For each $n \ge 0$ and configuration $\rho =  ((S_1, \aname_1), \cdots, (S_k, \aname_k))$ such that 
%
$([A_0], \aft(A_0)) \xrightarrow[\Mm]{}^n \rho$,  let $\theta  = \aname_1 \cdots \aname_k$, then 
%
there is  $(\Aut_1, \dots, \Aut_k)$ such that $(\theta, (\Aut_1, \dots, \Aut_k)) \in \AutReach_f$ and $(S_1, \cdots, S_k) \in \Rel((\Aut_1,\cdots, \Aut_k))$}.   

%$\rho \in  \ConfSet(\Aut_1) \times \dots \times \ConfSet(\Aut_k)$.

\smallskip

\noindent \emph{Induction base $n = 0$}. From $([A_0], \aft(A_0)) \xrightarrow[\Mm]{}^0 \rho$, we have $\rho = ([A_0], \aft(A_0))$. From the computation of $\AutReach_f$, we know that $(\aft(A_0), (\Aut_{A_0})) \in \AutReach_f$. Since $[A_0] \in \Lang(\Aut_{A_0})$, the claim holds for $n = 0$. 

\smallskip

\noindent \emph{Induction step $n > 0$}. Suppose the claim holds for $n-1$. Assuming $([A_0], \aft(A_0)) \xrightarrow[\Mm]{}^n \rho =  ((S_1, \aname_1), \dots, (S_k, \aname_k))$, we are going to show that there is  $(\theta, (\Aut_1, \dots, \Aut_k)) \in \AutReach_f$ such that $(S_1, \cdots, S_k) \in \Rel((\Aut_1,\cdots, \Aut_k))$, that is, $S_i \in \Lang(\Aut_i)$ for each $i \in [k]$. 

From $([A_0], \aft(A_0)) \xrightarrow[\Mm]{}^n \rho$,  we know that there are a configuration $\rho' = ((S'_1, \aname'_1), \dots, (S'_l, \aname'_l))$ and a transition rule $\tau$ such that $([A_0]) \xrightarrow[\Mm]{}^{n-1} \rho' \xrightarrow[\Mm]{\tau} \rho$.

Let $\theta'= \aname'_1 \cdots \aname'_l$. Then from the induction hypothesis, there exists $(\theta', (\Aut'_1, \dots, \Aut'_l)) \in \AutReach_f$ such that $S'_j \in \Lang(\Aut'_j)$ for each $j \in [l]$. Moreover, let $A \in \act_\singletask\cup\act_\singleinstance$ and $B \in \Gamma_A$ such that $\Aut'_1 = \Aut_{A \rightsquigarrow B}$. 

If $\tau = \back$, then we distinguish between whether $|S'_1| > 1$ or not. 
\begin{itemize}
\item If $|S'_1| > 1$, then $\theta' = \theta$, $k=l$, $S'_1 = [B] \cdot S_1$, and $(S'_2, \cdots, S'_l) = (S_2, \cdots, S_k)$. 
Therefore, $S_1 \in \Lang(\Aut_{A \rightsquigarrow B'})$ for some $B'$. According to the 4th saturation rule, $(\theta, (\Aut_{A \rightsquigarrow B'}, \Aut'_1, \cdots, \Aut'_l)) \in \AutReach_f$. Since $(S_1, \cdots, S_k) \in \Rel((\Aut_{A \rightsquigarrow B'}, \Aut'_2, \cdots, \Aut'_l))$, we conclude that the claim holds for $n$ in this situation. 
%
\item If $|S'_1| = 1$, then $k = l - 1$, and $(S_1,\cdots,S_k) = (S_2',\cdots,S_l')$. According to the 3rd saturation rule, $(\theta, (\Aut_2',\cdots,\Aut_l')) \in \AutReach_f$. From $(S_1,\dots,S_k) = (S'_2, \cdots, S'_l) \in \Lang(\Aut_2') \times \cdots \times \Lang(\Aut_l') = \Rel((\Aut'_2, \cdots, \Aut'_l))$, we conclude that the claim holds for $n$ in this situation. 
\end{itemize}


Let us assume $\tau = B \xrightarrow{\startactivity(\bot)} C$ in the sequel. From the definition of $\singletask$-dominating $\AMASS$, there are the following three cases. 
\begin{itemize}
\item $\lmd(B) \neq \singleinstance$ and $\lmd(C) = \standard$ or $\singletop$, 
%
\item $\lmd(B) \neq \singleinstance$ and $\lmd(C) = \singletask$ or $\singleinstance$, 
%
\item $\lmd(B) = \singleinstance$ and $\lmd(C) = \singletask$ or $\singleinstance$. 
\end{itemize}

\smallskip

\noindent \emph{Case $\lmd(B) \neq \singleinstance$ and $\lmd(C) = \standard$ or $\singletop$}. In this case, $\theta = \theta'$. Moreover, $\rho = \rho'$ or $\rho$ is obtained from $\rho'$ by pushing $C$ to the top task.

If $\rho = \rho'$, then we are done. 

If $\rho$ is obtained from $\rho'$ by pushing $C$ to the top task, then $S_1 \in \Lang(\Aut_{A \rightsquigarrow C})$. Moreover, from the 4th saturation rule, we know that $(\theta, (\Aut_{A \rightsquigarrow C}, \Aut'_2, \cdots, \Aut'_l)) \in \AutReach_f$. 
Since 
$$(S_1, \cdots, S_k) \in \Lang(\Aut_{A \rightsquigarrow C}) \times \Lang(\Aut'_2) \times \cdots \times \Lang(\Aut'_l) = \Rel(\Aut_{A \rightsquigarrow C}, \Aut'_2, \cdots, \Aut'_l),$$ 
we conclude that the claim holds for $n$ in this case. 

\smallskip

\noindent \emph{Case $\lmd(B) \neq \singleinstance$ and $\lmd(C) = \singletask$ or $\singleinstance$}. 

If $C = A$, then $S_1 = [A]$. From the 4th saturation rule, $(\Aut_{A \rightsquigarrow A} = \Aut_A, \Aut'_2, \cdots, \Aut'_l) \in \AutReach_f$. Since $(S_1, \cdots, S_k) \in \Rel((\Aut_A, \Aut'_2, \cdots, \Aut'_l))$, we conclude that the claim holds for $n$ in this case. 

Let us assume $C \neq A$ in the sequel.  Then $\theta' \neq \theta$.  We distinguish between the following two subcases. 
\begin{itemize}
    \item \emph{Subcase $\namefun_{C}(\theta') = \bot$}. Then $k=l+1$, $S_1' \in \Lang(\Aut_{A\rightsquigarrow B})$, $S_1=[C]$, and $(S_2,\cdots,S_k)=(S_1',\cdots,S_l')$.  According to the 1st saturation rule, $(\theta, (\Aut_{C}, \Aut_1', \cdots, \Aut_l')) \in \AutReach_f$.
    %, where $\Lang(\Aut_1') \cap A\act^* \neq \emptyset$, and $S_1=[A']\in\ConfSet(\Aut_{A'})$, 
   From $(S_1, \cdots, S_k) = ([C], S_1', \cdots, S_l') \in \Lang(\Aut_{C}) \times \Lang(\Aut_1') \times \cdots \times \Lang(\Aut_l')$, we conclude that the claim holds for $n$ in this subcase.
    %
    \item \emph{Subcase $\namefun_{C}(\theta') = i \neq \bot$, and $i > 1$}: Then $k = l$, 
    $S'_1  \in \Lang(\Aut_{A\rightsquigarrow B})$, $S_1 = [C]$, $(S_2, \dots, S_k) = (S'_1, \dots, S'_{i-1}, S'_{i+1}, \dots, S'_l)$. According to the 2nd saturation rule, we know that $(\theta, (\Aut_{C}, \Aut_1',\cdots, \Aut_{i-1}', \Aut_{i+1}', \cdots, \Aut_{l}')) \in \AutReach_f$. 
%    where $\ConfSet(\Aut_1')\cap A\act^*\neq\emptyset$, and  $S_1=[A']\in\ConfSet(\Aut_{A'})$, 
    From 
    $$
    \begin{array}{l}
    	(S_1,\cdots,S_k) = ([C], S_1', \cdots, S_{i-1}', S_{i+1}', \cdots, S_l') \in \\
    	\ \ \Lang(\Aut_{C}) \times \Lang(\Aut_1') \times \cdots \times \Lang(\Aut_{i-1}') \times \Lang(\Aut_{i+1}')\times \cdots \times \Lang(\Aut_{l}'),
    \end{array}
    $$  
    we conclude that the claim holds for $n$ in this subcase. 
%
\end{itemize}

\smallskip

\noindent \emph{Case $\lmd(B) = \singleinstance$ and $\lmd(C) = \singletask$ or $\singleinstance$}. 

The discussion is similar to the previous case. 


%%%%%%%%%%%%%%%%%%% original proof removed
%%%%%%%%%%%%%%%%%%% original proof removed
\hide{
%Moreover, $\rho$ is obtained from $\rho'$ by a transition of $\Mm$. 
We distinguish between the cases $\theta' = \theta$ and $\theta' \neq \theta$.

\smallskip

\noindent \emph{Case $\theta' = \theta$}.
In this case, the configuration $\rho$ is obtained from $\rho'$ by only updating the content of the top task. 


% Moreover if $\lmd(A) = \singleinstance$ then we have $\rho = \rho'$, we know that the claim holds for $n$. Therefore we consider $\lmd(A)$
Therefore $(p_0,S_1')\xRightarrow{\Pp_{A}}(p_0,S_1)$, $l=k$, and $(S_2',\cdots,S_l')=(S_2,\cdots,S_k)$. 
%
%Let $\Aut'_1 = \Aut_{A, B}$ for some $A \in \act_\singletask$ and $B \in \act \setminus \act_\singleinstance$. 
From  $S'_1 \in \Lang(\Aut'_1)$, we have
%where we let $\Aut_1' = \Aut_A^B$ is a $\Pp_A$-{\NFA} for some $A \in \act_{\STK}, B\in\act$, 
$S_1 \in \Lang((\Aut_1')^{\post^*}_{\Pp_A})$. Moreover, from $\Lang(\Aut_1') = \Lang(\Aut_{A\rightsquigarrow B}) \subseteq \Lang((\Aut_A)^{\post^*}_{\Pp_A})$, we deduce $\Lang((\Aut_1')^{\post^*}_{\Pp_A}) \subseteq \Lang((\Aut_A)^{\post^*}_{\Pp_A})$. Therefore, $S_1 \in \Lang((\Aut_A)^{\post^*}_{\Pp_A})$. 

From $ \Lang((\Aut_A)^{\post^*}_{\Pp_A}) = \bigcup \limits_{B' \in \Gamma_A } \Lang(\Aut_{A\rightsquigarrow B'})$, we know that $S_1 \in \Lang(\Aut_{A\rightsquigarrow B'})$ for some $B'$. 
Furthermore, according to the aforementioned 4th saturation rule, we know that $(\theta, (\Aut_{A\rightsquigarrow B'}, \Aut_2', \cdots, \Aut'_l)) \in \AutReach_f$.
From $\rho = (S_1, \cdots, S_k) \in \Lang(\Aut_{A\rightsquigarrow B'}) \times \Lang(\Aut_2') \times \cdots \times \Lang(\Aut'_l)$, we know that the claim holds for $n$. 

\paragraph{Case $\theta' \neq \theta$}
In this case, the top task of $\rho$ is different from that of $\rho'$.  There are the following three subcases. 
\begin{itemize}
    \item \emph{Subcase $\tau = B \xrightarrow{\startactivity(\bot)} B'$, $\lmd(B') = \STK$ or $\SIT$, and $\namefun_{B'}(\theta') = \bot$}. Then $k=l+1$, $S_1' \in \Lang(\Aut_{A\rightsquigarrow B})$, $S_1=[B']$, and $(S_2,\cdots,S_k)=(S_1',\cdots,S_l')$.  According to the 1st saturation rule, $(\theta, (\Aut_{B'}, \Aut_1', \cdots, \Aut_l')) \in \AutReach_f$.
    %, where $\Lang(\Aut_1') \cap A\act^* \neq \emptyset$, and $S_1=[A']\in\ConfSet(\Aut_{A'})$, 
   From $(S_1, \cdots, S_k) = ([B'], S_1', \cdots, S_l') \in \Lang(\Aut_{B'}) \times \Lang(\Aut_1') \times \cdots \times \Lang(\Aut_l')$, we conclude that the claim holds for $n$ in this subcase.
    %
    \item \emph{Subcase $\tau = B \xrightarrow{\startactivity(\bot)} B'$, $\lmd(B')=\STK$ or $\SIT$, $\namefun_{B'}(\theta') = i \neq \bot$, and $i > 1$}: Then $k = l$, 
    $S'_1  \in \Lang(\Aut_{A\rightsquigarrow B})$, $S_1 = [B']$, $(S_2, \dots, S_k) = (S'_1, \dots, S'_{i-1}, S'_{i+1}, \dots, S'_l)$. According to the 2nd saturation rule, we know that $(\theta, (\Aut_{B'}, \Aut_1',\cdots, \Aut_{i-1}', \Aut_{i+1}', \cdots, \Aut_{l}')) \in \AutReach_f$. 
%    where $\ConfSet(\Aut_1')\cap A\act^*\neq\emptyset$, and  $S_1=[A']\in\ConfSet(\Aut_{A'})$, 
    From 
    $$
    \begin{array}{l}
    	(S_1,\cdots,S_k) = ([B'], S_1', \cdots, S_{i-1}', S_{i+1}', \cdots, S_l') \in \\
    	\ \ \Lang(\Aut_{B'}) \times \Lang(\Aut_1') \times \cdots \times \Lang(\Aut_{i-1}') \times \Lang(\Aut_{i+1}')\times \cdots \times \Lang(\Aut_{l}'),
    \end{array}
    $$  
    we conclude that the claim holds for $n$ in this subcase. 
%
    \item \emph{Subcase $\tau = \back$ and $|S_1'|=1$}. Then $k = l - 1$, and $(S_1,\cdots,S_k) = (S_2',\cdots,S_l')$.  According to the 3rd saturation rule, $(\theta, (\Aut_2',\cdots,\Aut_l')) \in \AutReach_f$. From $(S_1,\dots,S_k) = (S'_2, \cdots, S'_l) \in \Lang(\Aut_2') \times \cdots \times \Lang(\Aut_l')$, we conclude that the claim holds for $n$ in this subcase. 
\end{itemize}
}
%%%%%%%%%%%%%%%%%%% original proof removed
%%%%%%%%%%%%%%%%%%% original proof removed
\end{proof}


\paragraph*{The proof of $\bigcup \limits_{(\theta, (\Aut_1, \cdots, \Aut_k)) \in \AutReach_f}  \Rel((\Aut_1, \cdots, \Aut_k)) \subseteq \RRel(\Mm, \theta)$}

%\begin{lemma}\label{lem:iff-backward}
%    Let $\AutReach$ be the set computed by the aforementioned saturation procedure. Then for each $\theta = \aname_1 \dots \aname_k \in \Theta_\Mm$, we have 
%    $$\bigcup \limits_{(\theta, (\Aut_1, \dots, \Aut_k)) \in \AutReach} \ConfSet(\Aut_1) \times \dots \times \ConfSet(\Aut_k) \subseteq \conf_\theta.$$
%\end{lemma}

\begin{proof}
Since $\AutReach_f$ is computed by applying the saturation rules and adding the tuples into $\AutReach$,  let us use $\AutReach_0$ to denote $\{(\aft(A_0), \Aut_{A_0})\}$, use $\AutReach_1$ to denote the set obtained by adding a tuple to $\AutReach_0$, and for $n \ge 2$, use $\AutReach_n$ to denote the set obtained by adding a new tuple to $\AutReach_{n-1}$. 
We prove by an induction on $n \ge 0$ the following claim.  

\smallskip
\noindent {\bf Claim}. For each $n \ge 0$, if $(\theta, (\Aut_1, \cdots, \Aut_k)) \in \AutReach_n$ with $\theta = \aname_1 \cdots \aname_k$ and $(S_1,\dots,S_k) \in \Rel((\Aut_1, \cdots, \Aut_k))$, then $(([A_0], \aft(A_0))) \xRightarrow[\Mm]{} ((S_1, \aname_1), \cdots, (S_k, \aname_k))$.

\smallskip

%    Let $\AutReach_n$ be the set computed by the aforementioned saturation procedure after the $n$-th tuple $(\theta,(\Aut_1,\dots,\Aut_k))$ is added, 


\noindent \emph{Induction base $n = 0$}. 
Because $\AutReach_0 = \{(\aft(A_0),(\Aut_{A_0}))\}$, if $(S_1, \cdots, S_k) \in \Rel((\Aut_{A_0}))$, then $k=1$ and $S_1 = [A_0]$. Since $(([A_0], \aft(A_0))) \xRightarrow[\Mm]{} (([A_0], \aft(A_0)))$, it follows that the claim holds for $n = 0$. 

\smallskip

\noindent \emph{Induction step $n > 0$}. 
Suppose that $(\theta, (\Aut_1, \cdots, \Aut_k)) \in \AutReach_n$ with $\theta = \aname_1 \cdots \aname_k$ and $(S_1,\dots,S_k) \in \Rel((\Aut_1, \cdots, \Aut_k))$. 

If $(\theta, (\Aut_1, \cdots, \Aut_k)) \in \AutReach_{n-1}$, then the claim follows directly from the induction hypothesis. 

Next, let us assume that $(\theta, (\Aut_1, \cdots, \Aut_k)) \not \in  \AutReach_{n-1}$.
Therefore, $\AutReach_{n} = \AutReach_{n-1} \cup \{(\theta,(\Aut_1,\dots,\Aut_k))\}$.  
%
%Then the $n$-th tuple $(\theta,(\Aut_1,\dots,\Aut_k))$ is added which obtained from $(\theta',(\Aut_1',\dots,\Aut_l'))\in \AutReach_{n-1}$.
We distinguish which saturation rule is used to add $(\theta,(\Aut_1,\dots,\Aut_k))$.


\paragraph*{The 1st saturation rule} Then  there are a transition rule $\tau = A \xrightarrow{\startactivity(\bot)} A'  \in \Delta$ and $(\theta', (\Aut'_1, \cdots, \Aut'_l)) \in \AutReach_{n-1}$ such that $\lmd(A')=\STK$ or $\singleinstance$, $\Lang(\Aut'_1) \neq \emptyset$, $\Aut'_1 = \Aut_{B\rightsquigarrow A}$ for some $B \in \act_\singletask\cup\act_\singleinstance$, $\namefun_{A'}(\theta') = \bot$, $\theta = \aft(A') \theta'$, and $(\Aut_1, \cdots, \Aut_k) = (\Aut_{A'}, \Aut'_1, \cdots, \Aut'_l)$. 
Evidently, $\aname_1 = \aft(A')$ and $\theta' = \aname_2 \cdots \aname_k$. 

Suppose $(S_1,\dots,S_k) \in \Rel((\Aut_1, \cdots, \Aut_k))$. Then $S_1 = [A']$ and $(S_2, \cdots, S_k) \in \Rel((\Aut'_1, \cdots, \Aut'_l))$. 

From $(\theta', (\Aut'_1, \cdots, \Aut'_l)) \in \AutReach_{n-1}$ and the induction hypothesis, we know that $(([A_0], \aft(A_0))) \xRightarrow[\Mm]{} ((S_2, \aname_2), \cdots, (S_k, \aname_k))$. 

From $\Aut'_1 = \Aut_{B\rightsquigarrow A}$, $S_2 \in \Lang(\Aut'_1)$, $A \xrightarrow{\startactivity(\bot)}A'  \in \Delta$, $\namefun_{A'}(\theta') = \bot$, we know that 
$((S_2, \aname_2), \cdots, (S_k, \aname_k)) \xrightarrow[\Mm]{} (([A'], \aft(A')), (S_2, \aname_2), \cdots, (S_k, \aname_k))$.
Therefore, 
$(([A_0], \aft(A_0))) \xRightarrow[\Mm]{} (([A'], \aft(A')), (S_2, \aname_2), \cdots, (S_k, \aname_k))$. 
The claim holds for $n$ in this case. 

%$A \xrightarrow{\startactivity(\bot)}A'  \in \Delta$, $\lmd(A')=\STK$, $\namefun_{A'}(\theta') = \bot$ $(\theta,(\Aut_1,\dots,\Aut_k))$ is obtained from $(\theta',(\Aut_1',\dots,\Aut_l'))$ by the first saturation rule [$A\xrightarrow{\startactivity(\bot)}A' \in\Delta$ with $\lmd(A')=\STK$, $\namefun_{A'}(\theta') = \bot$]} :

\paragraph*{The 2nd saturation rule} Then there are a transition rule $\tau = A \xrightarrow{\startactivity(\bot)} A'  \in \Delta$ and $(\theta', (\Aut'_1, \cdots, \Aut'_l)) \in \AutReach_{n-1}$ such that $\lmd(A')=\STK$ or $\singleinstance$, $\Lang(\Aut'_1) \neq \emptyset$, $\Aut'_1 = \Aut_{B\rightsquigarrow A}$ for some $B \in \act_\singletask \cup \act_\singleinstance$, $\namefun_{A'}(\theta') = i > 1$, 
$\theta' = \aname_2  \cdots  \aname_{i} \aname_1  \aname_{i+1}  \cdots  \aname_k$, and 
$(\Aut_1, \cdots, \Aut_k) = (\Aut_{A'}, \Aut'_1, \cdots, \Aut'_{i-1}, \Aut'_{i+1}, \cdots, \Aut'_l)$. Evidently, $k = l$.

From 
%
$$(S_1, \cdots, S_k) \in \Rel((\Aut_1, \cdots, \Aut_k)) = \Rel((\Aut_{A'}, \Aut'_1, \cdots, \Aut'_{i-1}, \Aut'_{i+1}, \cdots, \Aut'_k)),$$ 
%
we know that $S_1 = [A']$, $(S_2, \cdots, S_i) \in \Rel((\Aut'_1, \cdots, \Aut'_{i-1}))$, and $(S_{i+1}, \cdots, S_k) \in \Rel((\Aut'_{i+1}, \cdots, \Aut'_k))$. 

Let $S' \in \Lang(\Aut'_i)$. Then $(S_2, \cdots, S_i, S', S_{i+1}, \cdots, S_k) \in \Rel(\Aut'_1, \cdots, \Aut'_k)$.
From $(\theta', (\Aut'_1, \cdots, \Aut'_k)) \in \AutReach_{n-1}$ and the induction hypothesis, we know that  
%
$$(([A_0], \aft(A_0))) \xRightarrow[\Mm]{} ((S_2, \aname_2), \cdots, (S_i, \aname_i), (S', \aname_1), (S_{i+1}, \aname_{i+1}), \cdots, (S_k, \aname_k)).$$ 

From $S_2 \in \Lang(\Aut'_1) = \Lang(\Aut_{B\rightsquigarrow A})$,  $\tau = A \xrightarrow{\startactivity(\bot)} A'$, $\namefun_{A'}(\theta') = i > 1$, 
$\theta' = \aname_2  \cdots  \aname_{i} \aname_1  \aname_{i+1}  \cdots  \aname_k$, we deduce that 
$$
\begin{array}{l}
((S_2, \aname_2), \cdots, (S_i, \aname_i), (S', \aname_1), (S_{i+1}, \aname_{i+1}), \cdots, (S_k, \aname_k)) \xrightarrow[\Mm]{} \\
(([A'], \aname_1), (S_2, \aname_2), \cdots, (S_i, \aname_i), (S_{i+1}, \aname_{i+1}), \cdots, (S_k, \aname_k)).
\end{array}
$$ 

Since $S_1 = [A']$, we have
$(([A_0], \aft(A_0))) \xRightarrow[\Mm]{} ((S_1, \aname_1), \cdots, (S_k, \aname_k)).$
%
Therefore, in this case, the claim holds for $n$.


\paragraph*{The 3rd saturation rule} Then $\tau = \back$, $(\theta', (\Aut'_1, \cdots, \Aut'_l)) \in \AutReach_{n-1}$,  $\theta' = \aname' \theta$ for some $\aname'$, $\Aut'_1 = \Aut_{A'}$ for some $A' \in \act_\singletask\cup \act_\singleinstance$, and $(\Aut_1, \cdots, \Aut_k) = (\Aut'_2, \cdots, \Aut'_l)$.

From $(S_1, \cdots, S_k) \in \Rel((\Aut_1, \cdots, \Aut_k)) = \Rel((\Aut'_2, \cdots, \Aut'_l))$, we know that $([A'], S_1, \cdots, S_k) \in \Rel((\Aut'_1, \Aut'_2, \cdots, \Aut'_l))$. 
Then according to $(\theta', (\Aut'_1, \cdots, \Aut'_l)) \in \AutReach_{n-1}$ and the induction hypothesis, we know that 
$$(([A_0], \aft(A_0))) \xRightarrow[\Mm]{} (([A'], \aname'), (S_1, \aname_1), \cdots, (S_k, \aname_k)).$$ 

Moreover, $(([A'], \aname'), (S_1, \aname_1), \cdots, (S_k, \aname_k)) \xrightarrow[\Mm]{} ((S_1, \aname_1), \cdots, (S_k, \aname_k))$. Therefore, we deduce 
$$(([A_0], \aft(A_0))) \xRightarrow[\Mm]{} ((S_1, \aname_1), \cdots, (S_k, \aname_k)).$$ 
We conclude that the claim holds for $n$ in this case. 

\paragraph*{The 4th saturation rule} Then $\Aut_1 = \Aut_{A\rightsquigarrow B}$ for some $A \in \act_\singletask \cup \act_\singleinstance$ and $B \in \Gamma_A$,  and $(\theta, (\Aut_{A\rightsquigarrow B'}, \Aut_2, \cdots, \Aut_k)) \in \AutReach_{n-1}$ for  some $B' \in \Gamma_A$. 

Therefore, $(S'_1, S_2, \cdots, S_k) \in \Rel((\Aut_{A\rightsquigarrow B'}, \Aut_2, \cdots, \Aut_k))$ for some $S'_1 \in \Lang(\Aut_{A\rightsquigarrow B'})$. 

Then according to the induction hypothesis, 
$$(([A_0], \aft(A_0))) \xRightarrow[\Mm]{} ((S'_1, \aname_1), (S_2, \aname_2), \cdots, (S_k, \aname_k)).$$

From $S'_1 \in \Lang(\Aut_{A\rightsquigarrow B'})$, $S_1 \in \Lang(\Aut_{A\rightsquigarrow B})$, and the definition of $\Pp_A$, we know that $(p_0, S'_1) \xRightarrow{\Pp_A} (p_0, [A]) \xRightarrow{\Pp_A} (p_0, S_1)$. 

Therefore, 
$$((S'_1, \aname_1), (S_2, \aname_2), \cdots, (S_k, \aname_k)) \xRightarrow[\Mm]{} ((S_1, \aname_1), (S_2, \aname_2), \cdots, (S_k, \aname_k)).$$
It follows that 
$$(([A_0], \aft(A_0))) \xRightarrow[\Mm]{} ((S_1, \aname_1), (S_2, \aname_2), \cdots, (S_k, \aname_k)).$$
We conclude that the claim holds for $n$ in this case. 
%Since $[A] \in \Lang((\Aut_{A, B'})^{\post^*}_{\Pp_A})$  and $S_1 \in \Lang(\Aut_{A, B}) \subseteq \Lang((\Aut_A)^{\post^*}_{\Pp_A})$, 
\end{proof}



\subsection{The case $\lmd(A_0) \neq \STK$}\label{sec:lmamass-nostk}
%
We then turn to the more general case $\lmd(A_0)\neq\STK$ which is more involved. In this case we prove Lemma~\ref{lem:iff-recog} by also showing the set of configurations that are reachable from the initial configuration can be saturated by repeatedly applying a finite set of rules.

Without loss of generality, we assume that there is $A'_0 \in \act$ such that $\lmd(A_0')=\STK$ and $\aft(A_0') = \aft(A_0)$. Note that such an $A'_0$ is unique, if it exists, since the task affinities of $\singletask$ activities are mutually distinct, according to the definition of $\singletask$-dominating $\AMASS$. The situation that there does not exist $A'_0 \in \act$ such that $\lmd(A_0')=\STK$ and $\aft(A_0') = \aft(A_0)$ can be reasoned about in a simpler and similar way. 

%We assume that $\lmd(A_0')=\STK$ and $\aft(A_0') = \aft(A_0)$. Then an $A_0'$-task can only surface when the original $A_0$-task is popped empty. 
%If this happens, no $A_0$-task will be recreated again, and thus, it is the same case with $\lmd(A_0) = \STK$,
%the challenging case is that we have both $A_0$-task and non-$A_0'$-tasks. 

Because $\lmd(A_0) \neq \singletask$, the $A_0$-task is different from all the other tasks where the bottom activities are $\singletask$-activities. The difference is manifested  when a transition $A \xrightarrow{\startactivity(\bot)} A'_0$ is applied. In this case, although the $A_0$-task will be moved to the top (if it is not the top task), its content will not be reset to $A_0$, instead, all the activities above $A'_0$ (if there is any) will be popped, but everything below $A'_0$ (including $A'_0$ itself) will be preserved.  

%The difficulty is the bottom activity of $A_0$-task is not $A_0'$, since when the transition $A\xrightarrow{\startactivity(\bot)}A_0'$ is fired, the content of $A_0$-task is not $[A_0']$. Hence the behaviors of $A_0$-task is different from the non-$A_0$-tasks.  

Before presenting the rules, let us introduce some notations. At first, for each $A \in \act_\singletask \cup \act_\singleinstance$, we can still define a {\PDS} $\Pp_A$ as in Section~\ref{sec:lmamass-stk}, to simulate the behavior of an $A$-task. 
%Similar to the $\Pp_{A}$ defined in Section~\ref{sec:lmamass-stk},
Moreover, we define a {\PDS} $\Pp_{A_0}=(P_{A_0}, \Gamma_{A_0},\Delta_{A_0})$, to simulate the behavior of an $A_0$-task,  where
%$\Mm$ where $A_0$ is already in the bottom of the task, where
\begin{itemize}
    \item $P_{A_0} = \{p_0,p_1,p_{\pop}\}$,
    \item $\Gamma_{A_0} = \act_\standard \cup \act_\singletop \cup \{A_0'\}$,
    \item $\Delta_{A_0}$ is defined as follows:
    \begin{itemize}
        \item for each $A\in\Gamma_{A_0}$, 
            \begin{itemize}
                \item if $A=A_0'$, we have $(p_1,A,\epsilon,p_0)\in\Delta_{A_0}$,
                \item otherwise, for each $b\in\{0,1\}$, we have $(p_b,A,\epsilon,p_b)\in\Delta_{A_0}$, 
            \end{itemize}
        \item for each transition rule $A\xrightarrow{\startactivity(\bot)}A' \in \Delta$ such that $A, A' \in \Gamma_{A_0}$,
        \begin{itemize}
            \item if either $\lmd(A') = \STD$ or $\lmd(A')=\STP$ and $A\neq A'$, then for each $b\in\{0,1\}$ we have $(p_b,A,A'A,p_b)\in\Delta_{A_0}$,
            \item if $A\neq A_0'$ and $A'=A_0'$, we have 
            \begin{itemize}
                \item $(p_0, A, A_0'A, p_1) \in \Delta_{A_0}$, (Intuitively, this transition rule simulates the action of pushing $A_0'$ in the task.)
                \item $(p_1, A, \epsilon, p_{\pop}) \in \Delta_{A_0}$, and for each $B\in\Gamma_{A_0}\setminus\{A_0'\}$, we have $(p_{\pop},B,\epsilon,p_{\pop})\in\Delta_{A_0}$, moreover, $(p_{\pop},A_0',A_0',p_1)\in\Delta_{A_0}$. (Intuitively, these transition rules simulate the action of popping the activities until reaching $A_0'$.)
            \end{itemize}
        \end{itemize}
    \end{itemize}
\end{itemize}

In the sequel, we define the following $\Pp_{A_0}$-{\NFA}s.   
\begin{itemize}
    \item $\Aut_{A_0} = (\{p_0, p_1, p_\pop, p_f\}, \Gamma_{A_0}, \{(p_0,A_0,p_f)\}, \{p_0\}, \{p_f\})$. (It is easy to see $\Lang(\Aut_{A_0}) = \{A_0\}$. )
    %where $P_{A_0}' = P_{A_0'}\cup\{p_f\}\cup\{\langle p_b,A\rangle\mid b\in\{0,1\}, A\in\Gamma_{A_0}\}$.
    \item For each $B \in \Gamma_{A_0} \setminus \{A_0'\}$, $\Aut_{A_0\rightsquigarrow B}$ is obtained from $(\Aut_{A_0})^{\post^*}_{\Pp_{A_0}}(p_0)$ by \emph{removing all non-$B$ transitions out of $p_0$}, then removing all transitions that are unreachable from $p_0$. (From the construction, $\Lang(\Aut_{A_0\rightsquigarrow B}) = B\act^* \cap \Lang((\Aut_{A_0})^{\post^*}_{\Pp_{A_0}}(p_0))$.)
    \item For each $B\in\Gamma_{A_0}$ and $C\in\Gamma_{A_0}\setminus\{A_0'\}$, 
    \begin{itemize}
        \item if $B = A_0'$, then $\Aut_{A_0\stackrel{C}\rightsquigarrow A_0'}$ is obtained from $\Aut_{A_0 \rightsquigarrow C}$ by first removing all the transitions out of $p_1$, then adding for each transition $(p_0, C, p)$ the two transitions $(p_1, A'_0, \langle p_1, A'_0 \rangle)$ and $(\langle p_1, A'_0 \rangle, C, p)$, and changing the set of initial states to $\{p_1\}$,  (From the construction, $\Lang(\Aut_{A_0\stackrel{C}\rightsquigarrow A'_0})  = \{A'_0 w \mid w \in  \Lang(\Aut_{A_0 \rightsquigarrow C}) \}$.)
        
%         all non-$A'_0$ transitions out of $p_1$, then for each transition $(p_1, A'_0, p')$, removing all non-$C$ transitions out of $p'$, and finally removing all transitions that are unreachable from $p_1$, 
%                $\Aut_{A_0\rightsquigarrow C}$ by adding the transitions $(p_1,A_0',\langle p_1,A_0'\rangle)$ and $(\langle p_1, A_0'\rangle, C, p)$ if $p_0\xRightarrow[\Aut_{A_0\rightsquigarrow C}]{C}p$, then removing all transitions out of $p_0$ and removing all transitions cannot be reached from $p_0$.
%
        \item if $B \neq A_0'$, then $\Aut_{A_0\stackrel{C}\rightsquigarrow B}$ is obtained from $(\Aut_{A_0\stackrel{C}\rightsquigarrow A_0'})^{\post^*}_{\Pp_{A_0}}(p_1)$ by first removing all  non-$B$ transitions out of $p_1$, then removing all non-$C$ transitions out of $\langle p_1, A'_0\rangle$, and finally removing all transitions that are unreachable from $p_1$. (From the construction, $\Lang(\Aut_{A_0\stackrel{C}\rightsquigarrow B})  = B\act^*A_0'C\act^*\cap\Lang((\Aut_{A_0 \stackrel{C}\rightsquigarrow A_0'})^{\post^*}_{\Pp_{A_0}}(p_1)) \}$.)
    \end{itemize}
\end{itemize}
Note that the states in these $\Pp_{A_0}$-{\NFA}s are from the set $\{p_0, p_1, p_\pop, p_f\} \cup \{p_0, p_1, p_\pop\} \times \Gamma_{A_0}$.
Moreover, from the aforementioned construction of these $\Pp_{A_0}$-{\NFA}s and the fact that each time when $A'_0$ is pushed in a transition of $\Pp_{A_0}$, the transition is of the form $(p_0, A, A_0'A, p_1)$, we know that the states $\langle p_0, A'_0\rangle$ and $\langle p_\pop, A'_0\rangle$ are never used in the transitions of these $\Pp_{A_0}$-{\NFA}s, thus they can be ignored, in other words, only the state $\langle p_1, A'_0\rangle$ is actually used in the transitions of these $\Pp_{A_0}$-{\NFA}s. 


%%%%%%%%%%%%%%%%%%%%%%%% redundant and removed 
%%%%%%%%%%%%%%%%%%%%%%%% redundant and removed 
\hide{
In the sequel, we define the following $\Pp_{A_0}$-{\NFA}s,  
\begin{itemize}
    \item $\Aut_{A_0}$ such that $\Lang(\Aut_{A_0}) = \{A_0\}$, 
    \item $\Aut_{A_0\rightsquigarrow B}$ for $B \in \Gamma_{A_0} \setminus \{A_0'\}$ such that $\Lang(\Aut_{A_0\rightsquigarrow B}) = B\act^* \cap \Lang((\Aut_{A_0})^{\post^*}_{\Pp_{A_0}}(p_0))$,
    \item $\Aut_{A_0\stackrel{C}\rightsquigarrow B}$ for $B\in\Gamma_{A_0}$ and $C\in\Gamma_{A_0}\setminus\{A_0'\}$  such that
    $$
    \Lang(\Aut_{A_0\stackrel{C}\rightsquigarrow B})  = 
    \left\{ 
    \begin{array}{lc}
        \{A'_0 w \mid w \in  \Lang(\Aut_{A_0 \rightsquigarrow C}) \} & \mbox{ if } B=A_0', \\
        B\act^*A_0'C\act^*\cap\Lang((\Aut_{A_0 \stackrel{C}\rightsquigarrow A_0'})^{\post^*}_{\Pp_{A_0}}(p_1))& \mbox{ otherwise}.
    \end{array}
\right.$$
    % \ \ $\{[A_0']\cdot S\mid S\in\Lang(\Aut_{A_0\rightsquigarrow C})\}$, if $B = A_0'$,\\
    % \ \ $B\act^*A_0'C\act^*\cap\Lang((\Aut_{A_0 \stackrel{C}\rightsquigarrow A_0'})^{\post^*}_{\Pp'_{A_0}}(p_1))$, if $B\neq A_0'$.
\end{itemize}
}
%%%%%%%%%%%%%%%%%%%%%%%% redundant and removed 
%%%%%%%%%%%%%%%%%%%%%%%% redundant and removed 


Note that it is possible that $\Lang(\Aut_{A_0\rightsquigarrow B}) =\emptyset$ or $\Lang(\Aut_{A_0\stackrel{C}\rightsquigarrow B})=\emptyset$. From the construction, we have the following observations. 
\begin{itemize}
    \item $p_0,p_1$ (resp. $p_f$) has no incoming (resp. outgoing) transitions in $\Aut_{A_0}$, $\Aut_{A_0\rightsquigarrow B}$ and $\Aut_{A_0\stackrel{C}\rightsquigarrow B}$.
    %
    \item All the incoming non-$\varepsilon$ transitions of $\langle p_b, A\rangle$ for $b \in \{0,1\}$ and $A \in \Gamma_{A_0}$ in $\Aut_{A_0\rightsquigarrow B}$ and $\Aut_{A_0\stackrel{C}\rightsquigarrow B}$ are $A$-transitions.
 %moreover, $\langle p_1, A\rangle$ for $A\in\Gamma_{A_0}$ cannot be reached from $p_0$ in $\Aut_{A_0\stackrel{C}\rightsquigarrow B}$,
    %
%    \item $\Lang(\Aut_{A_0})\subseteq\Lang(\Aut_{A_0\rightsquigarrow A_0})$, 
    %
    \item For each $B \in \Gamma_{A_0} \setminus \{A'_0\}$ such that $\Lang(\Aut_{A_0 \rightsquigarrow B}) \neq \emptyset$,  we have $A_0 \in \Lang((\Aut_{A_0 \rightsquigarrow B})^{\post^*}_{\Pp_{A_0}}(p_0))$.  (Intuitively, the configuration $(p_0, A_0)$ is reachable from $(p_0, w)$ for every $w \in \Lang(\Aut_{A_0 \rightsquigarrow B})$ in $\Pp_{A_0}$. ) 
    %As a result, for $B \in \Gamma_{A_0} \setminus \{A'_0\}$ such that $\Lang(\Aut_{A_0 \rightsquigarrow B}) \neq \emptyset$, we have $\Lang((\Aut_{A_0})^{\post^*}_{\Pp_{A_0}}(p_0)) \subseteq \Lang((\Aut_{A_0 \rightsquigarrow B})^{\post^*}_{\Pp_{A_0}}(p_0))$.
\end{itemize}

%Moreover, we have the following proposition of $\Aut_{A_0\stackrel{C}\rightsquigarrow A_0'}$,
%
%%%%%%%%%%%%%%%%%%%%%%%%%%%%%%%%%%%%%%%% removed
%%%%%%%%%%%%%%%%%%%%%%%%%%%%%%%%%%%%%%%% removed
\hide{
\begin{proposition}\label{prop:nfa-A_0}
    Let $C\in\Gamma_{A_0}\setminus\{A_0'\}$ and $B\in\Gamma_{A_0}$, we have
\begin{enumerate}
    \item $\Lang(\Aut_{A_0\stackrel{C}\rightsquigarrow A_0'}) = \{[A_0']\cdot S \mid S\in\Lang(\Aut_{A_0\rightsquigarrow C})\}$.
    \item $\Lang(\Aut_{A_0 \stackrel{C}{\rightsquigarrow} A_0'}) = \{[A_0'] \cdot S \mid S'\cdot [A_0']\cdot S\in\Lang(\Aut_{A_0\stackrel{C}\rightsquigarrow B})\}$.
\end{enumerate}
    
\end{proposition}

\paragraph*{The proof of $\Lang(\Aut_{A_0\stackrel{C}\rightsquigarrow A_0'}) = \{[A_0']\cdot S \mid S\in\Lang(\Aut_{A_0\rightsquigarrow C})\}$}
\begin{proof}
    We prove the equation by showing the left-hand side is a subset of the right-hand side and vice versa.

    We prove the left-hand side first, $\Lang(\Aut_{A_0\stackrel{C}\rightsquigarrow A_0'}) \subseteq \{[A_0']\cdot S \mid S\in\Lang(\Aut_{A_0\rightsquigarrow C})\}$, that is 
    for each $S\in \Lang(\Aut_{A_0\stackrel{C}\rightsquigarrow A_0'})$, $S = [A_0']\cdot S'$ for some $S'$, $S'\in \Lang(\Aut_{A_0\rightsquigarrow C})$.
    % for each $S\in\Lang(\Aut_{A_0\stackrel{C}\rightsquigarrow A_0'})$, $S \in\{[A_0']\cdot S \mid S\in\Lang(\Aut_{A_0\rightsquigarrow C})\}$.

    From the definition of $\Aut_{A_0\stackrel{C}\rightsquigarrow A_0'}$ and $\Aut_{A_0\rightsquigarrow C}$, we know that $p_1, \langle p_1,A_0'\rangle$ have no ingoing and outgoing transitions in $\Aut_{A_0\rightsquigarrow C}$, hence $S$ must use the transitions $(p_1,A_0',\langle p_1,A_0'\rangle)$ and $(\langle p_1,A_0'\rangle, C, p)$ for some $p\in P_{A_0}'$ where $p_0\xRightarrow[\Aut_{A_0\rightsquigarrow C}]{C}p$, moreover,
    % we know that $S$ must use the transition $(p_1,A_0',\langle p_1,A_0'\rangle)$. From $\Lang(\Aut_{A_0\rightsquigarrow C})\cap \act^*A_0'\act^* = \emptyset$, we know that $\langle p_1,A_0'\rangle$ has no ingoing transitions in $\Aut_{A_0\rightsquigarrow C}$ which implies that $\langle p_1,A_0'\rangle$ has no outgoing transitions in $\Aut_{A_0\rightsquigarrow C}$, and since $p_1$ has no ingoing transitions, we know that $S$ must use the transition $(\langle p_1,A_0'\rangle, C, p)$ for some $p\in P_{A_0}'$ where $p_0\xRightarrow[\Aut_{A_0\rightsquigarrow C}]{C}p$, moreover,

    % we know that for each $S'\in\Lang(\Aut_{A_0\stackrel{C}\rightsquigarrow A_0'})$, $S' = [A_0',C]\cdot S$ for some $S\in\act^*$, moreover,
    $$p_1\xrightarrow[\Aut_{A_0\stackrel{C}\rightsquigarrow A_0'}]{A_0'}\langle p_1,A_0'\rangle\xrightarrow[\Aut_{A_0\stackrel{C}\rightsquigarrow A_0'}]{C}p\xRightarrow[\Aut_{A_0\stackrel{C}\rightsquigarrow A_0'}]{S''}p_f,$$
    where $S = [A_0']\cdot S' = [A_0'C]\cdot S''$ for some $S''\in\act^*$, moreover, since $p_1, \langle p_1,A_0'\rangle$ have no ingoing and outgoing transitions in $\Aut_{A_0\rightsquigarrow C}$, we know that there is no new transitions in $p\xRightarrow[\Aut_{A_0\stackrel{C}\rightsquigarrow A_0'}]{S''}p_f$, then
    $$p_0\xRightarrow[\Aut_{A_0\rightsquigarrow C}]{C}p\xRightarrow[\Aut_{A_0\rightsquigarrow C}]{S''}p_f,$$
    therefore $S' = [C]\cdot S''\in\Lang(\Aut_{A_0\rightsquigarrow C})$.
    % we now prove $[C]\cdot S\in\Lang(\Aut_{A_0\rightsquigarrow C})$.

    Now we prove the right-hand side, $\{[A_0']\cdot S \mid S\in\Lang(\Aut_{A_0\rightsquigarrow C})\}\subseteq \Lang(\Aut_{A_0\stackrel{C}\rightsquigarrow A_0'}) $, that is 
    for each $S\in\Lang(\Aut_{A_0\rightsquigarrow C})$, $[A_0']\cdot S\in\Lang(\Aut_{A_0\stackrel{C}\rightsquigarrow A_0'})$.
    % for each $S \in\{[A_0']\cdot S \mid S\in\Lang(\Aut_{A_0\rightsquigarrow C})\}$, $S\in\Lang(\Aut_{A_0\stackrel{C}\rightsquigarrow A_0'})$.

    From the definition of $\Aut_{A_0\rightsquigarrow C}$, we know that for each $S\in \Lang(\Aut_{A_0\rightsquigarrow C})$, $S = [C]\cdot S'$ for some $S'\in\act^*$, hence there exists some $p\in P_{A_0}'$, 
    $$p_0\xRightarrow[\Aut_{A_0\rightsquigarrow C}]{C}p\xRightarrow[\Aut_{A_0\rightsquigarrow C}]{S'}p_f.$$
    From the definition of $\Aut_{A_0\stackrel{C}\rightsquigarrow A_0'}$, we know that
    $$p_1\xrightarrow[\Aut_{A_0\stackrel{C}\rightsquigarrow A_0'}]{A_0'}\langle p_1,A_0'\rangle\xrightarrow[\Aut_{A_0\stackrel{C}\rightsquigarrow A_0'}]{C}p\xRightarrow[\Aut_{A_0\stackrel{C}\rightsquigarrow A_0'}]{S'}p_f,$$
    therefore, $[A_0']\cdot S = [A_0'C]\cdot S'\in\Lang(\Aut_{A_0\stackrel{C}\rightsquigarrow A_0'})$.
\end{proof}

\paragraph*{The proof of $\Lang(\Aut_{A_0\stackrel{C}\rightsquigarrow A_0'}) = \{[A_0']\cdot S \mid S'\cdot [A_0']\cdot S\in\Lang(\Aut_{A_0\stackrel{C}\rightsquigarrow B})\}$}.
\begin{proof}
    We prove the equation by showing the left-hand side is a subset of the right-hand side and vice versa.

    We prove the left-hand side first, $\Lang(\Aut_{A_0\stackrel{C}\rightsquigarrow A_0'}) \subseteq \{[A_0']\cdot S \mid S'\cdot [A_0']\cdot S\in\Lang(\Aut_{A_0\stackrel{C}\rightsquigarrow B})\}$, that is for each 
    $S\in\Lang(\Aut_{A_0\stackrel{C}\rightsquigarrow A_0'})$, $S = [A_0']\cdot S'$ for some $S'$, $S''\cdot[A_0']\cdot S'\in\Lang(\Aut_{A_0\stackrel{C}\rightsquigarrow B})$ for some $S''$.
    % $S\in\Lang(\Aut_{A_0\stackrel{C}\rightsquigarrow A_0'})$, $S \in\{[A_0']\cdot S \mid S'\cdot [A_0']\cdot S\in\Lang(\Aut_{A_0\stackrel{C}\rightsquigarrow B})\}$.

    From 
    $$\Lang(\Aut_{A_0\stackrel{C}\rightsquigarrow A_0'}) = \{[A_0']\cdot S\mid S\in \Lang(\Aut_{A_0\rightsquigarrow C})\}$$
    and the definition of $\Aut_{A_0\rightsquigarrow C}$, we know that $S' = [C]\cdot S'''$ for some $S'''$, 
    % for each $S\in\Lang(\Aut_{A_0\stackrel{C}\rightsquigarrow A_0'})$, $S\cap A_0'C\act^*\neq\emptyset$, then we let $S = [A_0'C]\cdot S'$ for some $S'\in\act^*$, 
    moreover, there exists some $S''\in B\act^*$, $(p_1,[A_0'C]\cdot S''')\xRightarrow{\Pp_{A_0}'}(p_1, S''\cdot [A_0'C]\cdot S''')$, hence $S''\cdot [A_0'C]\cdot S''' \in \Lang((\Aut_{A_0\stackrel{C}\rightsquigarrow A_0'})^{\post^*}_{\Pp'_{A_0}})$.

    From the definition of $\Aut_{A_0\stackrel{C}\rightsquigarrow B}$, we know that $$\Lang(\Aut_{A_0\stackrel{C}\rightsquigarrow B}) = B\act^*A_0'C\act^*\cap\Lang((\Aut_{A_0\stackrel{C}\rightsquigarrow A_0'})^{\post^*}_{\Pp'_{A_0}}),$$
    hence we have $S''\cdot[A_0']\cdot S' = S''\cdot[A_0'C]\cdot S''' \in\Lang(\Aut_{A_0\stackrel{C}\rightsquigarrow B})$.

    Now we prove the right-hand side, $\{[A_0']\cdot S \mid S'\cdot [A_0']\cdot S\in\Lang(\Aut_{A_0\stackrel{C}\rightsquigarrow B})\}\subseteq \Lang(\Aut_{A_0\stackrel{C}\rightsquigarrow A_0'})$, that is for each 
    $S'\cdot[A_0']\cdot S\in\Lang(\Aut_{A_0\stackrel{C}\rightsquigarrow B})$, $[A_0']\cdot S\in\Lang(\Aut_{A_0\stackrel{C}\rightsquigarrow A_0'})$.
    
    % $S \in\{[A_0']\cdot S \mid S'\cdot [A_0']\cdot S\in\Lang(\Aut_{A_0\stackrel{C}\rightsquigarrow B})\}$, $S\in\Lang(\Aut_{A_0\stackrel{C}\rightsquigarrow A_0'})$.

    From the definition of $\Aut_{A_0\stackrel{C}\rightsquigarrow B}$, we know that $S\cap C\act^*\neq\emptyset$, hence we let $S = [C]\cdot S''$ for some $S''$, then $S'\cdot[A_0'C]\cdot S''\in\Lang(\Aut_{A_0\stackrel{C}\rightsquigarrow B})$. 
    
    From $$\Lang((\Aut_{A_0\stackrel{C}\rightsquigarrow A_0'})^{\post^*}_{\Pp'_{A_0}}) = \Lang((\Aut_{A_0})^{\post^*}_{\Pp'_{A_0}})\cup\bigcup\limits_{B'\in\Gamma_{A_0}}\Lang(\Aut_{A_0\stackrel{C}\rightsquigarrow B'}),$$
    we know that
    $$S'\cdot[A_0'C]\cdot S''\in \Lang((\Aut_{A_0\stackrel{C}\rightsquigarrow A_0'})^{\post^*}_{\Pp'_{A_0}}),$$
    moreover, from the definition of $\Pp'_{A_0}$, we know that,
    $$[C]\cdot S''\in \Lang((\Aut_{A_0\stackrel{C}\rightsquigarrow A_0'})^{\post^*}_{\Pp'_{A_0}}).$$

    % From $$\Lang((\Aut_{A_0\stackrel{C}\rightsquigarrow A_0'})^{\post^*}_{\Pp'_{A_0}}) = \Lang((\Aut_{A_0})^{\post^*}_{\Pp'_{A_0}})\cup\bigcup\limits_{B'\in\Gamma_{A_0}}\Lang(\Aut_{A_0\stackrel{C}\rightsquigarrow B'}),$$
    Since $[C]\cdot S''\cap \act^*A_0'\act^* = \emptyset$, we know that $$[C]\cdot S''\in \Lang((\Aut_{A_0})^{\post^*}_{\Pp'_{A_0}}(p_0)).$$
    From the definition of $\Aut_{A_0\rightsquigarrow C}$, we know that $[C]\cdot S''\in \Lang(\Aut_{A_0\rightsquigarrow C})$,
    moreover, we have $[A_0']\cdot S' = [A_0'C]\cdot S''\in \Lang(\Aut_{A_0\stackrel{C}\rightsquigarrow A_0'})$.
\end{proof}
}
%%%%%%%%%%%%%%%%%%%%%%%%%%%%%%%%%%%%%%%% removed
%%%%%%%%%%%%%%%%%%%%%%%%%%%%%%%%%%%%%%%% removed

We are ready to present the procedure that computes effectively $\AutReach$, a finite set of pairs $(\aname_1\cdots\aname_k,(\Aut_1,\cdots,\Aut_k))$ where each of $\Aut_i$'s are of the forms as follows:
\begin{itemize}
    \item $\Aut_{A\rightsquigarrow B}$ for $A\in\act_{\STK}\cup\act_{\SIT}$ and $B\in\Gamma_A$,
    % \item $\Aut_{A_0}$,
    \item $\Aut_{A_0\rightsquigarrow B}$ for $B\in\Gamma_{A_0}\setminus\{A_0'\}$,
    \item $\Aut_{A_0\stackrel{C}\rightsquigarrow B}$ for $B\in\Gamma_{A_0}$, $C\in\Gamma_{A_0}\setminus\{A_0'\}$.
\end{itemize}
Initially, let $\AutReach := \{(\aft(A_0),(\Aut_{A_0\rightsquigarrow A_0}))\}$.
% where $\Aut_{A_0} = (Q_{A_0}, \act, \delta_{A_0}, \{p_0\}, \{p_f\})$, where $\delta_{A_0} = \{(p_0,A_0,p_f)\}$.
% where $\Lambda = [\gamma_{\init}]$ if $\gamma_{\init}\in\Gamma_{\STK}$, $\Lambda = []$ otherwise. 
Then it adds pairs to $\AutReach$, according to the following saturation rules, until no more pairs can be added.

\smallskip
\fbox
{
\begin{minipage}{\textwidth}
{\small
\begin{enumerate}
    %
    \item If $A \xrightarrow{\startactivity(\bot)}A' \in\Delta$, $\lmd(A')=\STK$ or $\singleinstance$, $(\theta, (\Aut_1,\cdots,\Aut_k)) \in \AutReach$ such that 
    $\Aut_1 = \Aut_{B \rightsquigarrow A}$ or $\Aut_{B \stackrel{C}{\rightsquigarrow} A}$ for some $B$ and $C$,
    % $\Lang(\Aut_1) \neq \emptyset$ and $\Aut_1 = \Aut_{B, A}$ for some $B \in \act_\singletask$ or $\Aut_1 = \Aut_{A_0,A}$ or $\Aut_1 = \Aut_{A_0,A,B}$ for some $B\in\act\setminus(\act_{\SIT}\cup\{A_0'\})$ 
    moreover, $\namefun_{A'}(\theta) = \bot$,
    then let $\AutReach: = \AutReach \cup \{(\aft(A')\theta, (\Aut_{A'},\Aut_1,\cdots,\Aut_k))\}$.
    % where $\Aut_1'$ is obtained from $\Aut_1$ by 
    % removing all non-$A$ transitions out of $p_0$, then removing all transitions cannot be reached from $p_0$.
        \textbf{[launch an $A'$-task]}

    \item If $A \xrightarrow{\startactivity(\bot)}A' \in\Delta$, $\lmd(A')=\STK$ or $\singleinstance$, $(\theta, (\Aut_1,\cdots,\Aut_k)) \in \AutReach$ such that 
    $\Aut_1 = \Aut_{B \rightsquigarrow A}$ or $\Aut_{B \stackrel{C}{\rightsquigarrow} A}$ for some $B$ and $C$,
    % $\Lang(\Aut_1) \neq \emptyset$ and $\Aut_1 = \Aut_{B, A}$ for some $B \in \act_\singletask$ or $\Aut_1 = \Aut_{A_0,A}$ or $\Aut_1 = \Aut_{A_0,A,B}$ for some $B\in\act\setminus(\act_{\SIT}\cup\{A_0'\})$, 
    moreover, $\namefun_{A'}(\theta) = i$ for some $i \in [k]$ with $i > 1$, then
    \begin{itemize}
        \item if $\Aut_i = \Aut_{A'\rightsquigarrow B'}$ for some $B' \in \Gamma_{A'}$,
        then let $\AutReach:= \AutReach \cup \{(\theta', (\Aut_{A'}, \Aut_1, \cdots,\Aut_{i-1},\Aut_{i+1},\cdots,\Aut_{k}))\}$, 
        % where $\theta' = \aname_i\aname_1\dots\aname_{i-1}\aname_{i+1}\dots\aname_k$. 
        % and $\Aut_1'$ is obtained from $\Aut_1$ by removing all non-$A$ transitions out of $p_0$, then removing all transitions cannot be reached from $p_0$.
        \textbf{[escalate the $A'$-task to be the top and reset its content to $[A']$]}
        \item if $\Aut_i = \Aut_{A_0\rightsquigarrow B'}$ or $\Aut_{A_0 \stackrel{B'}{\rightsquigarrow} C'}$ for some $B' \in \Gamma_{A_0}\setminus\{A_0'\}$ and $C' \in \Gamma_{A_0}$,
        then $A' = A'_0$,  let 
        $\AutReach:= \AutReach \cup \{(\theta', (\Aut_{A_0\stackrel{B'} {\rightsquigarrow} A_0'}, \Aut_1, \cdots,\Aut_{i-1},\Aut_{i+1},\cdots,\Aut_{k}))\}$,  
        % and $\Aut_1'$ is obtained from $\Aut_1$ by removing all non-$A$ transitions out of $p_0$, then removing all transitions cannot be reached from $p_0$.
        \textbf{[escalate the $A_0$-task to be the top and push $A_0'$ to the top task or pop the top task until reach an instance of $A_0'$]}
    \end{itemize}
        where $\theta' = \aname_i\aname_1\dots\aname_{i-1}\aname_{i+1}\dots\aname_k$. 
        %
    \item If $(\aname_1 \cdots \aname_k, (\Aut_1,\cdots,\Aut_k)) \in \AutReach$ and $k>1$, then let $\AutReach := \AutReach \cup \{(\aname_2\dots\aname_k, (\Aut_2,\cdots,\Aut_k))\}$.
        \textbf{[pop all activities in the top task]}
%
    \item If $(\theta, (\Aut_1,\cdots,\Aut_k)) \in \AutReach$, then
    \begin{itemize}
        \item if $\Aut_1 = \Aut_{A\rightsquigarrow B}$ for some $A\in\act_{\STK}\cup\act_{\SIT}$ and $B \in \Gamma_A$, moreover, $\Lang(\Aut_{A\rightsquigarrow B'}) \neq \emptyset$ for $B'  \in \Gamma_A$, then let $\AutReach := \AutReach \cup \{(\theta, (\Aut_{A\rightsquigarrow B'}, \Aut_2,\cdots,\Aut_k))\}$,
        %
%        \item if $\Aut_1 = \Aut_{A_0\rightsquigarrow B}$ and $\Lang(\Aut_{A_0\rightsquigarrow B'}) \neq \emptyset$ for $B, B' \in \Gamma_{A_0}\setminus\{A_0'\}$, then  $\AutReach := \AutReach \cup \{(\theta, (\Aut_{A_0 \rightsquigarrow B'}, \Aut_2,\cdots,\Aut_k))\}$,
      
      \item if  $\Aut_1 = \Aut_{A_0\rightsquigarrow B}$ for some $B \in \Gamma_{A_0} \setminus \{A'_0\}$, then
      \begin{itemize}
      \item if $\Lang(\Aut_{A_0\rightsquigarrow B'}) \neq \emptyset$ for $B' \in \Gamma_{A_0}\setminus\{A_0'\}$, then $\AutReach := \AutReach \cup \{(\theta, (\Aut_{A_0 \rightsquigarrow B'}, \Aut_2,\cdots,\Aut_k))\}$,
      %
      \item if $\Lang(\Aut_{A_0 \rightsquigarrow C}) \neq \emptyset$, $C \xrightarrow[]{\startactivity(\bot)} A_0' \in \Delta$, and $\Lang(\Aut_{A_0 \stackrel{C}{\rightsquigarrow} B'}) \neq \emptyset$ for $B, C \in \Gamma_{A_0} \setminus \{A'_0\}$ and $B' \in \Gamma_{A_0}$, then let $\AutReach := \AutReach \cup \{(\theta, (\Aut_{A_0\stackrel{C}\rightsquigarrow B'}, \Aut_2,\cdots,\Aut_k))\}$,
      \end{itemize}
        % \item if $\Aut_1 = \Aut_{A_0\rightsquigarrow B}$ for some $B\in\Gamma_{A_0}\setminus\{A_0'\}$ such that $B\xrightarrow[]{\startactivity(\bot)}A_0'\in\Delta$, then let $\AutReach := \AutReach \cup \{(\theta, (\Aut_{A_0\stackrel{B}\rightsquigarrow A_0'}, \Aut_2,\cdots,\Aut_k))\}$,
        \item if $\Aut_1 = \Aut_{A_0\stackrel{C}\rightsquigarrow B}$ for $C \in\Gamma_{A_0}\setminus\{A_0'\}$ and $B \in \Gamma_{A_0}$, then 
        \begin{itemize}
        \item let $\AutReach := \AutReach \cup \{(\theta, (\Aut_{A_0\rightsquigarrow C}, \Aut_2,\cdots,\Aut_k))\}$, 
        %
       \item if $\Lang(\Aut_{A_0 \stackrel{C}{\rightsquigarrow} B'}) \neq \emptyset$ for $B' \in \Gamma_{A_0}$, then let $\AutReach := \AutReach \cup \{(\theta, (\Aut_{A_0\stackrel{C}{\rightsquigarrow} B'}, \Aut_2,\cdots,\Aut_k))\}$.
       \end{itemize}
        % \item if $\Aut_1 = \Aut_{A_0\stackrel{C}\rightsquigarrow A_0'}$ for some $C\in\Gamma_{A_0}\setminus\{A_0'\}$ and $B\in\Gamma_{A_0}$ such that $\Lang(\Aut_{A_0\stackrel{C}\rightsquigarrow B})\neq\emptyset$, then let $\AutReach := \AutReach \cup \{(\theta, (\Aut_{A_0\stackrel{C}\rightsquigarrow B}, \Aut_2,\cdots,\Aut_k))\}$.
    \end{itemize}
    % then adds the tuple $(\theta, (\Aut_1^{\post^*},\Aut_2,\cdots,\Aut_k))$ to $\AutReach$.
%    such that $\Aut_1 = \Aut_A^B$ for some $A\in\act_{\STK},B\in\act$, then adds the tuple $(\theta, (\Aut_A^{B'},\Aut_2,\cdots,\Aut_k))$ to $\AutReach$ for each $B'\in\act$ with $\ConfSet(\Aut_A^{B'})\neq\emptyset$.
        \textbf{[simulate the behaviors of the top task]}
\end{enumerate}
}
\end{minipage}
}


\smallskip

The aforementioned 4th saturation rule is justified by the following proposition. 
%about $(\Aut_{A_0})^{\post^*}_{\Pp_{A_0}}$, $(\Aut_{A_0 \rightsquigarrow B})^{\post^*}_{\Pp_{A_0}}$ and $(\Aut_{A_0\stackrel{C}\rightsquigarrow B})^{\post^*}_{\Pp_{A_0}}$.

\begin{proposition} \label{prop-lm-A0-post}
The following facts hold for $(\Aut_{A_0 \rightsquigarrow B})^{\post^*}_{\Pp_{A_0}}$ and $(\Aut_{A_0\stackrel{C}\rightsquigarrow B})^{\post^*}_{\Pp_{A_0}}$.
\begin{itemize}
%    \item $\Lang((\Aut_{A_0})^{\post^*}_{\Pp_{A_0}}(p_0)) = \bigcup\limits_{B\in\Gamma_{A_0}\setminus\{A_0'\}} \Lang(\Aut_{A_0\rightsquigarrow B})$ and  
%    $$\Lang((\Aut_{A_0})^{\post^*}_{\Pp_{A_0}}(p_1)) = \bigcup\limits_{\Lang(\Aut_{A_0\rightsquigarrow C})\neq\emptyset,C \xrightarrow{\startactivity(\bot)}A'_0 \in \Delta, B' \in \Gamma_{A_0}} \Lang(\Aut_{A_0\stackrel{C}\rightsquigarrow B'}).$$ 
   %
    \item If $ \Lang(\Aut_{A_0 \rightsquigarrow B}) \neq \emptyset$, then $\Lang((\Aut_{A_0 \rightsquigarrow B})^{\post^*}_{\Pp_{A_0}}(p_0)) = \bigcup\limits_{B' \in \Gamma_{A_0}\setminus\{A_0'\}} \Lang(\Aut_{A_0\rightsquigarrow B'})$ and
    $$\Lang((\Aut_{A_0 \rightsquigarrow B})^{\post^*}_{\Pp_{A_0}}(p_1))  = \bigcup\limits_{\Lang(\Aut_{A_0\rightsquigarrow C})\neq\emptyset,C \xrightarrow{\startactivity(\bot)}A'_0 \in \Delta, B' \in \Gamma_{A_0}} \Lang(\Aut_{A_0\stackrel{C}\rightsquigarrow B'}).$$
        (Note that some of $\Lang(\Aut_{A_0\stackrel{C}\rightsquigarrow B'})$ here may be empty. ) 
    %moreover, for each $w \in \Lang((\Aut_{A_0})^{\post^*}_{\Pp_{A_0}}(p_0))$, $A_0'$ does not  occur in $w$. As a result, 
    %
%    \item For $B, B' \in \Gamma_{A_0} \setminus \{A'_0\}$ such that $\Lang(\Aut_{A_0\rightsquigarrow B}) \neq \emptyset$ and $\Lang(\Aut_{A_0\rightsquigarrow B'}) \neq \emptyset$, we have $\Lang(\Aut_{A_0\rightsquigarrow B'}) \subseteq  \Lang((\Aut_{A_0 \rightsquigarrow B})^{\post^*}_{\Pp_{A_0}}(p_0))$.
%

    %
%    \item $\Lang((\Aut_{A_0})^{\post^*}_{\Pp_{A_0}}(p_1)) = \bigcup\limits_{B\in\Gamma_{A_0},\Lang(\Aut_{A_0\rightsquigarrow C})\neq\emptyset,C\xrightarrow{\startactivity(\bot)}A_0\in\Delta}\Lang(\Aut_{A_0\stackrel{C}\rightsquigarrow B})$, 
%    moreover, for each $S\in\Lang(\Aut_{A_0\stackrel{C}\rightsquigarrow B})$, $A_0'\in S$ and $A_0'$ occurs in $S$ exactly once.
%
    \item If $ \Lang(\Aut_{A_0\stackrel{C}\rightsquigarrow B}) \neq \emptyset$, then $\Lang((\Aut_{A_0\stackrel{C}\rightsquigarrow B})^{\post^*}_{\Pp_{A_0}}(p_0)) = \Lang((\Aut_{A_0 \rightsquigarrow C})^{\post^*}_{\Pp_{A_0}}(p_0))$ and 
    $$\Lang((\Aut_{A_0\stackrel{C}\rightsquigarrow B})^{\post^*}_{\Pp_{A_0}}(p_1)) = 
    \bigcup\limits_{B'\in\Gamma_{A_0}}\Lang(\Aut_{A_0\stackrel{C}\rightsquigarrow B'}) \cup  \Lang((\Aut_{A_0 \rightsquigarrow C})^{\post^*}_{\Pp_{A_0}}(p_1)).$$
\end{itemize}
\end{proposition}

The aforementioned saturation procedure terminates since $\Theta_\Mm$ is finite and the set of {\NFA}s $\Aut_{A\rightsquigarrow B}$, $\Aut_{A_0\rightsquigarrow B}$ and $\Aut_{A_0\stackrel{C}\rightsquigarrow B}$ occurring in $\AutReach$ is finite.
Let $\AutReach_f$ denote the value of $\AutReach$ when the aforementioned procedure terminates. 

We are going to prove Lemma~\ref{lem:iff-recog} for the case $\lmd(A_0) \neq \singletask$, that is, for each $\theta=\aname_1\dots\aname_k\in\Theta_{\Mm}$, 
%
\[\RLang(\Mm, \theta) = \bigcup \limits_{(\theta, (\Aut_1, \cdots, \Aut_k)) \in \AutReach_f} \Rel((\Aut_1, \cdots, \Aut_k)).\]

We prove the equation by showing the left-hand side is a subset of the right-hand side and vice versa. 

\paragraph*{The proof of $\RLang(\Mm, \theta) \subseteq \bigcup \limits_{(\theta, (\Aut_1, \cdots, \Aut_k)) \in \AutReach_f} \Rel((\Aut_1, \cdots, \Aut_k))$}
\begin{proof}
    For $n \ge 1$, let $\xrightarrow[\Mm]{}^n$ denote the $n$-fold composition of $\xrightarrow[\Mm]{}$. Moreover, by convention, let $\xrightarrow[\Mm]{}^0$ be the identity relation on $\confs(\Mm)$. 
    Then for each $\theta = \aname_1 \dots \aname_k \in \Theta_\Mm$, we prove by an induction on $n \ge 0$ the following claim.
    
    \smallskip
    
    \noindent {\bf Claim}. \emph{For each $n \ge 0$ and configuration $\rho =  ((S_1, \aname_1), \cdots, (S_k, \aname_k))$ such that 
    %
    $(([A_0], \aft(A_0))) \xrightarrow[\Mm]{}^n \rho$, let $\theta = \aname_1 \cdots \aname_k$, then  
    %
    there is  $(\Aut_1, \dots, \Aut_k)$ such that $(\theta, (\Aut_1, \dots, \Aut_k)) \in \AutReach_f$ and $(S_1, \cdots, S_k) \in \Rel((\Aut_1,\cdots, \Aut_k))$}.   
    
    %$\rho \in  \ConfSet(\Aut_1) \times \dots \times \ConfSet(\Aut_k)$.
    
    \smallskip
    
    \noindent \emph{Induction base $n = 0$}. From $(([A_0], \aft(A_0))) \xrightarrow[\Mm]{}^0 \rho$, we have $\rho = (([A_0], \aft(A_0)))$. From the computation of $\AutReach_f$, we know that $(\aft(A_0), (\Aut_{A_0\rightsquigarrow A_0})) \in \AutReach_f$. Since $[A_0] \in \Lang(\Aut_{A_0})\subseteq\Lang(\Aut_{A_0\rightsquigarrow A_0})$, the claim holds for $n = 0$. 
    
    \smallskip
    
    \noindent \emph{Induction step $n > 0$}. Suppose the claim holds for $n-1$. Assuming $(([A_0], \aft(A_0))) \xrightarrow[\Mm]{}^n \rho =  ((S_1, \aname_1), \dots, (S_k, \aname_k))$, we are going to show that there is  $(\Aut_1, \dots, \Aut_k)$ such that $(\theta, (\Aut_1, \dots, \Aut_k)) \in \AutReach_f$ and $(S_1, \cdots, S_k) \in \Rel((\Aut_1,\cdots, \Aut_k))$, that is, $S_i \in \Lang(\Aut_i)$ for each $i \in [k]$. 
    
    From $(([A_0], \aft(A_0))) \xrightarrow[\Mm]{}^n \rho$,  we know that there is a configuration $\rho' = ((S'_1, \aname'_1), \dots, (S'_l, \aname'_l))$ such that $(([A_0], \aft(A_0))) \xrightarrow[\Mm]{}^{n-1} \rho' \xrightarrow[\Mm]{} \rho$.
    
    Let $\theta'= \aname'_1 \cdots \aname'_l$. Then from the induction hypothesis, there exists $(\theta', (\Aut'_1, \dots, \Aut'_l)) \in \AutReach_f$ such that $S'_j \in \Lang(\Aut'_j)$ for each $j \in [l]$. Moreover, $\Aut'_1$ is of one of the following forms,
 \begin{itemize}
   \item  $\Aut'_1 = \Aut_{A \rightsquigarrow B}$ for some $A \in \act_\singletask \cup \act_\singleinstance$ and $B \in \Gamma_A$, 
   \item $\Aut'_1 = \Aut_{A_0 \rightsquigarrow B}$ for some $B \in \Gamma_{A_0} \setminus \{A'_0\}$, 
   \item $\Aut'_1 = \Aut_{A_0 \stackrel{C}{\rightsquigarrow} B}$ for some $C \in \Gamma_{A_0} \setminus \{A'_0\}$ and $B \in \Gamma_{A_0}$. 
  \end{itemize}
    % Moreover, let $A \in \act_\singletask$ and $B \in \act \setminus \act_\singleinstance$ such that $\Aut'_1 = \Aut_{A, B}$. 
    
If $\tau = \back$, then we distinguish between whether $|S'_1| > 1$ or not. 
\begin{itemize}
\item If $|S'_1| > 1$, then $\theta' = \theta$, $k=l$, and $(S'_2, \cdots, S'_l) = (S_2, \cdots, S_k)$. 
\begin{itemize}
\item If $\Aut'_1 = \Aut_{A \rightsquigarrow B}$  for some $A \in \act_\singletask \cup \act_\singleinstance$ and $B \in \Gamma_A$, then $S_1 \in \Lang(\Aut_{A \rightsquigarrow B'})$ for some $B'$.  According to the 4th saturation rule, $(\theta, (\Aut_{A \rightsquigarrow B'}, \Aut'_1, \cdots, \Aut'_l)) \in \AutReach_f$. Since $(S_1, \cdots, S_k) \in \Rel((\Aut_{A \rightsquigarrow B'}, \Aut'_2, \cdots, \Aut'_l))$, we conclude that the claim holds for $n$ in this situation. 
%
\item If $\Aut'_1 = \Aut_{A_0 \rightsquigarrow B}$  for some $B \in \Gamma_{A_0} \setminus \{A'_0\}$, then the discussion is similar to the previous case, i.e. $\Aut'_1 = \Aut_{A \rightsquigarrow B}$. 
%
\item If $\Aut'_1 = \Aut_{A_0 \stackrel{C}{\rightsquigarrow} A'_0}$ for some $C \in \Gamma_{A_0} \setminus \{A'_0\}$, then $S_1 \in \Lang(\Aut_{A \rightsquigarrow C})$. According to the 4th saturation rule, $(\theta, (\Aut_{A \rightsquigarrow C}, \Aut'_1, \cdots, \Aut'_l)) \in \AutReach_f$. Since $(S_1, \cdots, S_k) \in \Rel((\Aut_{A \rightsquigarrow C}, \Aut'_2, \cdots, \Aut'_l))$, we conclude that the claim holds for $n$ in this situation. 
%
\item If $\Aut'_1 = \Aut_{A_0 \stackrel{C}{\rightsquigarrow} B}$ for some $C \in \Gamma_{A_0} \setminus \{A'_0\}$ and $B \in \Gamma_{A_0}$ such that $B \neq A'_0$, then $S_1 \in \Lang(\Aut_{A_0 \stackrel{C}{\rightsquigarrow} B'})$ for some $B'$. According to the 4th saturation rule, we have $(\theta, (\Aut_{A_0 \stackrel{C}{\rightsquigarrow} B'}, \Aut'_1, \cdots, \Aut'_l)) \in \AutReach_f$. Since 
%
$$(S_1, \cdots, S_k) \in \Rel((\Aut_{A_0 \stackrel{C}{\rightsquigarrow} B'}, \Aut'_2, \cdots, \Aut'_l)),$$ 
%
we conclude that the claim holds for $n$ in this situation. 
%
\end{itemize}
%
\item If $|S'_1| = 1$, then $k = l - 1$, and $(S_1,\cdots,S_k) = (S_2',\cdots,S_l')$. According to the 3rd saturation rule, $(\theta, (\Aut_2',\cdots,\Aut_l')) \in \AutReach_f$. From $(S_1,\dots,S_k) = (S'_2, \cdots, S'_l) \in \Lang(\Aut_2') \times \cdots \times \Lang(\Aut_l') = \Rel((\Aut'_2, \cdots, \Aut'_l))$, we conclude that the claim holds for $n$ in this situation. 
\end{itemize}


Let us assume $\tau = B \xrightarrow{\startactivity(\bot)} B'$ in the sequel. From the definition of $\singletask$-dominating $\AMASS$, there are the following three cases. 
\begin{itemize}
\item $\lmd(B) \neq \singleinstance$ and $\lmd(B') = \standard$ or $\singletop$, 
%
\item $\lmd(B) \neq \singleinstance$ and $\lmd(B') = \singletask$ or $\singleinstance$, 
%
\item $\lmd(B) = \singleinstance$ and $\lmd(B') = \singletask$ or $\singleinstance$. 
\end{itemize}

\smallskip

\noindent \emph{Case $\lmd(B) \neq \singleinstance$ and $\lmd(B') = \standard$ or $\singletop$}. In this case, $\theta = \theta'$. Moreover, $\rho = \rho'$ or $\rho$ is obtained from $\rho'$ by pushing $C$ to the top task.

If $\rho = \rho'$, then we are done. 

If $\rho$ is obtained from $\rho'$ by pushing $B'$ to the top task, then we distinguish between the different forms of $\Aut'_1$. 
\begin{itemize}
\item If $\Aut'_1 = \Aut_{A \rightsquigarrow B}$ for some $A \in \act_\singletask \cup \act_\singleinstance$, then 
 $S_1 \in  \Lang(\Aut_{A \rightsquigarrow B'})$. According to the 4th saturation rule, $(\theta, (\Aut_{A \rightsquigarrow B'}, \Aut'_2, \cdots, \Aut'_l)) \in \AutReach_f$. Since $(S_1, \cdots, S_k) \in \Rel((\Aut_{A \rightsquigarrow B'}, \Aut'_2, \cdots, \Aut'_l))$, we know that the claim holds for $n$ in this situation.

%
\item If $\Aut'_1 = \Aut_{A_0 \rightsquigarrow B}$, then the discussion is similar. 
%
\item If $\Aut'_1 = \Aut_{A_0 \stackrel{C}{\rightsquigarrow} B}$ for some $C \in \Gamma_{A_0} \setminus \{A'_0\}$, then $S_1 \in  \Lang(\Aut_{A \stackrel{C}{\rightsquigarrow} B'})$.  According to the 4th saturation rule, $(\theta, (\Aut_{A \stackrel{C}{\rightsquigarrow} B'}, \Aut'_2, \cdots, \Aut'_l)) \in \AutReach_f$. Since $(S_1, \cdots, S_k) \in \Rel((\Aut_{A \stackrel{C}{\rightsquigarrow} B'}, \Aut'_2, \cdots, \Aut'_l))$, we know that the claim holds for $n$ in this situation.
\end{itemize}

\smallskip

\noindent \emph{Case $\lmd(B) \neq \singleinstance$ and $\lmd(B') = \singletask$ or $\singleinstance$}. 

If $\theta = \theta'$, then we distinguish between the forms of $\Aut'_1$. 
\begin{itemize}
\item If $\Aut'_1 = \Aut_{A \rightsquigarrow B}$ for some $A \in \act_\singletask \cup \act_\singleinstance$, then $B' = A$ and $S_1 = [A]$. According to the 4th saturation rule, $(\theta, (\Aut_{A \rightsquigarrow A} = \Aut_A, \Aut'_2, \cdots, \Aut'_l)) \in \AutReach_f$. Since $(S_1, \cdots, S_k) \in \Rel((\Aut_{A \rightsquigarrow A}, \Aut'_2, \cdots, \Aut'_l))$, we know that the claim holds for $n$ in this situation.

%
\item If $\Aut'_1 = \Aut_{A_0 \rightsquigarrow B}$, then $B' = A'_0$ and $S_1 = [A'_0] \cdot S'_1$.  According to the 4th saturation rule, $(\theta, (\Aut_{A_0 \stackrel{B}{\rightsquigarrow} A'_0}, \Aut'_2, \cdots, \Aut'_l)) \in \AutReach_f$. Since $(S_1, \cdots, S_k) \in \Rel((\Aut_{A_0 \stackrel{B}{\rightsquigarrow} A'_0}, \Aut'_2, \cdots, \Aut'_l))$, we know that the claim holds for $n$ in this situation.

%
\item If $\Aut'_1 = \Aut_{A_0 \stackrel{C}{\rightsquigarrow} B}$ for some $C \in \Gamma_{A_0} \setminus \{A'_0\}$, then $B' = A'_0$ and $S_1 \in  \Lang(\Aut_{A \stackrel{C}{\rightsquigarrow} A'_0})$.  According to the 4th saturation rule, $(\theta, (\Aut_{A \stackrel{C}{\rightsquigarrow} A'_0}, \Aut'_2, \cdots, \Aut'_l)) \in \AutReach_f$. Since $(S_1, \cdots, S_k) \in \Rel((\Aut_{A \stackrel{C}{\rightsquigarrow} A'_0}, \Aut'_2, \cdots, \Aut'_l))$, we know that the claim holds for $n$ in this situation.
\end{itemize}

Let us assume $\theta \neq \theta'$ in the sequel. We distinguish between the following two subcases. 
\begin{itemize}
    \item \emph{Subcase $\namefun_{B'}(\theta') = \bot$}. Then $k=l+1$, $S_1=[B']$, and $(S_2,\cdots,S_k)=(S_1',\cdots,S_l')$.  According to the 1st saturation rule, $(\theta, (\Aut_{B'}, \Aut_1', \cdots, \Aut_l')) \in \AutReach_f$.
    %, where $\Lang(\Aut_1') \cap A\act^* \neq \emptyset$, and $S_1=[A']\in\ConfSet(\Aut_{A'})$, 
   From $(S_1, \cdots, S_k) = ([B'], S_1', \cdots, S_l') \in \Lang(\Aut_{B'}) \times \Lang(\Aut_1') \times \cdots \times \Lang(\Aut_l')$, we conclude that the claim holds for $n$ in this subcase.
    %
    \item \emph{Subcase $\namefun_{B'}(\theta') = i \neq \bot$ and $i > 1$}. Then $k = l$ and $(S_2, \dots, S_k) = (S'_1, \dots, S'_{i-1}, S'_{i+1}, \dots, S'_l)$. 
%    Moreover, $S_1$ depends on the form of $\Aut'_i$. 
\begin{itemize}
\item If $\Aut'_i = \Aut_{B' \rightsquigarrow C'}$ for some $C' \in \Gamma_{B'}$, then $S_1 = [B']$. According to the 2nd saturation rule, $(\theta, (\Aut_{B'}, \Aut'_1, \cdots, \Aut'_{i-1}, \Aut'_{i+1}, \cdots, \Aut'_l)) \in \AutReach_f$. Since $(S_1, \cdots, S_k) \in \Rel((\Aut_{B'}, \Aut'_1, \cdots, \Aut'_{i-1}, \Aut'_{i+1}, \cdots, \Aut'_l))$, we know that the claim holds for $n$ in this situation.
%
\item If $\Aut'_i = \Aut_{A_0 \rightsquigarrow C}$ for some $C \in \Gamma_{A_0} \setminus \{A'_0\}$, then $B' = A'_0$ and $S_1 = [A'_0] \cdot S'_i \in \Lang(\Aut_{A_0 \stackrel{C}{\rightsquigarrow} A'_0})$.  According to the 2nd saturation rule, $(\theta, (\Aut_{A_0 \stackrel{C}{\rightsquigarrow} A'_0}, \Aut'_1, \cdots, \Aut'_{i-1}, \Aut'_{i+1}, \cdots, \Aut'_l)) \in \AutReach_f$. Since $(S_1, \cdots, S_k) \in \Rel((\Aut_{A_0 \stackrel{B}{\rightsquigarrow} A'_0}, \Aut'_1, \cdots, \Aut'_{i-1}, \Aut'_{i+1}, \cdots, \Aut'_l))$, we know that the claim holds for $n$ in this situation.

%
\item If $\Aut'_i = \Aut_{A_0 \stackrel{C}{\rightsquigarrow} C'}$ for some $C \in \Gamma_{A_0} \setminus \{A'_0\}$ and $C' \in \Gamma_{A_0}$, then $B' = A'_0$ and $S_1 \in  \Lang(\Aut_{A \stackrel{C}{\rightsquigarrow} A'_0})$.  According to the 2nd saturation rule, $(\theta, (\Aut_{A \stackrel{C}{\rightsquigarrow} A'_0}, \Aut'_1, \cdots, \Aut'_{i-1}, \Aut'_{i+1}, \cdots, \Aut'_l)) \in \AutReach_f$. Since $(S_1, \cdots, S_k) \in \Rel((\Aut_{A \stackrel{C}{\rightsquigarrow} A'_0}, \Aut'_1, \cdots, \Aut'_{i-1}, \Aut'_{i+1}, \cdots, \Aut'_l))$, we know that the claim holds for $n$ in this situation.
\end{itemize}
\end{itemize}
 
 \smallskip
 
 \noindent \emph{Case $\lmd(B) = \singleinstance$ and $\lmd(B') = \singletask$ or $\singleinstance$}. 
 
 The discussion is similar to the previous case. 
    
%%%%%%%%%%%%%%%%% the original proof removed
%%%%%%%%%%%%%%%%% the original proof removed
\hide{
    %Moreover, $\rho$ is obtained from $\rho'$ by a transition of $\Mm$. 
    We distinguish between the two situations $\theta' = \theta$ and $\theta' \neq \theta$.
    
    \smallskip
    
    \noindent \emph{Situation $\theta' = \theta$}.
    In this situation, the configuration $\rho$ is obtained from $\rho'$ by only updating the content of the top task. There are the following three sub-siutations.
    \begin{itemize}
        \item \emph{Sub-situation $\Aut'_1 = \Aut_{A\rightsquigarrow B}$ for some $A \in \act_\singletask\cup\act_\singleinstance$ and $B \in \Gamma_{A}$.} This arguments for this sub-situation are similar to the case $\lmd(A_0) = \STK$.
        %
        \item \emph{Sub-situation $\Aut'_1 = \Aut_{A_0\rightsquigarrow B}$ for some $B \in \Gamma_{A_0}\setminus\{A_0'\}$.}
%

    From $\theta' = \theta$ and the definition of $\Pp_{A_0}$, we know that $(p_0,S_1') \xRightarrow{\Pp_{A_0}}(p_b,S_1)$ for some $b\in\{0,1\}$, $l=k$, and $(S_2',\cdots,S_l')=(S_2,\cdots,S_k)$. 
    %
    %Let $\Aut'_1 = \Aut_{A, B}$ for some $A \in \act_\singletask$ and $B \in \act \setminus \act_\singleinstance$. 
    From  $S'_1 \in \Lang(\Aut'_1)$, we have
    %where we let $\Aut_1' = \Aut_A^B$ is a $\Pp_A$-{\NFA} for some $A \in \act_{\STK}, B\in\act$, 
    $S_1 \in \Lang((\Aut_1')^{\post^*}_{\Pp_{A_0}}(p_b)) =  \Lang((\Aut_{A_0\rightsquigarrow B})^{\post^*}_{\Pp_{A_0}}(p_b))$. 
    According to Proposition~\ref{prop-lm-A0-post}, 
    $$\Lang((\Aut_{A_0 \rightsquigarrow B})^{\post^*}_{\Pp_{A_0}}(p_0)) = \bigcup\limits_{B' \in \Gamma_{A_0}\setminus\{A_0'\}} \Lang(\Aut_{A_0\rightsquigarrow B'})$$ 
    and
    $$\Lang((\Aut_{A_0 \rightsquigarrow B})^{\post^*}_{\Pp_{A_0}}(p_1))  = \bigcup\limits_{\Lang(\Aut_{A_0\rightsquigarrow C})\neq\emptyset, C \xrightarrow{\startactivity(\bot)}A'_0 \in \Delta, B' \in \Gamma_{A_0}} \Lang(\Aut_{A_0\stackrel{C}\rightsquigarrow B'}).$$
   
   Therefore, either $S_1 \in \Lang(\Aut_{A_0 \rightsquigarrow B'})$  for some $B' \in \Gamma_{A_0}$ or 
   $S_1 \in \Lang(\Aut_{A_0 \stackrel{C}{\rightsquigarrow} B'})$ for some $C \in \Gamma_{A_0} \setminus \{A'_0\}$ and $B' \in \Gamma_{A_0}$ such that $\Lang(\Aut_{A_0\rightsquigarrow C})\neq\emptyset$ and $C \xrightarrow{\startactivity(\bot)}A'_0 \in \Delta$.
   
   From the 4th saturation rule, we know that $(\theta, (\Aut_{A_0 \rightsquigarrow B'}, \Aut'_2, \cdots, \Aut'_l)) \in \AutReach_f$ or $(\theta, (\Aut_{A_0 \stackrel{C}{\rightsquigarrow} B'}, \Aut'_2, \cdots, \Aut'_l)) \in \AutReach_f$. Since $(S_1, \cdots, S_k) \in \Lang(\Aut_{A_0 \rightsquigarrow B'}) \times \Lang(\Aut'_2) \times \cdots \times \Lang(\Aut'_l) = \Rel((\Aut_{A_0 \rightsquigarrow B'}, \Aut'_2, \cdots, \Aut'_l))$ or $(S_1, \cdots, S_k) \in \Lang(\Aut_{A_0 \stackrel{C}{\rightsquigarrow} B'}) \times \Lang(\Aut'_2) \times \cdots \times \Lang(\Aut'_l) = \Rel((\Aut_{A_0 \stackrel{C}{\rightsquigarrow} B'}, \Aut'_2, \cdots, \Aut'_l))$, we conclude that the claim holds for $n$. 
    
%    Moreover, from $\Lang(\Aut_1') = \Lang(\Aut_{A_0\rightsquigarrow B}) \subseteq \Lang((\Aut_{A_0})^{\post^*}_{\Pp_{A_0}})$, we deduce $\Lang((\Aut_1')^{\post^*}_{\Pp_{A_0}}) \subseteq \Lang((\Aut_{A_0})^{\post^*}_{\Pp_{A_0}})$. Therefore, $S_1 \in \Lang((\Aut_{A_0})^{\post^*}_{\Pp_{A_0}}(p_b))$. 
%    
%    From 
%    $ \Lang((\Aut_{A_0})^{\post^*}_{\Pp'_{A_0}}) = \Lang((\Aut_{A_0})^{\post^*}_{\Pp'_{A_0}}(p_0))\cup\Lang((\Aut_{A_0})^{\post^*}_{\Pp'_{A_0}}(p_1))$, moreover,
%     $$\Lang((\Aut_{A_0})^{\post^*}_{\Pp_{A_0}}(p_0)) = \bigcup\limits_{B'\in\Gamma_{A_0}\setminus\{A_0'\}}\Lang(\Aut_{A_0\rightsquigarrow B'}),$$
%     $$\Lang((\Aut_{A_0})^{\post^*}_{\Pp_{A_0}}(p_1)) = \bigcup\limits_{\Lang(\Aut_{A_0\rightsquigarrow C})\neq\emptyset,C \xrightarrow{\startactivity(\bot)}A'_0 \in \Delta, B' \in \Gamma_{A_0}}\Lang(\Aut_{A_0\stackrel{C}\rightsquigarrow B'}), $$
%     we know that $S_1 \in \Lang(\Aut_{A_0\rightsquigarrow B'})$ or $S_1 \in \Lang(\Aut_{A_0\stackrel{C}\rightsquigarrow B'})$ for some $B',C$.
%    Moreover, according to the aforementioned 4th saturation rule, we know that 
%    $(\theta, (\Aut_{A_0 \rightsquigarrow B'}, \Aut_2', \cdots, \Aut'_l)) \in \AutReach_f$ or
%    $(\theta, (\Aut_{A_0 \stackrel{C}{\rightsquigarrow} B'}, \Aut_2', \cdots, \Aut'_l)) \in \AutReach_f$.
%    Therefore, $(S_1, \cdots, S_k) \in \Lang(\Aut_{A_0\rightsquigarrow B'}) \times \Lang(\Aut_2') \times \cdots \times \Lang(\Aut'_l) = \Rel(\Aut_{A_0\rightsquigarrow B'}, \Aut_2', \cdots, \Aut'_l)$, or $(S_1, \cdots, S_k) \in \Lang(\Aut_{A_0\stackrel{C}\rightsquigarrow B'}) \times \Lang(\Aut_2') \times \cdots \times \Lang(\Aut'_l) = \Rel(\Aut_{A_0\stackrel{C}\rightsquigarrow B'}, \Aut_2', \cdots, \Aut'_l)$. We conclude that the claim holds for $n$. 

        \item \emph{Sub-situation $\Aut'_1 = \Aut_{A_0\stackrel{C}\rightsquigarrow B}$ for $B\in\Gamma_{A_0}$ and $C \in \Gamma_{A_0}\setminus\{A_0'\}$.}
    Then $(p_1,S_1') \xRightarrow{\Pp_{A_0}}(p_b, S_1)$ for some $b\in\{0,1\}$, $l=k$, and $(S_2',\cdots,S_l')=(S_2,\cdots,S_k)$. 
    %
    %Let $\Aut'_1 = \Aut_{A, B}$ for some $A \in \act_\singletask$ and $B \in \act \setminus \act_\singleinstance$. 
    From  $S'_1 \in \Lang(\Aut'_1)$, we have
    %where we let $\Aut_1' = \Aut_A^B$ is a $\Pp_A$-{\NFA} for some $A \in \act_{\STK}, B\in\act$, 
    $S_1 \in \Lang((\Aut_1')^{\post^*}_{\Pp'_{A_0}}(p_b))$. 
    % Moreover, from $\Lang(\Aut_1') = \Lang(\Aut_{A_0\stackrel{C}\rightsquigarrow B}) \subseteq \Lang((\Aut_{A_0})^{\post^*}_{\Pp'_{A_0}})$, we deduce $\Lang((\Aut_1')^{\post^*}_{\Pp_A}) \subseteq \Lang((\Aut_{A_0})^{\post^*}_{\Pp'_{A_0}})$. Therefore, $S_1 \in \Lang((\Aut_{A_0})^{\post^*}_{\Pp'_{A_0}})$. 
    According to Proposition~\ref{prop-lm-A0-post},  
    $$
    \begin{array}{l c l}
    \Lang((\Aut_1')^{\post^*}_{\Pp_{A_0}}(p_0)) & = & \Lang((\Aut_{A_0\stackrel{C}\rightsquigarrow B})^{\post^*}_{\Pp_{A_0}}(p_0)) =\Lang((\Aut_{A_0 \rightsquigarrow C})^{\post^*}_{\Pp_{A_0}}(p_0)) \\
    & = & \bigcup\limits_{B' \in \Gamma_{A_0}\setminus\{A_0'\}} \Lang(\Aut_{A_0\rightsquigarrow B'})
    \end{array}
    $$ 
      and  
     $$\Lang((\Aut_{A_0\stackrel{C}\rightsquigarrow B})^{\post^*}_{\Pp_{A_0}}(p_1)) = 
    \bigcup\limits_{B'\in\Gamma_{A_0}} \Lang(\Aut_{A_0\stackrel{C}\rightsquigarrow B'}) \cup  \Lang((\Aut_{A_0 \rightsquigarrow C})^{\post^*}_{\Pp_{A_0}}(p_1)).$$
 
 Therefore, 
 \begin{itemize}
	\item if $b = 0$, then  $S_1 \in \Lang(\Aut_{A_0\rightsquigarrow B'})$ for some $B'$,
	\item if $b = 1$, then $S_1 \in \Lang(\Aut_{A_0\stackrel{C}{\rightsquigarrow} B'})$ for some $B'$ or $S_1 \in \Lang(\Aut_{A_0\stackrel{C'}{\rightsquigarrow} B'})$ for some $C', B'$ such that $\Lang(\Aut_{A_0\rightsquigarrow C'}) \neq \emptyset$ and $C' \xrightarrow[]{\startactivity(\bot)} A'_0 \in \Delta$.
\end{itemize} 
  
  Let us first consider $b =0$. According to the 4th saturation rule, we know $(\theta, (\Aut_{A_0\rightsquigarrow C}, \Aut'_2, \cdots, \Aut'_l)) \in \AutReach_f$. As a result, according to the 4th saturation rule again, $(\theta, (\Aut_{A_0\rightsquigarrow B'}, \Aut'_2, \cdots, \Aut'_l)) \in \AutReach_f$. Since $(S_1, \cdots, S_k) \in \Lang(\Aut_{A_0 \rightsquigarrow B'}) \times \Lang(\Aut'_2) \times \cdots \times \Lang(\Aut'_l) = \Rel(\Aut_{A_0 \rightsquigarrow B'}, \Aut'_2, \cdots, \Aut'_l)$, we conclude that the claim holds for $n$ when $b=0$. 
  
  Then let us consider $b = 1$. 
  \begin{itemize}
  \item If $S_1 \in \Lang(\Aut_{A_0\stackrel{C}{\rightsquigarrow} B'})$, then according to the 4th saturation rule, we know $(\Aut_{A_0\stackrel{C}{\rightsquigarrow} B'}, \Aut'_2, \cdots, \Aut'_l) \in \AutReach_f$. Therefore, the claim holds for $n$ in this situation.   
   \item If $S_1 \in \Lang(\Aut_{A_0\stackrel{C'}{\rightsquigarrow} B'})$, $\Lang(\Aut_{A_0\rightsquigarrow C'}) \neq \emptyset$, and $C' \xrightarrow[]{\startactivity(\bot)} A'_0 \in \Delta$, then 
   according to the 4th saturation rule, $(\theta, (\Aut_{A_0\rightsquigarrow C}, \Aut'_2, \cdots, \Aut'_l)) \in \AutReach_f$. Moreover, according to the 4th saturation rule again, we have $(\theta, (\Aut_{A_0\rightsquigarrow C'}, \Aut'_2, \cdots, \Aut'_l)) \in \AutReach_f$ and $(\theta, (\Aut_{A_0\stackrel{C'}{\rightsquigarrow} B'}, \Aut'_2, \cdots, \Aut'_l)) \in \AutReach_f$. 
   Since $(S_1, \cdots, S_k) \in \Rel((\Aut_{A_0\stackrel{C'}{\rightsquigarrow} B'}, \Aut'_2, \cdots, \Aut'_l))$,  we conclude that the claim holds for $n$ in this situation. 
   \end{itemize}
%    we know that $S_1\in\Lang((\Aut_{A_0})^{\post^*}_{\Pp'_{A_0}})$ or $S_1\in\Lang(\Aut_{A_0\stackrel{C}\rightsquigarrow B'})$ for some $B'$. If $S_1\in\Lang((\Aut_{A_0})^{\post^*}_{\Pp'_{A_0}})$ we have proved in the second subcase, hence we consider $S_1\in\Lang(\Aut_{A_0\stackrel{C}\rightsquigarrow B'})$ for some $B'$. 
%    Furthermore, according to the aforementioned 4th saturation rule, we know  
 %   $(\theta, (\Aut_{A_0\stackrel{C}\rightsquigarrow B'}, \Aut_2', \cdots, \Aut'_l)) \in \AutReach_f$.
%    From $\rho = (S_1, \cdots, S_k) \in \Lang(\Aut_{A_0\stackrel{C}\rightsquigarrow B'}) \times \Lang(\Aut_2') \times \cdots \times \Lang(\Aut'_l)$, we know that the claim holds for $n$. 
    \end{itemize}
   
%   \zhilin{stopped here}
    
    \paragraph{Situation $\theta' \neq \theta$}
    In this case, the top task of $\rho$ is different from that of $\rho'$.  There are the following three subcases. 
    \begin{itemize}
        \item \emph{Subcase $\tau = B \xrightarrow{\startactivity(\bot)} B'$, $\lmd(B') = \STK$ or $\SIT$, and $\namefun_{B'}(\theta') = \bot$}. Then $k=l+1$, 
        $S_1'\in\Lang(\Aut_1')$ with $S_1'\cap B\act^*\neq\emptyset$, moreover,
        $S_1' \in \Lang(\Aut_{A\rightsquigarrow B})$ for some $A\in\act_{\STK}\cup\act_{\SIT}$ or $S_1'\in\Lang(\Aut_{A_0\rightsquigarrow B})$ or $S_1'\in\Lang(\Aut_{A_0\stackrel{C}\rightsquigarrow B})$ for some $C\in\Gamma_{A_0}\setminus\{A_0'\}$, 
        $S_1=[B']$, and $(S_2,\cdots,S_k)=(S_1',\cdots,S_l')$.  According to the 1st saturation rule, $(\theta, (\Aut_{B'}, \Aut_1', \cdots, \Aut_l')) \in \AutReach_f$.
        %, where $\Lang(\Aut_1') \cap A\act^* \neq \emptyset$, and $S_1=[A']\in\ConfSet(\Aut_{A'})$, 
       From $(S_1, \cdots, S_k) = ([B'], S_1', \cdots, S_l') \in \Lang(\Aut_{B'}) \times \Lang(\Aut_1') \times \cdots \times \Lang(\Aut_l')$, we conclude that the claim holds for $n$ in this subcase.
        %
        \item \emph{Subcase $\tau = B \xrightarrow{\startactivity(\bot)} B'$, $\lmd(B')=\STK$, $\namefun_{B'}(\theta') = i \neq \bot$, and $i > 1$}: Then $k = l$, 
        \begin{itemize}
            \item if $S_i'\in\Lang(\Aut_i') = \Lang(\Aut_{B'\rightsquigarrow C})$ for some $C\in\Gamma_{B'}$, 
            $S_1'\in\Lang(\Aut_1')$ with $S_1'\cap B\act^*\neq \emptyset$, moreover,
            $S_1' \in \Lang(\Aut_{A\rightsquigarrow B})$ for some $A\in\act_{\STK}\cup\act_{\SIT}$ or $S_1'\in\Lang(\Aut_{A_0\rightsquigarrow B})$ or $S_1'\in\Lang(\Aut_{A_0\stackrel{D}\rightsquigarrow B})$ for some $D\in\Gamma_{A_0}\setminus\{A_0'\}$, 
            $S_1 = [B']$, $(S_2, \dots, S_k) = (S'_1, \dots, S'_{i-1}, S'_{i+1}, \dots, S'_l)$. According to the 2nd saturation rule, we know that 
            $$(\theta, (\Aut_{B'}, \Aut_1',\cdots, \Aut_{i-1}', \Aut_{i+1}', \cdots, \Aut_{l}')) \in \AutReach_f.$$
            From 
            $$
            \begin{array}{l}
                (S_1,\cdots,S_k) = ([B'], S_1', \cdots, S_{i-1}', S_{i+1}', \cdots, S_l') \in \\
                \ \ \Lang(\Aut_{B'}) \times \Lang(\Aut_1') \times \cdots \times \Lang(\Aut_{i-1}') \times \Lang(\Aut_{i+1}')\times \cdots \times \Lang(\Aut_{l}'),
            \end{array}
            $$  
            we conclude that the claim holds for $n$ in this subcase. 
            \item if $S_i'\in\Lang(\Aut_i') = \Lang(\Aut_{A_0\rightsquigarrow C})$ for some $C\in\Gamma_{A_0}\setminus\{A_0'\}$, moreover $B' = A_0'$, $A_0'\notin S_i'$, $S_1' \in \Lang(\Aut_{A\rightsquigarrow B})$ for some $A\in\act_\STK\cup\act_\SIT$, $S_1 = [A_0']\cdot S_i'$, $(S_2, \dots, S_k) = (S'_1, \dots, S'_{i-1}, S'_{i+1}, \dots, S'_l)$. According to the 2nd saturation rule, we know that 
            $$(\theta, (\Aut_{A_0\stackrel{C}\rightsquigarrow A_0'}, \Aut_1',\cdots, \Aut_{i-1}', \Aut_{i+1}', \cdots, \Aut_{l}')) \in \AutReach_f.$$
            From $\Lang(\Aut_{A_0\stackrel{C}\rightsquigarrow A_0'}) = \{[A_0']\cdot S\mid S \in \Lang(\Aut_{A_0\rightsquigarrow C})\}$, we know that $[A_0']\cdot S_i'\in \Lang(\Aut_{A_0\stackrel{C}\rightsquigarrow A_0'})$. From
            % $S_i'\in\Lang(\Aut_i') = \Lang(\Aut_{A_0\rightsquigarrow C})$ and $\Lang(\Aut_{A_0\rightsquigarrow C}) = C\act^*\cap \Lang((\Aut_{A_0})^{\post^*}_{\Pp'_{A_0}})$, we let $S_i' = [C]\cdot S$, hence we know that there is some $p\in P_{A_0}'$,
            % $$p_0\xRightarrow[\Aut_{A_0\rightsquigarrow C}]{C}p\xRightarrow[\Aut_{A_0\rightsquigarrow C}]{S}p_f.$$
            % From the definition of $\Aut_{A_0\stackrel{C}\rightsquigarrow A_0'}$, we know that,
            % $$p_1\xrightarrow[\Aut_{A_0\stackrel{C}\rightsquigarrow A_0'}]{A_0'}\langle p_1,A_0'\rangle \xRightarrow[\Aut_{A_0\stackrel{C}\rightsquigarrow A_0'}]{C}p\xRightarrow[\Aut_{A_0\stackrel{C}\rightsquigarrow A_0'}]{S}p_f,$$
            % hence we have $[A_0'C]\cdot S = [A_0']\cdot S_i'\in\Lang(\Aut_{A_0\stackrel{C}\rightsquigarrow A_0'})$, therefore,
            % From $\Lang(\Aut_{A_0\stackrel{C}\rightsquigarrow A_0'}) = \{[A_0']\cdot S\mid S\in\Lang(\Aut_{A_0\rightsquigarrow C})\}$, 
            % we know that $S_1 = [A_0']\cdot S_i' \in\Lang(\Aut_{A_0\stackrel{C}\rightsquigarrow A_0'})$, therefore,
            $$
            \begin{array}{l}
                (S_1,\cdots,S_k) = ([A_0']\cdot S_i', S_1', \cdots, S_{i-1}', S_{i+1}', \cdots, S_l') \in \\
                \ \ \Lang(\Aut_{A_0\stackrel{C}\rightsquigarrow A_0'}) \times \Lang(\Aut_1') \times \cdots \times \Lang(\Aut_{i-1}') \times \Lang(\Aut_{i+1}')\times \cdots \times \Lang(\Aut_{l}'),
            \end{array}
            $$  
            we conclude that the claim holds for $n$ in this subcase. 
            \item if $S_i' \in \Lang(\Aut_i') = \Lang(\Aut_{A_0\stackrel{C}\rightsquigarrow D})$ for some $C\in\Gamma_{A_0}\setminus\{A_0'\}$, $D\in\Gamma_{A_0}$, moreover $B' = A_0'$, $S_i' = S\cdot [A_0]\cdot S'$ for some $S,S'\in\Gamma_{A_0}^*$ with $A_0'\notin S$, $S_1' \in \Lang(\Aut_{A\rightsquigarrow B})$ for some $A\in\act_\STK\cup\act_\SIT$, $S_1 = [A_0]\cdot S'$, $(S_2, \dots, S_k) = (S'_1, \dots, S'_{i-1}, S'_{i+1}, \dots, S'_l)$. According to the 2nd saturation rule, we know that 
            $$(\theta, (\Aut_{A_0\stackrel{C}\rightsquigarrow A_0'}, \Aut_1',\cdots, \Aut_{i-1}', \Aut_{i+1}', \cdots, \Aut_{l}')) \in \AutReach_f.$$
            From 
            $\Lang(\Aut_{A_0\stackrel{C}\rightsquigarrow A_0'}) = \{[A_0']\cdot S\mid S'\cdot[A_0']\cdot S\in\Lang(\Aut_{A_0\stackrel{C}\rightsquigarrow D})\}$,
            we know that $[A_0']\cdot S'\in\Lang(\Aut_{A_0\stackrel{C}\rightsquigarrow A_0'})$. From
            % From $\Lang(\Aut_{A_0\stackrel{C}\rightsquigarrow D}) = D\act^*A_0'C\act^* \cap \Lang((\Aut_{A_0\stackrel{C}\rightsquigarrow A_0'})^{\post^*}_{\Pp'_{A_0}})$, we know that $(p_1,[A_0'C]\cdot S')\xRightarrow{\Pp'_{A_0}}(p_1, [D]\cdot S\cdot [A_0'C]\cdot S')$,
            % From $\Lang(\Aut_{A_0\stackrel{C}\rightsquigarrow A_0'}) = \{[A_0'C]\cdot S'\mid [D]\cdot S\cdot[A_0'C]\cdot S'\in\Lang(\Aut_{A_0\stackrel{C}\rightsquigarrow D})\}$, we know that $S_1 = [A_0'C]\cdot S'\in\Lang(\Aut_{A_0\stackrel{C}\rightsquigarrow A_0'})$, therefore,
            $$
            \begin{array}{l}
                (S_1,\cdots,S_k) = ([A_0']\cdot S', S_1', \cdots, S_{i-1}', S_{i+1}', \cdots, S_l') \in \\
                \ \ \Lang(\Aut_{A_0\stackrel{C}\rightsquigarrow A_0'}) \times \Lang(\Aut_1') \times \cdots \times \Lang(\Aut_{i-1}') \times \Lang(\Aut_{i+1}')\times \cdots \times \Lang(\Aut_{l}'),
            \end{array}
            $$  
            we conclude that the claim holds for $n$ in this subcase. 
        \end{itemize}
    %
        \item \emph{Subcase $\tau = \back$ and $|S_1'|=1$}. Then $k = l - 1$, and $(S_1,\cdots,S_k) = (S_2',\cdots,S_l')$.  According to the 3rd saturation rule, $(\theta, (\Aut_2',\cdots,\Aut_l')) \in \AutReach_f$. From $(S_1,\dots,S_k) = (S'_2, \cdots, S'_l) \in \Lang(\Aut_2') \times \cdots \times \Lang(\Aut_l')$, we conclude that the claim holds for $n$ in this subcase. 
    \end{itemize}
}
%%%%%%%%%%%%%%%%% the original proof removed
%%%%%%%%%%%%%%%%% the original proof removed
%
\qed
    \end{proof}


    \paragraph*{The proof of $\bigcup \limits_{(\theta, (\Aut_1, \cdots, \Aut_k)) \in \AutReach_f}  \Rel((\Aut_1, \cdots, \Aut_k)) \subseteq \RRel(\Mm, \theta)$}

%\begin{lemma}\label{lem:iff-backward}
%    Let $\AutReach$ be the set computed by the aforementioned saturation procedure. Then for each $\theta = \aname_1 \dots \aname_k \in \Theta_\Mm$, we have 
%    $$\bigcup \limits_{(\theta, (\Aut_1, \dots, \Aut_k)) \in \AutReach} \ConfSet(\Aut_1) \times \dots \times \ConfSet(\Aut_k) \subseteq \conf_\theta.$$
%\end{lemma}

\begin{proof}
Since $\AutReach_f$ is computed by applying the saturation rules and adding the tuples into $\AutReach$,  let us use $\AutReach_0$ to denote $\{(\aft(A_0), \Aut_{A_0\rightsquigarrow A_0})\}$, use $\AutReach_1$ to denote the set obtained by adding a tuple to $\AutReach_0$, and for $n \ge 2$, use $\AutReach_n$ to denote the set obtained by adding a new tuple to $\AutReach_{n-1}$. 
We prove by an induction on $n \ge 0$ the following claim.  

\smallskip
\noindent {\bf Claim}. For each $n \ge 0$, if $(\theta, (\Aut_1, \cdots, \Aut_k)) \in \AutReach_n$ with $\theta = \aname_1 \cdots \aname_k$ and $(S_1,\dots,S_k) \in \Rel((\Aut_1, \cdots, \Aut_k))$, then $$(([A_0], \aft(A_0))) \xRightarrow[\Mm]{} ((S_1, \aname_1), \cdots, (S_k, \aname_k)).$$

\smallskip

%    Let $\AutReach_n$ be the set computed by the aforementioned saturation procedure after the $n$-th tuple $(\theta,(\Aut_1,\dots,\Aut_k))$ is added, 


\noindent \emph{Induction base $n = 0$}. 
Because $\AutReach_0 = \{(\aft(A_0),(\Aut_{A_0\rightsquigarrow A_0}))\}$, if $(S_1, \cdots, S_k) \in \Rel((\Aut_{A_0\rightsquigarrow A_0}))$, then $k=1$ and $S_1 \in\Lang(\Aut_{A_0\rightsquigarrow A_0})$. 

From $\Lang(\Aut_{A_0\rightsquigarrow A_0}) \subseteq \Lang((\Aut_{A_0})^{\post^*}_{\Pp_{A_0}}(p_0))$, we know $S_1\in \Lang((\Aut_{A_0})^{\post^*}_{\Pp_{A_0}}(p_0))$. Therefore, $(p_0,[A_0])\xRightarrow{\Pp_{A_0}}(p_0,S_1)$. From the definition of $\Pp_{A_0}$, we have $(([A_0], \aft(A_0))) \xRightarrow[\Mm]{} ((S_1, \aft(A_0)))$. Thus, the claim holds for $n = 0$. 

\smallskip

\noindent \emph{Induction step $n > 0$}. 
Suppose that $(\theta, (\Aut_1, \cdots, \Aut_k)) \in \AutReach_n$ with $\theta = \aname_1 \cdots \aname_k$ and $(S_1,\dots,S_k) \in \Rel((\Aut_1, \cdots, \Aut_k))$. 

If $(\theta, (\Aut_1, \cdots, \Aut_k)) \in \AutReach_{n-1}$, then the claim follows directly from the induction hypothesis. 

Next, let us assume that $(\theta, (\Aut_1, \cdots, \Aut_k)) \not \in  \AutReach_{n-1}$.
Therefore, $\AutReach_{n} = \AutReach_{n-1} \cup \{(\theta,(\Aut_1,\dots,\Aut_k))\}$.  
%
%Then the $n$-th tuple $(\theta,(\Aut_1,\dots,\Aut_k))$ is added which obtained from $(\theta',(\Aut_1',\dots,\Aut_l'))\in \AutReach_{n-1}$.
We distinguish which saturation rule is used to add $(\theta,(\Aut_1,\dots,\Aut_k))$.


\paragraph*{The 1st saturation rule} Then  there are a transition rule $\tau = A \xrightarrow{\startactivity(\bot)} A'  \in \Delta$ and $(\theta', (\Aut'_1, \cdots, \Aut'_l)) \in \AutReach_{n-1}$ such that $\lmd(A')=\STK$ or $\singleinstance$, $\Aut'_1 = \Aut_{B \rightsquigarrow A}$ or $\Aut_{B \stackrel{C}{\rightsquigarrow} A}$ for some $B$ and $C$,
% which implies that $\Aut'_1 = \Aut_{B\rightsquigarrow A}$ for some $B \in \act_\singletask\cup\act_\singleinstance$ or $\Aut'_1 = \Aut_{A_0\rightsquigarrow A}$ or $\Aut'_1 = \Aut_{A_0\stackrel{C}\rightsquigarrow A}$ for some $C\in\Gamma_{A_0}\setminus\{A_0'\}$, 
$\namefun_{A'}(\theta') = \bot$, $\theta = \aft(A') \theta'$, and $(\Aut_1, \cdots, \Aut_k) = (\Aut_{A'}, \Aut'_1, \cdots, \Aut'_l)$. 
Evidently, $\aname_1 = \aft(A')$ and $\theta' = \aname_2 \cdots \aname_k$. 

Let $(S_1,\dots,S_k) \in \Rel((\Aut_1, \cdots, \Aut_k))$. Then $S_1 = [A']$ and $(S_2, \cdots, S_k) \in \Rel((\Aut'_1, \cdots, \Aut'_l))$. 

From $(\theta', (\Aut'_1, \cdots, \Aut'_l)) \in \AutReach_{n-1}$ and the induction hypothesis, we know that $(([A_0], \aft(A_0))) \xRightarrow[\Mm]{} ((S_2, \aname_2), \cdots, (S_k, \aname_k))$. 

Because $S_2 \in \Lang(\Aut'_1)$ and $\Aut'_1 = \Aut_{B \rightsquigarrow A}$ or $\Aut_{B \stackrel{C}{\rightsquigarrow} A}$, we know that $S_2 = [A] \cdot S'_2$ for some $S'_2$.  
Since $A \xrightarrow{\startactivity(\bot)}A'  \in \Delta$, $\namefun_{A'}(\theta') = \bot$, we deduce that 
$$((S_2, \aname_2), \cdots, (S_k, \aname_k)) \xrightarrow[\Mm]{} (([A'], \aft(A')), (S_2, \aname_2), \cdots, (S_k, \aname_k)).$$
Therefore, 
$$(([A_0], \aft(A_0))) \xRightarrow[\Mm]{} (([A'], \aft(A')), (S_2, \aname_2), \cdots, (S_k, \aname_k)).$$ 
%
From $(([A'], \aft(A')), (S_2, \aname_2), \cdots, (S_k, \aname_k)) = ((S_1, \aname_1), (S_2, \aname_2), \cdots, (S_k, \aname_k))$, we deduce that  
$$(([A_0], \aft(A_0))) \xRightarrow[\Mm]{} ((S_1, \aname_1), (S_2, \aname_2), \cdots, (S_k, \aname_k)).$$ 
Thus, the claim holds for $n$ in this case. 

% From $\Aut'_1 = \Aut_{B\rightsquigarrow A}$ or $\Aut'_1 = \Aut_{A_0\rightsquigarrow A}$ or $\Aut'_1 = \Aut_{A_0\stackrel{C}\rightsquigarrow A}$, 
%From $\Lang(\Aut'_1)\cap A\act^* \neq \emptyset$,
%$S_2 \in \Lang(\Aut'_1)$, $A \xrightarrow{\startactivity(\bot)}A'  \in \Delta$, $\namefun_{A'}(\theta') = \bot$, we know that 
%$$((S_2, \aname_2), \cdots, (S_k, \aname_k)) \xrightarrow[\Mm]{} (([A'], \aft(A')), (S_2, \aname_2), \cdots, (S_k, \aname_k)).$$
%Therefore, 
%$(([A_0], \aft(A_0))) \xRightarrow[\Mm]{} (([A'], \aft(A')), (S_2, \aname_2), \cdots, (S_k, \aname_k))$. 
%The claim holds for $n$ in this case. 

%$A \xrightarrow{\startactivity(\bot)}A'  \in \Delta$, $\lmd(A')=\STK$, $\namefun_{A'}(\theta') = \bot$ $(\theta,(\Aut_1,\dots,\Aut_k))$ is obtained from $(\theta',(\Aut_1',\dots,\Aut_l'))$ by the first saturation rule [$A\xrightarrow{\startactivity(\bot)}A' \in\Delta$ with $\lmd(A')=\STK$, $\namefun_{A'}(\theta') = \bot$]} :

\paragraph*{The 2nd saturation rule} Then there are a transition rule $\tau = A \xrightarrow{\startactivity(\bot)} A'  \in \Delta$ and $(\theta', (\Aut'_1, \cdots, \Aut'_l)) \in \AutReach_{n-1}$ such that $\lmd(A')=\STK$ or $\singleinstance$, $\Aut'_1 = \Aut_{B \rightsquigarrow A}$ or $\Aut_{B \stackrel{C}{\rightsquigarrow} A}$ for some $B$ and $C$, $\namefun_{A'}(\theta') = i > 1$, 
$\theta' = \aname_2  \cdots  \aname_{i} \aname_1  \aname_{i+1}  \cdots  \aname_k$.
Evidently, $k = l$. There are the following two situations. 

Let us first consider the situation $\Aut_i' = \Aut_{A'\rightsquigarrow B'}$ for some $B' \in \Gamma_{A'}$. 
Then 
%
$$(\Aut_1, \cdots, \Aut_k) = (\Aut_{A'}, \Aut'_1, \cdots, \Aut'_{i-1}, \Aut'_{i+1}, \cdots, \Aut'_l).$$  

From 
%
$$(S_1, \cdots, S_k) \in \Rel((\Aut_1, \cdots, \Aut_k)) = \Rel((\Aut_{A'}, \Aut'_1, \cdots, \Aut'_{i-1}, \Aut'_{i+1}, \cdots, \Aut'_k)),$$ 
%
we know $S_1 = [A']$, $(S_2, \cdots, S_i) \in \Rel((\Aut'_1, \cdots, \Aut'_{i-1}))$, and $(S_{i+1}, \cdots, S_k) \in \Rel((\Aut'_{i+1}, \cdots, \Aut'_k))$. 

Let $S' \in \Lang(\Aut'_i)$. Then $(S_2, \cdots, S_i, S', S_{i+1}, \cdots, S_k) \in \Rel(\Aut'_1, \cdots, \Aut'_k)$.
From $(\theta', (\Aut'_1, \cdots, \Aut'_k)) \in \AutReach_{n-1}$ and the induction hypothesis, we know that  
%
$$(([A_0], \aft(A_0))) \xRightarrow[\Mm]{} ((S_2, \aname_2), \cdots, (S_i, \aname_i), (S', \aname_1), (S_{i+1}, \aname_{i+1}), \cdots, (S_k, \aname_k)).$$ 

Because $\Aut'_1 = \Aut_{B \rightsquigarrow A}$ or $\Aut_{B \stackrel{C}{\rightsquigarrow} A}$, $S_2 \in \Lang(\Aut'_1)$,  $\tau = A \xrightarrow{\startactivity(\bot)} A'$, and $\namefun_{A'}(\theta') = i > 1$, 
%$\theta' = \aname_2  \cdots  \aname_{i} \aname_1  \aname_{i+1}  \cdots  \aname_k$, 
we deduce that 
$$
\begin{array}{l}
((S_2, \aname_2), \cdots, (S_i, \aname_i), (S', \aname_1), (S_{i+1}, \aname_{i+1}), \cdots, (S_k, \aname_k)) \xrightarrow[\Mm]{} \\
(([A'], \aname_1), (S_2, \aname_2), \cdots, (S_i, \aname_i), (S_{i+1}, \aname_{i+1}), \cdots, (S_k, \aname_k)).
\end{array}
$$ 

Since $S_1 = [A']$, we have
$(([A_0], \aft(A_0))) \xRightarrow[\Mm]{} ((S_1, \aname_1), \cdots, (S_k, \aname_k)).$
%
Therefore, the claim holds for $n$ in this situation.


Let us consider the situation $\Aut_i' = \Aut_{A_0 \rightsquigarrow B'}$ or $\Aut_{A_0 \stackrel{B'}{\rightsquigarrow} C'}$ for some $B' \in\Gamma_{A_0}\setminus\{A_0'\}$ and $C' \in \Gamma_{A_0}$. 

Then $A' = A_0'$ and
$$(\Aut_1, \cdots, \Aut_k) = (\Aut_{A_0\stackrel{B'}{\rightsquigarrow} A_0'}, \Aut'_1, \cdots, \Aut'_{i-1}, \Aut'_{i+1}, \cdots, \Aut'_l).$$ 
From 
%
$$(S_1, \cdots, S_k) \in \Rel((\Aut_1, \cdots, \Aut_k)) = \Rel((\Aut_{A_0\stackrel{B'}{\rightsquigarrow} A_0'}, \Aut'_1, \cdots, \Aut'_{i-1}, \Aut'_{i+1}, \cdots, \Aut'_k)),$$ 
we know that $S_1 \in \Lang(\Aut_{A_0\stackrel{B'}{\rightsquigarrow} A_0'})$, $(S_2, \cdots, S_{i}) \in \Rel((\Aut'_1, \cdots, \Aut'_{i-1}))$ and $(S_{i+1}, \cdots, S_k) \in \Rel((\Aut'_{i+1}, \cdots, \Aut'_k))$. 

Let us define a string $S'_1 \in \Lang(\Aut'_i)$ as follows. 
\begin{itemize} 
\item If $\Aut_i' = \Aut_{A_0 \rightsquigarrow B'}$, then $S'_1$ is defined as the string in $\Lang(\Aut_{A_0 \rightsquigarrow B'})  = \Lang(\Aut'_i)$ such that $S_1 = [A'_0] \cdot S'_1$.  Such a string exists since $S_1 \in \Lang(\Aut_{A_0\stackrel{B'}{\rightsquigarrow} A_0'})$. 
%we know that $S_1 = [A'_0] \cdot S'_1$ for some $S'_1 \in \Lang(\Aut_{A_0 \rightsquigarrow B'}) = \Lang(\Aut'_i)$. 
%
\item If $\Aut'_i = \Aut_{A_0 \stackrel{B'}{\rightsquigarrow} C'}$, then from $S_1 \in \Lang(\Aut_{A_0\stackrel{B'}{\rightsquigarrow} A_0'})$, we know that there is $S''$ such that $S'' S_1 \in \Lang(\Aut_{A_0 \stackrel{B'}{\rightsquigarrow} C'}) = \Lang(\Aut'_i)$. Let $S'_1 = S''S_1$ in this case. 
\end{itemize}

Then we have  $(S_2, \cdots, S_i, S'_1, S_{i+1}, \cdots, S_k) \in \Rel((\Aut'_1, \cdots, \Aut'_k))$.
From the induction hypothesis, 
$$(([A_0], \aft(A_0))) \xRightarrow[\Mm]{} ((S_2, \aname_2), \cdots, (S_i, \aname_i), (S'_1, \aname_1), (S_{i+1}, \aname_{i+1}), \cdots, (S_k, \aname_k)).$$ 
%
Because $S_2 \in \Lang(\Aut'_1)$, $\Aut'_1 = \Aut_{B \rightsquigarrow A}$ or $\Aut_{B \stackrel{C}{\rightsquigarrow} A}$, $\tau = A \xrightarrow{\startactivity(\bot)} A'  \in \Delta$, $\namefun_{A'}(\theta') = i > 1$, we deduce that 
$$
\begin{array}{l}
((S_2, \aname_2), \cdots, (S_i, \aname_i), (S'_1, \aname_1), (S_{i+1}, \aname_{i+1}), \cdots, (S_k, \aname_k)) \xrightarrow[\Mm]{} \\
((S_1, \aname_1), (S_2, \aname_2), \cdots, (S_i, \aname_i), (S_{i+1}, \aname_{i+1}), \cdots, (S_k, \aname_k)).
\end{array}
$$
Therefore, 
$$
\begin{array}{l}
(([A_0], \aft(A_0))) \xRightarrow[\Mm]{} ((S_1, \aname_1), \cdots, (S_k, \aname_k)).
\end{array}
$$
We conclude that the claim holds for $n$ in this situation.

\paragraph*{The 3rd saturation rule} In this situation, there are $\aname'$ and $\Aut'_1$ such that $(\aname' \theta, (\Aut'_1, \Aut_1, \cdots, \Aut_k)) \in \AutReach_{n-1}$.
Let $S' \in \Lang(\Aut'_1)$. Then $(S', S_1, \cdots, S_k) \in \Rel((\Aut'_1, \Aut_1, \cdots, \Aut_k))$. From the induction hypothesis, 
\[([A_0], \aft(A_0)) \xRightarrow[\Mm]{} ((S', \aname'), (S_1, \aname_1), \cdots, (S_k, \aname_k)). \]
Evidently, in $\Mm$, the configuration $((S_1, \aname_1), \cdots, (S_k, \aname_k))$ can be reached from the configuration $((S', \aname'), (S_1, \aname_1), \cdots, (S_k, \aname_k))$ by repeatedly applying the $\back$ action. Therefore, 
\[ ((S', \aname'), (S_1, \aname_1), \cdots, (S_k, \aname_k))  \xRightarrow[\Mm]{} ((S_1, \aname_1), \cdots, (S_k, \aname_k)). \]

We deduce that 
 \[([A_0], \aft(A_0)) \xRightarrow[\Mm]{} ((S_1, \aname_1), \cdots, (S_k, \aname_k)). \]
 The claim holds for $n$ in this case. 
 
\paragraph*{The 4th saturation rule} Then $(\Aut'_1, \Aut_2, \cdots, \Aut_k) \in \AutReach_{n-1}$ for some $\Aut'_1$ satisfying one of the following conditions. 
\begin{itemize}
\item $\Aut_1 = \Aut_{A \rightsquigarrow B}$ and $\Aut'_1 = \Aut_{A \rightsquigarrow B'}$ for $A \in \act_\singletask \cup \act_\singleinstance$ and $B, B' \in \Gamma_A$. 
%
\item $\Aut_1 = \Aut_{A_0 \rightsquigarrow B}$ and $\Aut'_1 = \Aut_{A_0 \rightsquigarrow B'}$ for $B, B' \in \Gamma_{A_0} \setminus \{A'_0\}$.  
%
\item $\Aut_1 = \Aut_{A_0 \stackrel{B}{\rightsquigarrow} C}$, $\Aut'_1 = \Aut_{A_0 \rightsquigarrow B'}$, and $B \xrightarrow{\startactivity(\bot)} A'_0 \in \Delta$ for $B, B' \in \Gamma_{A_0} \setminus \{A'_0\}$ and $C \in \Gamma_{A_0}$. 
%
\item $\Aut_1 = \Aut_{A_0 \rightsquigarrow B}$ and $\Aut'_1 = \Aut_{A_0 \stackrel{B}{\rightsquigarrow} C}$ for $B \in \Gamma_{A_0} \setminus \{A'_0\}$ and $C \in \Gamma_{A_0}$. 
%
\item $\Aut_1 = \Aut_{A_0 \stackrel{B}{\rightsquigarrow} C}$ and $\Aut'_1 = \Aut_{A_0 \stackrel{B}{\rightsquigarrow} C'}$ for $B \in \Gamma_{A_0} \setminus \{A'_0\}$ and $C, C' \in \Gamma_{A_0}$.
\end{itemize}

For each of the aforementioned conditions, 
we can show that there is $S'_1 \in \Lang(\Aut'_1)$ such that $(p_{b'}, S'_1) \xRightarrow{\Mm} (p_b, S_1)$ for some $b, b' \in \{0, 1\}$.  
It follows that $(S'_1, S_2, \cdots, S_k) \in \Rel((\Aut'_1, \Aut_2, \cdots, \Aut_k))$ and
 \[((S'_1, \aname_1), (S_2, \aname_2), \cdots, (S_k, \aname_k)) \xRightarrow[\Mm]{} ((S_1, \aname_1), (S_2, \aname_2), \cdots, (S_k, \aname_k)). \]
 % 
Then according to the induction hypothesis, 
 \[([A_0], \aft(A_0)) \xRightarrow[\Mm]{} ((S'_1, \aname_1), (S_2, \aname_2), \cdots, (S_k, \aname_k)). \]
It follows that  
 \[([A_0], \aft(A_0)) \xRightarrow[\Mm]{} ((S_1, \aname_1), (S_2, \aname_2), \cdots, (S_k, \aname_k)). \]
We conclude that the claim holds for $n$ in this situation.  

It remains to show that such a string $S'_1$ exists. 
\begin{itemize}
\item If $\Aut_1 = \Aut_{A \rightsquigarrow B}$ and $\Aut'_1 = \Aut_{A \rightsquigarrow B'}$, then $(p_0, S'_1) \xRightarrow{\Pp_A} (p_0, S_1)$ for some $S'_1 \in \Lang(\Aut'_1)$. 
%
\item The arguments for the second condition are similar. 
%
\item If $\Aut_1 = \Aut_{A_0 \stackrel{B}{\rightsquigarrow} C}$, $\Aut'_1 = \Aut_{A_0 \rightsquigarrow B'}$, and $B \xrightarrow{\startactivity(\bot)} A'_0 \in \Delta$, then $(p_0, S'_1) \xRightarrow{\Pp_{A_0}} (p_0, S''_1) \xRightarrow{\Pp_{A_0}} (p_1, S_1)$ for some $S'_1 \in \Lang(\Aut'_1)$ and $S''_1 \in \Lang(\Aut_{A_0 \rightsquigarrow B})$. 
%
\item If $\Aut_1 = \Aut_{A_0 \rightsquigarrow B}$ and $\Aut'_1 = \Aut_{A_0 \stackrel{B}{\rightsquigarrow} C}$, then $(p_1, S'_1) \xRightarrow{\Pp_{A_0}}  (p_0, S_1)$ for some $S'_1 \in \Lang(\Aut'_1)$. 
%
\item If $\Aut_1 = \Aut_{A_0 \stackrel{B}{\rightsquigarrow} C}$ and $\Aut'_1 = \Aut_{A_0 \stackrel{B}{\rightsquigarrow} C'}$, then $(p_1, S'_1) \xRightarrow{\Pp_{A_0}}  (p_1, S_1)$ for some $S'_1 \in \Lang(\Aut'_1)$. 
\end{itemize}

 
 
 %%%%%%%%% the original texts removed
  %%%%%%%%% the original texts removed
 \hide{
Therefore, $(S'_1, S_2, \cdots, S_k) \in \Rel((\Aut_{A\rightsquigarrow B'}, \Aut_2, \cdots, \Aut_k))$ for some $S'_1 \in \Lang(\Aut_{A\rightsquigarrow B'})$. 

Then according to the induction hypothesis, 
$$(([A_0], \aft(A_0))) \xRightarrow[\Mm]{} ((S'_1, \aname_1), (S_2, \aname_2), \cdots, (S_k, \aname_k)).$$

From $S'_1 \in \Lang(\Aut_{A\rightsquigarrow B'})$, $S_1 \in \Lang(\Aut_{A\rightsquigarrow B})$, and the definition of $\Pp_A$, we know that $(p_0, S'_1) \xRightarrow{\Pp_A} (p_0, [A]) \xRightarrow{\Pp_A} (p_0, S_1)$. 

Therefore, 
$$((S'_1, \aname_1), (S_2, \aname_2), \cdots, (S_k, \aname_k)) \xRightarrow[\Mm]{} ((S_1, \aname_1), (S_2, \aname_2), \cdots, (S_k, \aname_k)).$$
It follows that 
$$(([A_0], \aft(A_0))) \xRightarrow[\Mm]{} ((S_1, \aname_1), (S_2, \aname_2), \cdots, (S_k, \aname_k)).$$
We conclude that the claim holds for $n$ in this case. 
    \item if $\Aut_1' = \Aut_{A_0\rightsquigarrow B'}$ for some $B'\in\Gamma_{A_0}\setminus\{A_0'\}$, then $\Aut_1 = \Aut_{A_0\rightsquigarrow B}$ for some $B\in\Gamma_{A_0}\setminus\{A_0'\}$.

Therefore, $(S'_1, S_2, \cdots, S_k) \in \Rel((\Aut_{A_0\rightsquigarrow B'}, \Aut_2, \cdots, \Aut_k))$ for some $S'_1 \in \Lang(\Aut_{A_0\rightsquigarrow B'})$. 

Then according to the induction hypothesis, 
$$(([A_0], \aft(A_0))) \xRightarrow[\Mm]{} ((S'_1, \aname_1), (S_2, \aname_2), \cdots, (S_k, \aname_k)).$$

From $S'_1 \in \Lang(\Aut_{A_0\rightsquigarrow B'})$, $S_1 \in \Lang(\Aut_{A_0\rightsquigarrow B})$, and the definition of $\Pp'_{A_0}$, we know that $(p_0, S'_1) \xRightarrow{\Pp'_{A_0}} (p_0, [A_0]) \xRightarrow{\Pp'_{A_0}} (p_0, S_1)$. 

Therefore, 
$$((S'_1, \aname_1), (S_2, \aname_2), \cdots, (S_k, \aname_k)) \xRightarrow[\Mm]{} ((S_1, \aname_1), (S_2, \aname_2), \cdots, (S_k, \aname_k)).$$
It follows that 
$$(([A_0], \aft(A_0))) \xRightarrow[\Mm]{} ((S_1, \aname_1), (S_2, \aname_2), \cdots, (S_k, \aname_k)).$$
We conclude that the claim holds for $n$ in this case. 

Moreover if $B'\xrightarrow{\startactivity(\bot)}A_0'$, then $\Aut_1 = \Aut_{A_0\stackrel{B'}\rightsquigarrow A_0'}$.

From $\Lang(\Aut_{A_0\stackrel{B'}\rightsquigarrow A_0'}) = \{[A_0']\cdot S\mid S \in \Lang(\Aut_{A_0\rightsquigarrow B'})\}$, $S'_1 \in \Lang(\Aut_{A_0\rightsquigarrow B'})$ with $S'_1\cap B'\act^* \neq \emptyset$, $S_1 = [A_0']\cdot S'_1 \in \Lang(\Aut_{A_0\stackrel{B'}\rightsquigarrow A_0'})$ and $B'\xrightarrow{\startactivity(\bot)}A_0'$, we deduce that
% and the definition of $\Pp'_{A_0}$, we know that $(p_0, S'_1) \xRightarrow{\Pp'_{A_0}} (p_0, S_1)$. 
% Therefore, 
$$((S'_1, \aname_1), (S_2, \aname_2), \cdots, (S_k, \aname_k)) \xrightarrow[\Mm]{} ((S_1, \aname_1), (S_2, \aname_2), \cdots, (S_k, \aname_k)).$$
It follows that 
$$(([A_0], \aft(A_0))) \xRightarrow[\Mm]{} ((S_1, \aname_1), (S_2, \aname_2), \cdots, (S_k, \aname_k)).$$
We conclude that the claim holds for $n$ in this case. 
    \item if $\Aut_1' = \Aut_{A_0\stackrel{C}\rightsquigarrow A_0'}$ for some $C\in\Gamma_{A_0}\setminus\{A_0'\}$, then $\Aut_1 = \Aut_{A_0\rightsquigarrow C}$.

Therefore, $(S'_1, S_2, \cdots, S_k) \in \Rel((\Aut_{A_0\stackrel{C}\rightsquigarrow A_0'}, \Aut_2, \cdots, \Aut_k))$ for some $S'_1 \in \Lang(\Aut_{A_0\stackrel{C}\rightsquigarrow A_0'})$. 

Then according to the induction hypothesis, 
$$(([A_0], \aft(A_0))) \xRightarrow[\Mm]{} ((S'_1, \aname_1), (S_2, \aname_2), \cdots, (S_k, \aname_k)).$$
From $\Lang(\Aut_{A_0\stackrel{C}\rightsquigarrow A_0'}) = \{[A_0']\cdot S\mid S \in \Lang(\Aut_{A_0\rightsquigarrow C})\}$,  $S_1\in\Lang(\Aut_{A_0\rightsquigarrow C})$, $S'_1 = [A_0']\cdot S_1\in\Lang(\Aut_{A_0\stackrel{C}\rightsquigarrow A_0'})$, we deduce that
$$((S'_1, \aname_1), (S_2, \aname_2), \cdots, (S_k, \aname_k)) \xrightarrow[\Mm]{} ((S_1, \aname_1), (S_2, \aname_2), \cdots, (S_k, \aname_k)).$$
It follows that 
$$(([A_0], \aft(A_0))) \xRightarrow[\Mm]{} ((S_1, \aname_1), (S_2, \aname_2), \cdots, (S_k, \aname_k)).$$
We conclude that the claim holds for $n$ in this case. 

Moreover, if $B\in\Gamma_{A_0}$ then $\Aut_1 = \Aut_{A_0\stackrel{C}\rightsquigarrow B}$.

From $S'_1 \in \Lang(\Aut_{A_0\stackrel{C}\rightsquigarrow A_0'})$, $S_1 \in \Lang(\Aut_{A_0\stackrel{C}\rightsquigarrow B})$, and the definition of $\Pp'_{A_0}$, we know that $(p_1, S'_1) \xRightarrow{\Pp'_{A_0}} (p_1, [A_0'C]\cdot S) \xRightarrow{\Pp'_{A_0}} (p_1, S_1)$ for some $S$.

Therefore, 
$$((S'_1, \aname_1), (S_2, \aname_2), \cdots, (S_k, \aname_k)) \xRightarrow[\Mm]{} ((S_1, \aname_1), (S_2, \aname_2), \cdots, (S_k, \aname_k)).$$
It follows that 
$$(([A_0], \aft(A_0))) \xRightarrow[\Mm]{} ((S_1, \aname_1), (S_2, \aname_2), \cdots, (S_k, \aname_k)).$$
We conclude that the claim holds for $n$ in this case. 
\end{itemize}
}
 %%%%%%%%% the original texts removed
  %%%%%%%%% the original texts removed

\qed
\end{proof}